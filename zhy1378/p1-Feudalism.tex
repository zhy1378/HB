\chapter{封建篇}

写这篇文章,是因为当前国内的



\section{3+2推动}

\begin{enumerate}
\item 自愿
\item 利益
\item 暴力
\item 习惯
\item 氛围 
\end{enumerate}

久而久之,封建主义

听起来很简单,是不是?但可惜很多人、很多历史人物没有理解其精髓。

朱元璋设立了开历史倒车的户籍制度。他把人分为民户、匠户、军户。民户就世世代代务农

这种制度首先就忽视了自愿原则。

再从利益角度考虑,军户



元代统治者为便于强制征调各类工匠服徭役,将工匠编入专门户籍,称为"匠户"。子孙世代承袭,不得脱籍改业。至明改为轮班轮作,除分班定期服役外,其余时间可以自制成品出售,成为半自由的手工业者。清顺治二年(1645)废除匠籍,匠人才获自由身份。

有明一朝,实行的是户籍管理制度,当时的人按职业划分可大致分为:民户、军户、匠户。其中民户包括儒户、医户等,军户包括校尉、力士、弓兵、铺兵等,匠户分委工匠户、厨役户、裁缝户等。这些户的划分是很严格的,主要是为了用人方便,要打仗就召集军户,要修工程就召集匠户。看上去似乎也没有什么问题,但其实缺陷很大。

  尤其是军户制度更是一种非常操蛋的制度,明朝的开国皇帝朱元璋亲自规定,只要是当了军户的,除非是经皇帝特许或官至兵部尚书,否则任何人都不得自行改籍。比如你是军户,你的儿子也一定要是军户,那万一没有儿子呢,这个简单,看你的亲戚里有没有男丁,随便拉一个来充数,如果你连亲戚都没有,那也不能算完,总之你一定要找一个人来干军户,拐来、骗来上街拉随便你,去哪里找是你自己的事情。

  比起民户或者其他的职业,军户的地位更是十分地下,各级官吏甚至一普通生员都可以任意役使军丁,克扣月粮。有明一代﹐军户逃亡的现象十分严重﹐明朝的官府曾多次派人勾补逃军,甚至专门设有清军御史处理军户逃亡及勾补军伍事宜。明朝中后期以后,军户制度更是形同虚设﹐募兵渐渐成为明朝官军的重要来源。
  
将军的子孙会打仗 

\section{事儿多}

后来我才意识到,人生来便有极大的惰性。如果没有推动力,几乎所有的人都会惰化。

一个人只有做出了值得我尊敬的事,我才会尊敬他。他的地位、职业与“可敬度”无关。

大陆历史教科书上,认为社会主义是高于资本主义的社会形态。

