\chapter{乌托邦}

前文已经论述了共产主义的虚无缥缈,读者或许要问,我自己也一直在问:“最好的社会制度是什么?”

我的答案是:很可能是改良后的资本主义。

“资本主义来到世间,每个毛孔”

残酷的剥削第三世界,对本国的下层人民也毫不留情。欧洲的知识分子大声疾呼:“不,这不是我们想要的国度!”

现在的发达资本主义国家,是颇有点“温情脉脉”的色彩。

很多经济学家已经意识到一点:如果彻底消灭了富人,穷人会更穷。

共产主义的思维,是把社会总体财富视作定量。如果一部分人获取过多,则其他人必然受到剥削,因而陷入贫困。而事实上,社会总体财富是可变的。“为富有仁”的富人是财富的创造者,穷人只是从他们的财富中获得一点残渣似的恩赐,就足以使得生活大为改善。

穷人不能太穷,富人可以很富。

都是富豪,欧洲的比起美国的要小一号。

一般情况下,要保证基本需求。

“一般情况”,指的是没有战争、瘟疫、天灾、地震等不可抗因素。基本需求指的是食物、饮水和空气



\section{治标?治本?}
有不少网友说:“资本主义是治标不治本的。”
我不禁要问:“有什么制度是治标治本的?”
现实的看这个社会,

让社会上最有智慧的人来做决策
决策者无需顾虑个人的基本安全
在一般情况下,保证基本人权。 
保证社会阶梯