\newpage
\chapter{诗词集}

\zRef{这里收集一些本人写的歪诗。}

\section{张氏家族谱字歌}
本张氏家族准备续家谱,算是给家里老爹派了件大事。因为之前传下来的行辈用字(也叫谱字)已然用尽,故需要指定新的。老爹就把这个任务交给了我。

这种事我自然没有遇到过——一个谱字就够一代人用,续一次就是好几十辈,一个人一辈子能碰上一次就算运气好。我接到任务后就开始查资料。最有名的家谱就是所谓孔孟颜曾四姓的“通天”家谱,自孔子时代就一直传用。网上很容易找到弘历赏给孔府的三十个字:

\zBold{
希言公彦承 宏闻贞尚衍\\
兴敏传继广 昭宪庆繁祥\\
令德维垂佑 钦绍念显扬\\}

也很容易找到民国时期又续下的二十个字:

\zBold{建道敦安定 懋修肇益常\\
裕文焕景瑞 永锡世绪昌}

从孔子到弘历这两千多年间的谱字呢?很遗憾,我没有查到。

坦白讲,通天谱上这五十个字的组合水平挺一般的。虽然勉强可以押韵,但并无任何额外含义,所以很难记。除了跟最高统治阶级粘上点关系外,并无任何优异之处。

客观上说,谱字歌要写好不容易。谱字当然是要好字眼,要男性化,保证本家族的男丁能用谱字取个不错的名字。可惜这样的字并不多,我大量翻查资料后感觉达标的汉字也就三百左右——或许我该写个小程序统计一下。好字眼以形容词、动词、名词为主,连词就严重匮乏,想串起来不易;再者,美好的字眼摞成堆也成不了诗文,那叫“堆彻辞藻”,一个挺负面的词语;还有一个限制,用字不能重复。

在此三个条件的基础上编谱字歌,真好比是戴着脚镣跳舞,浑身上下都难受。

虽然难度很大,但还是有若干个家族用心良苦,指定了颇为顺口又有内涵的谱字。比如固始县沙岗保何氏的谱字歌:

\zBold{
世运除升平,祖宗创业成。\\
流长源自远,本固枝必荣。\\
道德为传训,永守望贤英。\\
留此三十字,阖族作派名。\\}

在谱字歌里面算是很上档次的了。

本人经过殚精竭虑的长期思考,苦恼于三字、四字还是五字,纠结于有了含义没了韵律,挣扎于有了韵律失了意境,翻来覆去、反复折腾若干小时后,选定本家族的谱字如下:

\zEmph{
敬长尊师 诚意正心\\
敏学尚知 明礼修身\\
博文广识 慎行谨言\\
勤睦家兴 常健寿延\\
秉公持义 为国维宪\\
昌盛有时 德佑久远}

歌成之时,颇为自得。此歌可笑傲谷歌、百度五十页(前五十页找不到更好的)!四十八个字,要够用一千年了。爽!

\section{柏发春}

\zEmph{又是一年好光阴,嫩红新绿染柏林。\\
碧水穿城东流去,微风轻雨鸟鸣颦。}

\zRef{如果你的思维足够发散的话,就会知道“柏发春”是“在柏林发春”的简称;如果你的思维再跳跃一下的话,就该明白“发春”是“发现春天”的缩写。}
