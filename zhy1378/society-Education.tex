\chapter{教育篇}

\section{素质教育}

素质教育,教的是什么?

我高考的时候,赶上了坑爹的大综合。

当然,录取人数还是

\section{高考与考研}

高考有它存在的历史价值

这样筛选出来的学生,具有什么特点呢?是听话。

历史证明,很多桀骜不驯的人有他的价值。他们更勇于探险 这样的一群人,常常会被排除在现今的高等教育体制外。

\section{过滤器}

有句话说得好,中国的考试是分配稀缺资源的手段。

找到最符合

假如我们需要顶尖长跑运动员,但选拔方式却是举重,那所有人都会觉得不合适。

目前的大多数人批评中国现行的教育制度,但少有人把眼光集中到选拔方式上。

以我所了解的计算机专业为例,郑州大学计算机系出来的很多学生编程能力都很差,不管理论课学的好坏。

\subsection{人的智力}

初中、高中时代,遇到的偏理科的学生不在少数。他们是学不好中国的政史外吗?不想学而已。

因为中国的政史教学不过是死记硬背罢了,

我并非要性别歧视,而是描述一个这样的客观事实:男孩更不愿忍受枯燥乏味的反复练习。


\section{事务的难度}

网上流传一个

事实上,培养一个围棋高手所需时间还要更少。目前成为围棋国手因为

竞赛与工作

竞赛性质的行业

举起250kg 跟249.5 kg,在生活中有什么差别?恐怕很小
但是会影响金牌的归属,以及附带的大量金钱和荣誉。

如果一个人有顶级

而工作性质的行业,只要达到一个某个水平即可

当然,工作行业也可能对人的天赋有要求。比如数学、理论物理研究,没有120以上的智力还是不要涉足的好,这些学科的前沿已经进展到一般的聪明人都难以理解的地步。

大多数工作性质的事务,一般人的天赋只需要一万小时的努力就可以很好的胜任了。

很多事务,之所以我们会更信任年龄较大者,主要是有个“资质考察”的过程。

作家Gladwell 在《异类:》一书中指出:“人们眼中的天才之所以卓越非凡,并非天资超人一等,而是付出了持续不断的努力。只要经过1万小时的锤炼,任何人都能从平凡变成超凡。”他将此称为“一万小时
定律”。 

一个事务,到底有多难?

比如做个清洁工,几十个小时的培训足以让一个人成为高手。

\subsection{潜知识}

但是依据我们的经验,在很多时候,确实是年龄较大、从业时间较长的人干活比较好
这往往意味着,这些行业内有很多潜知识
“明知识”是做成教程,大面积培训出去的
“潜知识”是只有在从业之中遇到了,才能学到的。
因此,在这些行业中,从业时间长遇到的潜知识比较多,

在软件业中,潜知识少,明知识多,因为这个行业是崇尚开放、交流和自由的。于是,



“党和国家培养一个干部不容易”

不容易吗?

找个合适的官容易极了。

不但是当官很容易,

就按120 的智力算,

如果一个人的智力还不错,那么

甚至包括写程序。

大学时候,每学期一般就学1~3周。

我在网上跟人进行过无数次的辩论,

论据和论述过程的选择还有问题。

中国的政史教育中,常常划分出来明显的对错。

另外说一句,

“公民权利”研究论文

要求:在3到5页纸之间,打印出来,要双空行,至少用3种资料来源(如网上,书籍等),至少有5句引文。

对比以下四人关于黑色美国(BLACK-AMERICA)的观点:BOOKER T.WASHINGTON(布克·华盛顿),W.E.B.DUBOIS(杜伯依斯),MARTIN LUTHER KING,Jr(马丁·路德·金),MALCOLM X(马尔科姆·X)。

在你的论文里,应该控制关于他们生命的故事,我不想读传记。但是,需要把每个人介绍一点,还必须纳入贴切的材料在你的论文中。然后,讨论他们关于黑色美国的观点,要把你的想法写进去。还要把你的引文或材料的来源列出来,比如某某网页,某某书。

关于南北战争:

1.你是否同意林肯总统关于美国不能存活除非它全部解放或全部奴役的声明?解释。

2.解释为什么北方白人反对奴隶制,南方白人拥护奴隶制,但他们都感觉他们在为自由而战?

3.自由对于黑人意味着什么?

4.林肯总统和格兰特将军表示在内战后,南方不应被粗鲁地对待。为什么这是一个聪明的政策?解释。

5.在内战期间,女人开始担任很多以前男人的工作。你能对由于内战造成的社会、经济和政治冲突的问题做出怎样的概括?

构造一个争论,运用历史证据来支持或反对下面的观点:美国内战是地区差别不可避免的结果。

菲律宾问题:

1.什么样的美国人可能会同意Josiah Strong的“我们的国家”?什么样的人会不同意?他们为什么会同意或不同意?

2.Bryan如何将“帝国主义”同获得西班牙领土联系起来?你认为他联系得对吗?为什么?

3.Lodge对获得菲律宾这件事的辩词是如何反映了美国的传统政策的?

4.你认为有比麦金利总统以控制菲律宾来处置菲律宾的命运更好的选择吗?


两害相权取其轻、两利相权取其重

生育率不足、

与非父母抚养孩子的

平安夜

堵?堵得住吗?

圣诞节、平安夜,以及其它的西方文化

我们身上的西方文化还少吗?住的房子、开的车子、穿的衣服、走的道路、政治结构、现代汉语无不是受西方文化影响的结果。

为什么偏偏要对圣诞节喊打喊杀呢?

我想,这是宗教的因素。

然而,今天的基督教,早已不是两千年前的基督教

当今世上,倒是有一个原教旨教派,忝列世界几大宗教之中,假以时日

未知其底细者,常敬佩于

这个民族人均产生的科学家

\section{有限的智力}

如果人的智力潜力无限,那很多

但如果人的智力潜力有限,

比如,人的大脑中,不该装入太多垃圾信息

比如,教育中,觉得人的智力无限,就可能会施加过高的压力

\section{熊孩子}



\section{智力的差异}

好啊,学习方法。

简单题目不做,只做难题。上课

欢迎推广,看看能把多少学生领到沟里。

运动神经的差异

但后来越来越发现我运动神经确实不太行

甚至在打游戏中我也可以感到,