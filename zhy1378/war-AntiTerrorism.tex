\chapter{反恐篇}

2014年,德国最关注的国际新闻是乌克兰局势和ISIL。\textbf{ISIL} 是英文\textit{Islamic State of Iraq and the Levant} 的缩写,中文全称是\textbf{伊拉克和黎凡特伊斯兰国}。

虽然名字里带“国”,但其实是个组织。

美国是一个有底线的国家

我站在美帝的角度,反对美帝的两个做法:
1. 大批训练当地士兵 2. 撤军后大量军械就地销售
美帝这两个做法的原因当然是可以理解的,但是后患无穷

训练当地士兵是为了避免让美国大兵冲锋上阵,美国大兵的命金贵呀,死一个人就得付出100万美元有余(子女数量和配偶寿命会影响最终发放金额,具体数据待查证)。

撤军后的军械如果要运回本土,又是一笔开支,不如就地销售,能收回不少成本。

2014-02-03,“基地”组织发表声明,表示自己与ISIL 毫无关系,称其不是基地组织的分支机构,对他们的行为表示不满,并要求他们停止活动。原因不仅由于ISIL手段过于极端,“基地”组织目的只是袭击欧美,对教派斗争没兴趣,而ISIL 则要求建立政教合一的逊尼派国家并消灭什叶派。

比“基地”还要极端,

军事科学是非常复杂的,如果肆意把这些技巧传授给有暴力传统的种群,就会大大提升他们的暴力水平,最后搬起石头砸自己的脚

反复轰炸还有一个害处,ISIL 的士兵会越来越有经验。

普通人第一次遭遇轰炸会惊慌失措,

可以想象,几个月以来,ISIL 的防空洞越挖越深了。

造炸药没那么难,但是造出高质量的炸药就难了。如果所有的工业化国家都向ISIL 关闭大门

我的策略可以归纳如下:
\begin{enumerate}
\item 用较高教育水平的人从军,保持军队的素质。
\item 对恐怖分子减少空袭准备时间,在密集空袭后迅速出动地面部队,避免恐怖分子在战争中成长。
\item 执行严格的武器禁运,甚至打击有可能转化为军工的化工产业,防止恐怖分子获取现成武器和自制炸药
\end{enumerate}

迟迟不出动地面部队的原因当然我也了解,美帝惜命。但是不出动地面部队就不能实现实际占领

快速击毙,不给恐怖分子

军事技能也是需要训练和学习

我建议的措施如下:
\begin{itemize}
	\item 收缴枪支和炸药,能销毁的一律销毁,对外称之为“冲突地区非军事化”
	\item 当地警察一律不配枪,只配电击枪和警棍、防暴盾牌
	\item 各地派驻美帝特警。如果遇到危险情况,临时给警察配发手枪,并且由特警担任主攻
	\item 全面封杀ISIS的网络宣传
\end{itemize}

我提出这个想法的时候,当然在网上也是遭到了一片反对。

这对于欧美国家来说,简直是不可完成的任务。

两男子A与B争吵,突然,A拔出手枪,一枪将B击倒。B的女友C见状上前查看B的伤势,并与A厮打。A怒,欲枪击C,但枪卡壳了。C躲开了一下,然后又回头厮打,直到A重新装弹,又将C击倒,近距离直击颅骨。然而神奇的是,C没死,又爬起来了。

整个画面持续约两分钟。如果是在美帝,第一声枪响的时候,C就该跑的远远的,并且大呼“报警”。

其一,欧美国家的军火工业具有强大的话语权。美国到现在都不禁枪,即使每年的枪击案一打一打的。

这世界上有极多很SB又很难反驳的理论。


