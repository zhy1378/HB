\chapter{碳排量}
柴静的视频

柴静在碳排放上的立场:柴静和中科院院士丁仲礼和钱维宏的辩论

  柴静一直坚持碳排放要以国为单位,丁院士坚持碳排放应以人均为单位,分别代表西方发达国家和中国的立场,泾渭分明。

  第33分钟:丁院士问柴静:中国人是不是人,为什么你洋人要消耗一个中国人四倍的碳排放量?
 笨笨 
第36分钟:丁院士悲愤的质问按IPCC的方案中国2020年后每年要花1万亿人民币购买碳排放权是否公平。柴静不敢接话,反问丁院士作为一个科学家说话时用激烈的带情绪色彩的语气会否合适
 笨笨 
第40分钟:柴静谴责丁院士,说科学家不应该关心参与政治,不应该以国家利益为出发点而应该以人类利益为出发点。丁院士说:我为发展中国家人民争取生存权发展权和联合国的人类千年发展规划一致,难道不是以人类利益为出发点?
 笨笨 
丁回答说人类拯救地球是伪命题,人类需要拯救人类自己,地球不需要拯救
 笨笨 
柴静和她的小伙伴们都被惊呆了

在我看来,丁仲礼站在中国经济的角度说话,没有错;柴静站在人类全局的角度,更没有错。辩论双方都没有错,因为这是个无解的论题。

发达国家的排放量

说的再夸张一点,如果发达国家真的不随便消费了,发展中国家也会跟着倒霉。因为发展中国家全都指望着富国来拉动经济。

地球变暖都不用仪器检测,用皮肤都可以感受出来。这几年的冬天比我小时候暖和多了。

无需遮掩,发展中国家普遍腐败蔓延。

整体技术进步

什么地方可以保持山青水秀呢?

人迹罕至
穷的没裤衩
后工业化

如果再考虑

人均排量是无法解决问题的。发达国家的收入在那摆着,但又不可能禁止发展中国家向前看
2015-03-02 16:04
或许有效的办法是,对每种工业单独设置排量限制,每种工业品单产的碳排量限制。
但其技术复杂度是无法解决的