\newpage
\chapter{古都篇}

\begin{quote}
本来这篇文章想拿去投稿的,所以写成了论文的形式。后来懒得投了,主要是一些细节上的考据做起来太累。我自认为,创新性、严格性都是具备的。
\end{quote}

\begin{abstract}
本文旨在解决“几大古都”是比较合理的说法。网上流传有“四大古都”、“五大古都”
甚至还冒出来一个“中国古都学会”。
\end{abstract}

基本规则如下:


以下地区,合并起来算1分:
\begin{enumerate}
\item 陕西
\item 山西
\item 河南
\item 河北,含河北、北京、天津
\item 山东
\item 安徽
\item 江苏,含江苏、上海
\item 湖北
\item 湖南
\item 四川,含四川、重庆
\item 云南
\item 贵州
\item 广西,含广西、越南
\item 广东,含广东、澳门、香港、海南
\item 江西
\item 福建,含福建、台湾
\item 东北,含东北、库页岛、外兴安岭等地
\item 蒙古,含内蒙古、外蒙古
\item 西藏,含西藏、青海
\item 新疆,含新疆、甘肃、宁夏、中亚
\end{enumerate}

超级短命王朝不计算

西安得分
西周:7分,陕西、山西、河南、河北、山东、安徽、江苏,约前1046年-前771年
东周:1分



北京
明朝:18分
清朝:全20分,


\subsection{洛阳}

\textbf{五代后唐}:5分
13年,后唐(923年—936年)
共计52分
河南 山东 山西 安徽 
河北 江苏的一半合起来算1分


\subsection{南京}
南京是六朝古都
东吴
东晋
南朝宋
南朝齐
南朝梁
南朝陈

\subsection{开封}
共计得分2602

\subsubsection{东周魏国}
1分 河南
146年,146分
公元前361年,魏惠王从魏县迁都大梁,即开封
公元前225年,灭亡

\subsubsection{五代后梁}
3分 河南 山西 山东
907年-923年,历时17年
3 × 17 = 51分

\subsubsection{五代后晋}
4分:山东 山西 河南 陕西
936年-947年  12年
4 * 12 = 48分

\subsubsection{五代后汉}
947年-951年,存在仅四年,不予计算

\subsubsection{五代后周}
6分
河南 山东 山西 陕西 湖北
安徽 江苏的江北部分,合计算1分
951年-960年,10年
共60分

\subsubsection{北宋}
13分,1127年6月12日—1279年3月19日,167年,13 * 167 = 2171

缺少东北、蒙古

\subsubsection{大金}:1214年-1232年农历十二月廿五日
东北、河北、河南、山东、山西、安徽、江苏

7分,18年,7 * 18 = 126分


\subsection{杭州}
南宋:146年,1130年夏-1276年2月4日

\section{总结}
本文用翔实的论据说明了“五大古都”是比较合理的说法。由于本人非历史学相关专业,故此对于细节上的考据难以做到翔实充分。

可以预见的,还有以下两种积分方法:
\begin{enumerate}
\item 基于统治人口的积分法
\item 基于财政收入的积分法——为了便于跨时代比较,将财政收入换算成粮价。
\end{enumerate}

本来中国的行政区划,是将地理、气候、文化相近的地区归为同一个行政区。但目前的省级行政区划,是元朝基本成型的。元朝统治者为了防止占人口大多数的汉人起义,将原有的行政区划劈开,
在此基础上进行积分式统计,难免有不足之处。