\newpage
\chapter{红裙子}

\begin{quoting}
这是我第二篇小小说。
\end{quoting}

小欣幸福的生活在乌托邦。这里又富裕又宁静,是全世界人民都向往的国度。据说在其它国家流行一句祝福语:“你真是个好人,下辈子乌托邦。”

小欣也觉得自己很幸福。接受过乌托邦基础教育之后,她像其他孩子一样接受了最科学的职业选择评估。看过她的数学成绩之后,和蔼的职业鉴定师建议她不要选择科学类工作。

经过几轮筛选之后,小欣成为餐厅的服务员。

\zPar
在餐厅工作挺好的,就是有点无聊。餐厅旁边是服装厂,每天都有大量的碎布被拉走。小欣就问在服装厂工作的晓蓉:“这里的碎布我能拿一些吗?”

“随便拿,都没什么用了。”

\zPar
小欣从小很喜欢手工,于是她就捡拾碎布,做了一件红裙子。

“哎呀,小欣,你这件裙子真漂亮。这件裙子的申请号是多少?”刚出门,她就遇到了邻家姐姐小兰。小兰绕着她看了两圈,赞不绝口。

乌托邦是按需分配,想要什么只要申请就行了。如果调度中心核准,很快就可以拿到。

小欣听到夸奖也很高兴:“谢谢夸奖。这是我自己做的。”

“啊,你真棒。你把这件裙子让给我吧,我用物品跟你换。嗯——”小兰想了一下说:“我用一盒白松露换。”

“白松露!这类珍贵食材可不容易申请到。”小欣马上就答应了下来。她很馋,虽然她一直在克制,并且从来都不好意思承认。

\zPar
一周之后,小欣又做出来一条红裙子。这次,她刚穿着出门,就被小梅和小丹同时看到。两人围着小欣夸赞了一会儿之后,都想得到这条裙子。

“我用文竹跟你换,反正我有很多。”小梅家里养了好多文竹,都是她的宝贝,没想到她愿意拿出来交换。

“我用笛子跟你换,反正我好久不吹了。”小丹有一条很好的笛子,她以前是业余乐团的。小欣只能申请到普通的笛子,因为她吹奏水平很一般。

小欣觉得很为难,她权衡了一会,选择了交换笛子。但是小梅很不高兴,“我以后再也不理你了!”小梅离开的时候说。

小欣觉得很内疚,于是她连着两天熬夜,做出了一条更好的裙子,交换了小梅的文竹。

\zPar
晓蓉找到小欣,想跟小欣学习服装制作,小欣愉快的答应了,乌托邦的公民应该共享技艺。

晓蓉每天晚上都来找小欣学习,但是她的脸色越来越不好。“哎,我很想做出最好的衣服,可我的衣服总是不符合设计师的标准。”晓蓉无奈的说。

晓蓉的厂里有三个设计师,所有的衣服样式都要经过他们的审核。

小欣不知道说什么好。她觉得晓蓉的手艺很好,但既然专业的设计师不认可,那一定还有不足之处。“毕竟我们是业余的。”小欣想。

\zPar
渐渐的,小欣家里开始拥挤起来,各种各样交换来的物品堆满了她的房间。

终于,阿珰敲响了她的房门。阿珰是乌托邦的管理员,据说他是永远伟大光荣正确的。

“小欣,你的财富好多啊。”阿珰站在门口,打量着小欣的房间,他看到了两架钢琴、三个沙发、四盏大吊灯、五台电脑……

小欣有点局促不安,她拦在阿珰面前,想要挡住他的视线,但是她做不到。因为从来没有人能阻挡阿珰的视线,他无处不在。

接着,阿珰用商量的语气说:“小欣,你家里堆的满满的。你看是不是该上交一部分,保持乌托邦的财富均匀?”

小欣也觉得自己拥有的财富太多,但是她又舍不得交上去。于是她小心翼翼的问:“可不可以不交?”

阿珰想了一会说:“当然是可以的,乌托邦不可以强制收缴公民的合法财产。”

“那我就不交了。”小欣松了一口气。

“但是……”阿珰想要说点什么,又觉得不知道说什么好。

\zPar
晓蓉去西国旅行了一个月。西国是个不幸福的国家,那里的人富的极富,穷的很穷。

她回来之后,告诉小欣说,她要移民到西国了。西国的资本家愿意给她很多很多金钱。

小欣大吃一惊,从来都是西国人想方设法移民到乌托邦,想不到会有乌托邦人移民到西国。但乌托邦是自由的国度,移民是合法的。

临走时,晓蓉突然抱住了小欣:“跟我一起去西国吧,你的才华比我强得多,西国的资本家愿意给你十倍的金钱。”

小欣的心里痒痒的,但是她不想走,她爱乌托邦。

\zPar
阿珰又来到小欣家里,开门见山的说:“经过乌托邦代表们讨论,决定为你成立一家工厂,你就是厂长。以后你们可以在工厂里上班,用最新最好的设备和最优质的布匹,不需要费劲收集布头了。”

这就意味着她可以整天从事自己喜欢的行业,不用熬夜了。但是,生产出的衣服也就不能用来随意交换。

小欣工厂开始运转了。没过多久,工人小荷找到小欣说:“我想加班。我组里的姐妹都想加班。”

“我们全乌托邦都是四十小时工作制,没有例外的。”小欣很为难,她拿不定主意:“而且这些设备属于乌托邦,让你们私下里用似乎不太好……” 

“咱乌托邦所有财产属于人民,我们也是人们啊。要不这样吧,”小荷跟小欣商量:“每天晚上我们额外生产三件衣服,一件归乌托邦,两件归我们自己。”

“那就……好吧!”小欣觉得这也没什么不好,反正是为乌托邦创造更多的财富。

\zPar
阿珰再次来到小欣工厂的时候,脸上有些阴沉:“我发现你的工人在业余时间的生产效率比工作时间高了20%,而且产品质量更高。”

小欣觉得很不好意思:“我也不懂怎么回事,好像大家下班之后突然有了力气。”

阿珰说:“这样不行啊,乌托邦的人民必须保持平等。你们要重新学习乌托邦精神!”

\zPar
一个“乌托邦精神学习班”在小欣工厂的会议室成立了。所有的工人都要在下班后参与半小时的学习。阿珰亲自担任学习班的教师。他是个优秀的演说家,每次演讲都声泪俱下。

他先回顾了奴隶制下奴隶主对奴隶们的生杀予夺,又讲述了封建制下地主对雇农的残酷剥削,接着展示资本主义西国的血汗工厂。

\zPar
然而,过了两个月,阿珰的脸色愈加阴沉了。

所有的女工都学会了织毛衣和编花,她们可以一边盯着阿珰一边做活儿,根本不会耽误。

阿珰禁止女工在学习乌托邦精神的时候干活。

\zPar
接下来发生的事情让阿珰更为抓狂。

一个叫阿豆的家伙发明了一种防噪音耳塞,可以良好的隔绝声音。更神奇的是,这种耳塞能接受脑电波控制,可以在不知不觉间开启或关闭。而且,所有参加学习班的女工都有一副。

乌托邦大会上,大家众说纷纭。有人觉得加班没什么不好,还有人觉得小欣大大破坏了乌托邦的平等。最后,乌托邦大会通过了一项决议:简单的就是最好的。

如果所有人穿的衣服都一样,也就没有高低贵贱和不平等了。小欣工厂将只能生产指定的样式。

\zPar
从乌托邦大会回来,小欣突然觉得很累。

她穿上自己第一条红裙子,这是她从小兰手里换回来的,留作纪念。她倒了一杯红酒,在镜子前跳了一支舞,然后打开电脑,订购了一张到西国的机票。

“或许这里真的不适合我。”她想。 

