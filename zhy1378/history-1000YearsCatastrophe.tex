\chapter{千年劫}
袁腾飞曾经对比明清皇帝的治国水平,在他看来,明朝皇帝基因和家教都很低劣,大多数都是某种程度的混蛋;而清朝皇帝个个英明神武,读书时间超长,连最差的几个皇帝都有可取之处。“幸好”是清朝时候遇上了西方入侵,如果是明朝时候,中国就彻底沦为殖民地了【缺少引用】。

本篇将会尝试回答这样一个脑洞大开的问题:如果明清在时空上互换,谁能更好的应对1840年之后的大变局?
如果换成汉、唐、宋,又会如何呢?

如果我们仅仅对比明清皇帝的个人素质,答案就会非常简单:明朝惨败!而且双方的差距不是一般的大。
问题就这样简单么?

\section{新文明}
如果西方真的想要灭亡中国,满清能抵御吗?

看鸦片战争的进程吧。英军只有1.9万人,其中还有很大比例的印度兵。满清动员了30万人,但从未取得过哪怕一场局部战役的胜利,也从未给英军造成重大杀伤。英军战死只有69人,其最大的损失来自风暴和瘟疫,但也只死亡了不到一千【缺引用】。

可以说,满清对英军的战绩,比明军对满清的战绩要惨淡的多。明清战争还停留在冷兵器中期的水平,而第一次鸦片战争则是热兵器晚期军队对冷兵器中期军队的碾压。\footnote{中国的技术水平有多次反复,中国的战争水平曾经进步到冷兵器晚期和热兵器早期,但明清战争时又退化到了冷兵器中期。更多内容,见《战争篇》。}

如果英国想要彻底灭亡中国的话,满清根本无法抵御。

\zPar
但是,英国的思维模式与北方少数民族完全不同,它并不打算占领中国全境,而是选择了保全满清政府、跟中国做生意。

当然英国也狠狠的从中国身上捞了2100万银元——然而这笔钱并不算很多,尤其是比起后来几个条约的赔款。

之前,中国屡屡遇到的问题是,先进的农业文明被落后的野蛮民族袭扰,中国即使击败了野蛮民族,也很少能从他们身上学到什么;而一旦被野蛮民族击败,就是一次文明的大倒退。

但这一次,中国遇到了先进得多的西方工业文明,它们身上有太多值得学习之处,为什么中国没有像日本一样迎头赶上呢?

\section{启民智}
\textit{1900-06-23 北京}
一场大火把翰林院烧个精光,同时化为灰烬的,还有附近的铺户1800余家,民居7000余间。\footnote{《中国翰林院被焚真相》 \url{http://bbs.creaders.net/education/bbsviewer.php?trd_id=41138}}

放火的是义和团,而放火的目的是进攻北京东交民巷外国使馆区。使馆区内大约有三千人,其中有约两千寻求保护的华人,六百多非战斗人员,士兵只有409名\footnote{维基百科:八国联军。维基不规范,以后再查证。}。他们的重火力只有三挺机枪和四门小炮,却挡住了试图进攻使馆区的几十万义和团和清军。

翰林院紧挨着英国使馆。它是当时世界上最大最古老的图书馆,储藏着包括七万九千卷的《四库全书》的大量底本和两万卷的已成孤本的《永乐大典》。这座木制为主的建筑和其中的大量书籍是巨大的易燃物。

由于使馆区屡攻不克,“董福祥率领的回教徒们”\footnote{董福祥的甘军,主要来自青陕甘一带。}点燃了翰林院。这次大火并非偶然,因为据记载,“他们有条不紊地一个庭院一个庭院地烧”\footnote{《围城北京》,The Siege At Peking [英]傅勒铭,P121}。

烧毁了翰林院,使馆区却还是攻不下来,因为英国使馆“墙厚八尺,高二丈有奇”,“其大可容千人”。于是义和团决定采取围困战术。

神奇的事情又出现了。义和团围困使馆区的时候,慈禧一面派人抚慰义和团,另一面偷偷的往使馆区内送粮食和弹药。
\zPar

这些义和团“拳民”,主要是山东、直隶(京津冀地区)的农民。他们的“大师兄”(领导人)宣称练功之后可以成“金钟罩”,达到“刀枪不入”的效果。
看到这里,略有军事常识的现代人都会提出疑问:“符咒”怎么可能抵御枪弹?
当然不可能!在义和团“竞冲头阵”的时候,“联军御以洋枪,死者如风驱草”。
那又要问:这种事怎么可能反复发生?难道团民吃亏不长记性?
秘密全在吞下的符咒上!“画符用的红砂是兴奋药做的,喝过符一小时内,心神烦昏,光想打仗。一个时辰过了就没事了。” [3]
无独有偶,伊拉克反美武装分子为了对付美军,发起进攻前都要服用一种药物,吃后感觉自己像“超人”,对美军毫不惧怕。

慈禧派了两个大臣去检验义和团是不是真的刀枪不入,这两个大臣回报“是真的” 。于是慈禧就任由义和团闹起来。
如果想从理性、理智的角度分析慈禧的行为,那是无论如何都想不通的。只能把她作为一个赌气任性的小女人:她对洋人不爽,所以她要想办法报复。至于会产生怎样的后果,她才不管。
义和团疯狂起来,就不按清政府的想法办事了。义和团对洋人是不分男女老幼的屠杀,对西洋文明产物是不管青红皂白的破坏 。然而死于义和团奸淫掳掠的,主要还是中国人。并且有一些清军也加入了他们的行列。

随着义和团运动的失控,外国势力介入已然是难以避免。慈禧便试图耍一些两面派的手法,偷偷援助被困的使馆区,做出一副“义和团是屁民暴动,我中央政府跟他们不一伙”的姿态。
晚了!八国联军已经向北京进发,带着第二次工业革命中发明的武器,击破清军和义和团的防线如同热刀切黄油。
唯一给八国联军造成一点麻烦的是淮军将领聂士诚。但聂士诚在朝廷里是反对义和团的。于是,义和团不仅在聂士诚与八国联军作战时偷袭其后方,还屠杀了聂士诚全家老幼 。
在糟糠一样的清末军队里,聂士诚可称名将,他在甲午战争中抵抗过日本。在他战死之后,慈禧再也没胆量把她小女人的小聪明玩下去,化装成小脚老太太逃到了西安。

接着,便是八国联军对北京长达一年的占领。《辛丑条约》的谈判过程中,慈禧的指示是,只要能归还北京、保住满清帝位,对各国要求全部答应、要啥给啥【缺少引用】。
最终,签署了“每个中国人赔偿一两白银”的羞辱性条约。但此前的《中日马关条约》已经把满清彻底掏空,这四亿五千万两从何而来?
清政府只好找外国银行团贷款,并把赔偿期限延长到30年。最后总赔款本息为9.82亿两白银。

义和团就是这么一场闹剧转悲剧。它兴起的原因是宗教矛盾,即基督教徒和其他农民的矛盾。
其根本原因在于无知和愚昧。不了解西方宗教,不了解西方文明,不了解现代科技,不了解国际局势,只是一味的排外。
要知道,此时距离1840年已经过去了60年,义和团的重灾区山东、直隶并非中国最落后的地区,更确切的说是排名中上的两个省,为何民众如此愚昧。
中国人一直愚昧,但愚昧到这种程度,不是太令人惊诧了吗?


当时的关系可以这样理解:清政府是土财主,义和团是土财主养的二傻子,

慈禧的行为表明,清政府从来都是把义和团当作一条狗。这条狗狂叫乱吠,仿佛可以赶走家门口的壮汉。于是,

僵持数日后,
义和团试图采用火攻。但英国使馆

究竟是中国人民组织义和团跑到欧美、日本各帝国主义国家去造反,去“杀人放火”呢?还是各帝国主义国家跑到中国这块地方来侵略中国、压迫剥削中国人民,因而激起中国人民群众奋起反抗帝国主义及其在中国的走狗、贪官污吏?这是大是大非问题,不可以不辩论清楚。 

三千人
它的前身“大刀会”兴起于1894年,1898年闹起了义和团,它的口号是“扶清灭洋”。义和团的
1900年,义和团攻击北京各国使馆。

当时的欧美文明,如果贫弱不堪,那它们很乐意“宰肥羊”。如果该民族愿意学习先进,从派遣留学生、考察团,到翻译书籍、普及教育,甚至到购买先进武器,欧美各国都很乐意合作。
西方传教士既是殖民先锋,也是许多落后民族的启蒙人。



这大概
愚昧啊,愚昧。

那么问题来了,启民智,怎样启?

主要有两个路径:教育和新闻。

教育又可以分为基础教育和高等教育。基础教育要兴建学堂、消除文盲,普及地理、历史、法律和科学常识。高等教育要外派留学生、改革科举制、建立新式大学。

新闻要放开新闻和出版业限制,翻译外国书籍,鼓励自由办报和出书。

如果1900年的义和团普遍有初小水平(小学三年级,能认字),配合一些介绍西方的媒体宣传。

\subsection{外派留学生}
学习外国先进技术必须通过外派留学生的方式。有些复杂的技术只能由年轻人来学,年龄大了就学不成了。
	留学生涯对人有多大影响呢?
	比较著名的
TG建国领导人里
党史有所了解的
留苏的一群人往往是主张计划经济的。注意,大跃进、文革时期的经济甚至不是计划经济,只能算“胡搞经济”。
没有留学经历或者仅仅留学苏联的,

大致可以说,留学过西欧和美国的,主张
而极左的一批人,


	当我对清末历史有了进一步的了解后,我发现了一个恐怖的事实:非常遗憾的是,满清统治集团既不聪明也不傻。这句话听起来比较奇怪,却恰恰表达了我的意思。满清统治集团既没有聪明到认识差距、学习先进、锐意改革,也没有傻到可以被轻松推翻。
满清统治集团的政治才能像是为了中国的千年劫难而量身打造:既足以左支右绌的压住国内的反抗,又愚蠢到可以被列强随意摆布。

于是乎,从1840到1900,满清浪费掉了一个甲子。
60年啊!这可是工业革命之后的60年!
这60年间的革命性技术发明有:
•	硫化橡胶:此后橡胶才变得实用
•	塑料:全新的材料
•	内燃机:比蒸汽机更轻、更小,功率更强大
•	发电机、电动机、电灯:完全改变了世界
•	火棉、黄色炸药、无烟火药:加速了开矿和修路,使大炮可以打到视界之外,枪弹射程猛增
•	机枪:射速由每分钟发迅猛提升到发,带来了军事战术上的剧变 
•	电报和电话:通讯方式的剧变
•	照相机、原始电影:传媒方式的剧变
•	射线:新的检测方式
•	元素周期表:全新的角度研究化学

第一次鸦片战争时期,义和团式的“无脑冲锋”或许还能造成一些杀伤;而在八国联军时期,他们不过是枪靶子而已。
1895
错过了第二次工业革命之后,
“庚子赔款”使中国背负了四亿五千万两白银的包袱 ,就这样仍然磕磕绊绊过了十年。




义和团的口号“扶清灭洋”也颇有值得玩味之处。假如某个团体的口号是“扶中”的话,你一定会觉得这是个外国人的对华友好团体。如果义和团认定自己是“清”的一部分,那就不应该“扶清”,而是“兴清”。


对于满清统治者来说,只要民众开始思考,就是危险的信号。所以,民众最好什么也不想,进一步说,民众最好什么都不敢想,什么都不会想。

然而这是不可能的,就如同,勇敢的人更喜欢
培养出强悍的国民,就别指望他们在国内驯服如绵羊
培养出爱智求真的国民,就别指望他们对不合理的政策

\section{满清文字狱}
从字里行间吹毛求疵的挑毛病,

西方也有类似的情况,但我
满清统治集团想要通过文字狱让“屁民”觉得
无论怎样思考都是错,思考什么都很危险,

龚自珍有“避席畏闻文字狱,著书都为稻粱谋”

其他触犯庙讳、御名以及提到皇帝应该换行抬写而没有换行抬写,因此获罪的,不可胜数。有一个河南人刘峨,编印《圣讳实录》一书出售,本来是为了告诉人们应当怎样避讳,所以把应避讳的清代诸帝的名字\textit{各依本字正体写刻},却被控大不敬,惨遭斩首。 

某大师回忆道,他小时候念私塾
就如同现在的中国,不管
奴酋弘历还编写四库全书,宣布华夏只有三千本书是可以存在地,禁毁而留书名则有近七千本,至于禁毁而不留书名地更是不计其数——天啊,不要说煌煌两千年华夏。仅在明朝、仅天启皇帝批准刊行地书籍就有两万余本。
《苏报》案是满清最后一场文字狱,
此案能够“轻判”,主要是

	清朝的原罪使得满清统治集团对于“启民智”具有发自内心的恐惧。如果换成一个汉人朝廷,我不能说问题不存在,但症状要轻得多。
三、兴民权
1911-05-08
	满清政府迫于压力,宣布实施“责任内阁制”,裁撤旧内阁及军机处,成立由十三人的新内阁,以庆亲王奕劻为总理大臣,那桐、徐世昌为协理大臣。十三人中,满洲贵族九人,汉族官僚仅四人,而满洲贵族中皇族又占七人,史称“皇族内阁”。
纯属扯淡!我也不会接受这样的议会。

	人如果处于极其糟糕的环境中,心里想的是“还能更糟一点吗?”后来的事已经管不了了,先摆脱当前的环境再说。
	“皇族内阁”就是一个屎坑,当时的中国人已经顾不了以后会怎样了,哪怕是前刀山、后火海,左龙潭、右虎穴,也得先跳出来再说。
	当年十一月,辛亥革命爆发;次年二月,宣统退位,满清寿终正寝。

	为什么清朝对于君主立宪一直推三阻四?要知道,君主立宪并非废除皇帝,而是限制皇权。这是一种比较温和的改良。因此,君主虽然对权力受限而很不爽,但并非完全不可接受。
君权越大的国家,越难以限制君权。

( 1 ) 满清统治经验
应该说,满清统治集团是很精明的。他们虽然对西方经验全无了解,但已然对中国大陆上发生的历史了如指掌。
中国历史上王朝覆灭的主要原因有六种:后宫、强敌、天灾、强藩
清帝
满清的政治制度是高度集权的。
这一点不得不说,清帝整体来说还是很勤政的。顺治、康熙、雍正、乾隆的个人才能都算不错,
历史经验
1.  后宫
隋文帝杨坚本来是外戚
吕后
2.  强敌
唐、宋、明、
3.  天灾
明末处于“小冰河”时期,连年灾荒。与此同时,当时还是部落形态的后金逐步增大东北的军事压力。
如果满清也碰到了恐怖天灾,它有办法吗?够呛。但它运气好,没碰上。
4.  农民暴动

如果明朝的君权像朱元璋、朱棣时期一样,那确实是
( 2 ) 明朝辅臣
明朝从1368~1644,国祚276年。放在中国历史上还是比较长的。虽然明朝的昏君层出不穷、花样翻新,但既然明朝可以延续这么久,就一定有其它的力量支撑。
这个力量就是辅臣,差不多相当于宰相。明朝辅臣的作用是非常大的。

如果换成明朝,这个过程会容易很多。明朝从第三个皇帝明仁宗朱高炽开始,辅臣的势力就非常大——因为皇帝经常是笨蛋、懒蛋或者笨蛋加懒蛋。
明世宗嘉靖帝,
明神宗万历帝,
天启帝,爱好是木工,跟“锁匠国王”路易十六、“画家皇帝”宋徽宗、“词人皇帝”李煜是一个档次。魏忠贤经常在天启帝忙于木工活的时候找他,天启帝会很不耐烦的回一句:“没看朕忙着呢?你自己看着办。”

也真就奇怪了,虽然明朝皇帝颇多幺蛾子,其宰辅忠臣普遍都还行。

应该说,清朝是中国历史上君权最大的时期,没有之一。
这要是换成明思宗万历帝
这爷三
要是天启皇帝朱由检呢?更没关系了,这哥们喜欢的是木匠活。
噻,当什么大事儿呢,不就立个宪吗?没关系呀,反正我也不喜欢上朝。

所谓兴民权,指的是两个层面:
在中央,用君主立宪限制王权。更进一步的,实行三权分立。
在地方,建立各级议会,参与地方事务乃至选举地方长官

对于满清来说,它的心里过不去这两个坎

一旦地方议会兴起,那占绝对多数的汉人肯定会选举汉人官长。失去控制怎么办?

如果是宋朝呢?更好说了。两宋是知识分子最幸福的朝代,祖训是“不杀士大夫及上书言事者”。而且宋朝的宰相权力很大,比较有名的故事是,王安石跟皇帝争辩,唾沫星子喷了皇帝一脸,皇帝都不敢擦。搁清朝够诛九族了


的入侵方式完全不同。西方是为了

西方的文明是更高的
中国北方少数民族的文明水平很低
我认为一八五〇与六〇年代中国与西方列强之间的漫长交战议和过程中,明显可见汉人一般都主张为维护原则而战,满人却宁愿姑息侵略者以换来清室残喘的机会。 
在《民族篇》里,我已经论述了:主体民族统治,民族平等。
没有一个国家可以在非主流民族
少数民族统治主体民族就是清朝的“原罪”。这个原罪的影响远不止民族冲突。
真的是这样吗?明朝就这么差吗?


1851年爆发了
虽然历史书上计算清朝是从1644算起,但准确的说,该从1662南明灭亡算起。即使从1644算到1851,也不过207年。这一年爆发了太平天国暴动,
是太平天国用宗教做洗脑工具,万众一心、战斗力强大吗?非也,是满清军队的战斗力实在“渣”到了一定程度。
历史课本上提到了太平军战胜美国人华尔的“洋枪队”,似乎太平军战斗力不错的样子。但要知道,洋枪队不过是募集了一些白人流浪汉当军官,士兵主要是菲律宾人。人数也不多,高峰期也不过5000人。首领Frederick T. Ward,也就是华尔,并非一个受过长期专业训练的军官。所以,洋枪队是杂牌军中的杂牌军,战而胜之值得吹嘘吗?
值得!因为满清对列强的战绩实在太惨,只好连反对满清的农民军的战绩也拉来凑个数,给高考之后再也不接触严肃历史的学生们增强一点民族自信心。TG知道,像我这样喜欢寻根究底的同学只是一小撮,因此从来不怕被戳破。

如果不能给福利,就给自由!
给自由发展的权利。
No taxation without representation!
无代表,不纳税
——北美十三块殖民地的口号
换句话就是:要想让我们履行义务,就得给我们相称的权利!

【在此需要一个论据】

虽然汉人的比例是满人的几十倍【好像有上百倍,但是没找到具体数据】,但科举考试中汉人与满人的录取人数是相同的【缺少引用】。

那为什么
既得利益集团

“无代表,不纳税!”这是北美十三块殖民地与英国斗争的口号之一。

四、复中华
清末仁人志士在救亡图存的过程中,心里始终有一道坎:救的是中国还是大清?
如果一个行为伤害了中国,但是却间接打击了满清统治集团,间接的有益于建立汉人自己的国家,那值得做吗?
清末志士的回答是:值得!

来看看“国父”孙中山的行为吧,虽然他近年来的名声并不太好,但仍然可以作为一个典型例子。
1894年底,他在檀香山建立兴中会,提出了“驱逐鞑虏,恢复中国,创立合众政府”的主张。1904年,黄兴领导的华兴会提出 “驱逐鞑虏,复兴中华”。
1905年,兴中会、华兴会、光复会等多个团体在东京成立同盟会。同盟会的十六字诀是“驱除鞑虏、恢复中华、创立民国、平均地权”。
可以看到,“鞑虏”和“中华”是两个概念。满清统治集团入关250年以后,还是被视为“鞑虏”。

袁腾飞说:“满清”
( 1 ) 民族平等
北魏孝文帝迁都洛阳。
坦白讲,我可以接受孝文帝这样的异族统治者,也可以接受隋唐这种参杂异族血统统治者,但我决不会接受元清的统治。
原因很简单,在北魏和隋唐,汉人是没有受到歧视的。而在元清,歧视无处不在。
金钱、财富、地位、荣誉。
现在的英国王室是温莎王朝,历史上其实是德国汉诺威王朝,一战时要和德国死磕,所以改名叫温莎王朝。公爵,为什么英国人可以接受德国人做王?最起码的,这位德国王没有推行英国人低一级吧?在、

民族和平共处的大前提是民族平等 。在此前提下,才可以讨论其它问题,才可以民族融合。否则,将会永无宁日。

关于民族问题更详细的论述,见《民族篇》。
比如上海宝钢。炼钢最需要的是煤和铁矿石,上海是有煤还是有铁矿石?它最大优势就是交通便利。
元朝分天下为四等人,

现在讨论这个问题都要讲究“政治正确”,而民族团结无疑是个“大政治”。连岳飞都不是民族英雄了 ,还怎么公正评价清末志士?
所以,现在的学界主流是“不能啊”,但

如果你要参加高考的话,“阶级局限性”。


也是1905年,清政府派遣五大臣出洋考察。从进步的角度说,这是一件好事。孙岳 说“若清廷得以实行立宪,则彼君主政权益为巩固,汉旗将无复兴之望”

当时的革命党人,甚至密谋刺杀清政府的革命派。

这是什么样的心理呢?
家里有一个精神病人,成天搞破坏。

可以看到,满清统治集团从未取得汉人的认同。
这是汉人的错吗?
河南九千万人口只有一所211大学,
难道我们要说:跪谢教育部!竟然还赏赐了河南一个211名额,每年还给上亿拨款。看呀,郑州大学建国以来的教育部拨款总和竟然比得上清华一年的教育部拨款了,河南人叩谢天恩!



在《经济篇》里,我论述过:各地区收入基本平等是后工业化才能出现的。
在农业社会

在《忠孝篇》里,我论证了:没有形成认同感的国家打不赢倾国之战。


在革命者眼中,满清统治集团根本就不算“中华”的一部分。所以,当一场战争可以损害满清的统治,即使会对中国造成严重伤害,
换成任何一个汉人的朝廷,
孙中山的主张是


就像是街头打架的小混混,
社会矛盾的积累已经很深了。
你觉得
推行议会

我并不喜欢明朝,只是更不喜欢清朝。
为什么有些国家可以成为强国,而有些国家不行?
内部认同
五、护侨民
保护本国侨民是所有的民族国家的义务。

我们可以想想看,什么样的国家才算强国?
面积够大
人口够多
人民够富
军队够强
在有些人眼中:所谓尊重,就是你尊重我;所谓民主,就是你同意我;所谓感受,就是我个人的感受。所谓沟通,就是要达成一致;所谓达成一致,就是你来跟我保持一致。

从生物学角度,他们成功的把自己的基因散布到了全世界,而中国人的基因只能蜗居在东亚一隅。
从语言的角度,英语是世界通用的语言,更多关于英语的压倒性优势见《文化篇》

\section{签条约}

如果一个国家的政治制度不合理,各种幺蛾子都会出现。日本在二战时的

所有有不少中国人鄙视日本曰:“岛国小民的思维根本无法产生一个战略家。”

写到这里,我不禁想要吐槽一下:说的好像中国大陆有战略家似的。

六、千年劫
所有的中国封建王朝都讲究君权至上,但从来没有一个朝代控制到满清的程度;
所有的中国封建王朝都喜欢集权,但从来没有一个朝代控制到满清的程度;
所有的中国封建王朝都喜欢愚民,
所有的中国封建王朝都喜欢禁锢思想,但从来
稍有不慎就会诛家灭族

在工业革命之前,集权、愚民、禁锢思想都未必是坏事。偏偏在这一切皇权手段走上巅峰的时候,这是华夏族的劫数啊


终清一朝,所有皇帝都有极其强烈的控制欲

\begin{description}
\item[春秋战国] 埋下了虚无主义和形式主义的种子。即使在百家争鸣时期,中国也没有出现类似于古希腊的思辨和抽象科学理论。
\item[秦朝] 埋下了君权至上的种子。
\item[汉朝] 埋下了独尊儒术的种子。
•	晋朝让虚无主义生根发芽
•	隋朝埋下了《大隋律》的种子
•	唐朝埋下了科举制的种子
•	宋朝埋下了崇文抑武的种子
•	元朝把苗圃里正确的种子扫荡一空
•	明朝埋下了八股文的种子
\end{description}


七、观后清
政府对人民没有任何畏惧
政府可以放手“启民智”
在资本主义国家,如果社会出现严重的危机,可以合法的让政府倒台,重新选举,由此完成一次社会矛盾的释放。
但是在中国,有一个组织是永远伟大光荣正确的,所以它不会犯错——至少它不会承认自己犯了致命错误,它也不会下台,因为它代表了全国人民的利益、和正确的发展方向。于是,在社会压力积累的时候,它就只能用“堵”,而不能用某种释放压力的方式。
苏联解体带给这个组织这样一个“教训”,绝对不可以放松言论控制。
于是,在“启民智”方面,TG肯定不敢放手去做。

“兴民权”呢?TG可曾允许有其他人来限制其权力?

八、总结

明清是非常复杂的话题。
《忠孝篇》 为什么中国“有孝无忠”
《过滤器》 为什么科举制是垃圾
《战争篇》 为什么中国古代总打败仗
《封建篇》 为什么封建主义被资本主义碾压
《强国篇》 怎样才能成为强国


