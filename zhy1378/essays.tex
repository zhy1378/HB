\chapter{短篇集}

\section{存在即合理}
黑格尔这句名言估计很少有人能真正弄明白,用错这句话的比例接近于使用“空穴来风”的比例。

德语原文是:
\begin{quoting}
Was vernuenftig ist, das ist wirklich;\\
und was wirklich ist, das ist vernuenftig.
\end{quoting}

英文翻译是:
\begin{quoting}
What is rational is actual and what is actual is rational. 
\end{quoting}

因为
凡合乎理性的即是真实的,凡真实的即是合乎理性的。

我是这样来理解的:如果一个事物存在,那一定是有原因的。如果一个事物有存在的理由,那就真的会存在。

翻译成“合理”是不恰当的。因为中文的“合理”,有理所应当、正确、良好的褒义暗示。

比如说,腐败的普遍存在,必然有它普遍存在的原因,而不是说这样的存在是“合理”的。

\section{可知论}
“可知论”指的是这样的一种理论,这个世界最终是可以被认知的;“不可知论”当然是相反的,也就是这个世界最终是不可以被认知的。

对此,我的观点可以被浓缩为七个字:\emph{求知 不问 或可知}。解释如下:
\begin{description}
\item[求知] 科学和技术的巨大价值已经被人类所了解,所以无论如何人类都必须继续求知。
\item[不问] 两种理论的争辩在短期内(也许几百年内)都不会有结果,故此这种争辩价值不大。“不问”指的是不用去急于争辩这个问题。
\item[或可知] 我个人倾向于认为这个世界最终是可知的。
\end{description}


\section{左和右}
•	本文参考了很多文献,但与任何类似文章的思路都不相同。我认为,我提出的区分方式更为简洁。
	虽然我们经常听到“左派”、“右派”、“右倾机会主义”、“左倾冒险主义”的说法,但很少有人把它搞得特别明白。

\subsection{历史起源}
左派和右派的称呼最初起源于18世纪末的法国大革命。在大革命期间的各种立法议会里,尤其是1791年的法国制宪议会上,温和派的保王党人都坐在议场的右边,而激进的革命党人都坐在左边,从此便产生了“左派”、“右派”两种称呼。

追求平等是左,反之为右。


图钉
在政治观点上,会有某些人的观点成为“图钉”。于是,与之相对的观点
( 1 ) 大政府和小政府
大政府指的是扩大政府权力,
中国政府的权力大极了。从这一点上来说,中国政府走的是左派路线。
但是中国政府的二次分配是加剧了不平等而非缩小,从这一点来说
( 2 ) 毛的倾向
	毛发动了诸如大跃进、农村公社化、割资本主义尾巴等极端的促进平等的运动。因此他的倾向是极左。
( 3 ) 纳粹
纳粹的本意是“国家/民族社会主义” ,也就是社会主义的一种形式。按照其观点,个人必须服从国家意志。但是纳粹党和新纳粹被认为是极右翼。这是为什么?
纳粹追求的不是全人类的平等,而是一个民族统治其它民族。
	

自由派
人类是否发展到了极限呢?
比如,我们看看两次大战。大量的人死亡,对人类造成了
二、我的政治倾向
我不认为市场可以解决所有问题。

\subsection{极左和极右}
极左或者极右的人是这样的,如果别人不肯接受自己的政治诉求,就要不惜一切代价逼迫别人接受。

是不是所有的极左派政治诉求都

我觉得,只要人有强烈的“\textit{政治刚性需求}”,即使他所追求的是那就容易走极端。即使比如说,有人坚决要求民主,



再比如,尊重人的生命权是好的,但反对在任何情况下执行死刑就走了极端。更极端的是,在电影《大卫 戈尔的一生 The Life of David Gale》中,一男一女为了反对死刑用了极端手段:他们的目的是为了证明死刑中可能存在冤案,于是女的自杀,然后布置证据把男的变为嫌疑人。男的不为自己辩护,坦然接受死刑,但是他留下了一卷公布真相的录像带。

虽然我支持死刑,但我也承认死刑反对者是有道理的。

\emph{坚守政治追求,保持政治弹性。}


原因很简单,人是有惰性的,而战争是高效催化剂。因为有战争,
	本来都是松松垮垮的四十小时工作制,一旦下达总动员令,就开始全国连轴转。
	1906年,徐锡麟刺杀安徽巡抚恩铭。徐锡麟说:“他待我是很仁厚,可这是私惠;我杀他,这是天下的公愤。”
	


就像《三体》描绘的场景那样,因为有同归于尽的勇气,所以人类才能

\section{充分和必要}
充分必要条件是一个数学概念,也是一个逻辑学概念。

解释这个概念显得有点小儿科。但是根据我的经验,高考十年后还能清楚记得这个概念的人不到1/3. 

如果条件A可以推出条件B,则A是B的充分条件,B是A的必要条件;如果条件B可以推出条件A,则A是B的必要条件,B是A的充分条件。

比如,前文我提到民主自由是成为智力型发达国家的必要非充分条件。意思是,如果一个国家是智力型发达国家,则它一定是民主自由的;如果一个国家是民主自由的,它未必是智力型发达国家。用数学公式表达就是:
%\[ \mbox{民主自由} \nRightarrow \mbox{智力型发达国家}\]
\[ \mbox{智力型发达国家} \Rightarrow \mbox{民主自由}\]

同样的,地广人稀是成为资源型发达国家的必要非充分条件。

然而,民主和自由具有非常高的相关性。我们可以说,在和平时期,真正民主的国家必然是自由的,真正自由的国家必然是民主的。

如果某生物是昆虫,则它有六条腿。昆虫是有六条腿的充分条件,有六条腿

中国是2014年人口最多的国家


\section{理论数据 采样数据 经验数据}
\textbf{理论数据}

\textbf{采样数据}指的是统计数据,

比如统计麦子千粒重,一般情况下是在麦田里选几个不同的位置,

\textbf{经验数据}指的是根据以往的经验猜测而得的数据。它的可靠性会更低。

本文中涉及的数据我都会给出来源。