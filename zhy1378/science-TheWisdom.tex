\chapter{智慧论}
\zSubtitle{窦梓钟}
\begin{quoting}
此文原来是给一本小说写的,假托它写于1663 年的明朝(实际上明朝亡于1644 年),化名作者窦梓钟。发布后反响不错,所以我将其独立出来,略加改动,收录于此。
\end{quoting}

赵州桥是一座建于耶历610年的桥梁,迄今已经有1053年的历史。它仅仅严重受损过一次,是因为运输薪炭的船只造成的火灾事故。在1563年维修之后,这座桥又安然无恙的使用了一百年。

开封铁塔建成于耶历1049年。该塔结构坚固,历经多次地震和黄河水患都没有损伤。

如果明国的建筑都有这样优秀的质量,虽然建造的速度慢一些,但是只要保护得当,这样优秀的建筑就会越来越多。或许到了某个年代,明国后人已经再也不需要大兴土木了,只用给现有的建筑修修补补即可。他们的生活也就过的越来越轻松。

这些建筑的技术并非在尧舜和孔孟的时代就存在,而是在几千年的时间里一点点的发明出来的。可惜,每次战乱都会造成技术倒退。后人就必须重新研究,再次探索前人走过的道路。大唐和大宋是两个辉煌的时代,在战乱之后不知有多少技术湮灭在历史的尘埃中。

因此,国家应该制定这样的制度:
\begin{itemize}
\item 保证在新技术被发明之后可以迅速扩散到其它地区。
\item 保证技术不会遗失。
\item 当一种新技术或者新理论被发明之后,要谨慎的、反复的验证它,以确认它是正确无误的。
\end{itemize}

按照“科技树”理论,技术之间是互相促进的。可惜,这棵科技树伫立在黑暗之中,必须摸索前行。这棵树上还有无数的岔口枝桠,稍不注意就行差踏错。轻者浪费几年光阴,重者就会损失大量财富甚至生命。每个人攀爬的时候,都沿着前人刻下的痕迹,然后再刻下自己的痕迹给后人。

“聪明”和“智慧”是两个不同的概念。“聪明”是一个人攀登科技树的能力,而“智慧”是一个人攀登到的高度。“聪明”是学习的速度,“智慧”是最终获得的有用的知识的总和。

“生也有涯,而知也无涯”,想要学会所有的知识是绝不可能的。一个聪明人虽然博闻强识、勤奋好学,但是却从不对所学的知识加以思索考证,于是真假难辨、实妄不分的知识就充塞了他的头脑,他很难用这些知识给后人带来一些益处,最终他只不过是一个“\textbf{聪明的笨蛋}”罢了。另一个人或许天资平凡,但是他诚恳认真、坚持验证,不管是在科技树上摸到了一根新的树枝,还是证明了某些枝杈绝不可行,都给后人带来了帮助,于是他就是一个“\textbf{愚笨的贤者}”。

对于一个民族来说,很多人一起摸索肯定是要比一个人单打独斗更好。这时候,“相互验证”极为重要。如果一个技术或者理论被多人验证过,那就可以被认为是“真理”,后人就可以放心的踏在这根树枝上,摸索其它枝桠了。

最有智慧的民族,就是在科技树上攀爬的最高的民族。人数众多和聪明人多都不一定代表更智慧。

如此看来,明国人其实是一个“\emph{聪明无智慧}”的民族,因为几千年来都没有形成“相互验证”的习惯。“黄帝内经”和“本草纲目”里面都充满了大量未经验证的知识。明国学会里都是聪明人,个个都有一些“语不惊人死不休”的奇谈怪论。这不过是知晓了科技树上的很多枝杈罢了。哪些枝杈是对的?哪些是错的?对的为什么对?错的为什么错?没有人知道。因为大家都是在空想。

如果明国还是不能制定一套相互验证的机制,我恐怕后人很快就迷失在海量的知识中了。