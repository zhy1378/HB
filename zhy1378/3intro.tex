\makenomenclature

\nomenclature[1]{\textbf{宗教篇}}{有过很多称呼,比如香帅、提督大人、香菇大帝等。\\ 如果不知道这个名字,主编只能代表编审委员会对您的“大明联邦史”考试成绩默哀三分钟。请即速购买阅读一品文华殿大学士\textbf{灰衣熊猫}主编的《伐清——大明复国战争史》。}

\nomenclature[2]{\textbf{艺术篇}}{历史上艺术的发展趋势。}

\nomenclature[3]{\textbf{科学篇}}{科学是什么?}

\nomenclature[4]{\textbf{共产篇}}{共产主义是虚妄的吗?它有多大的可能性成为现实?为什么共产主义的尝试总是失败?}

\nomenclature[5]{\textbf{开封篇}}{开封,我的故乡,曾经的七朝古都,}

开封的地理条件非常适合修铁路,地形起伏小,土壤层厚,河流较少,地下水位低,

交通给人带来的感觉很大

然而,为什么开封的发展半死不活?

中子豆读书列表





\printnomenclature[1em]

%latex filename.tex
%makeindex 0Main.nlo -s nomencl.ist -o 0Main.nls
%latex filename.tex