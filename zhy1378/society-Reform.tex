\chapter{改革篇}
\begin{quoting}
我常自嘲为\textit{键盘政治家},
\end{quoting}

\section{推动有中国特色的改革}

\begin{description}
\item[打不赢] 美帝枪支泛滥,如果政府真的要搞独裁的话,民众可以自发武装起来,马上就达到二战步兵的水平。而在中国,大多数民众的武器只有菜刀木棍,还维持在陈胜吴广时期,顶天了弄出来些煤气罐、土手雷。真打起来的话,政府一把冲锋枪就能镇住局面。因此,暴力反抗几乎没有胜利的可能性。

或许有人寄希望于高层军官起义,直接瓦解军队。我只能说,这是概率极低的事情。请了解下近代历史,近现代国家形成之后,兵变首领掌握国家政权的例子很少。尤其是在现代,通讯技术发达,大规模的军队调动几乎不可能逃过中央政府的眼线。
\item[不能打] 如果真的开战,中国几十年来的建设成果可能被一扫而空。在混乱中,容易得手的是“流氓无产者”。而且,如果中国陷入混乱,新疆、西藏、南海等边缘领土可能就要失手。

\item[]
\end{description}

\subsection{替代品}
假设,只是假设,TG真的倒台,谁能执政呢,谁有执政能力呢?很遗憾的是,我还没有看到。

比如《零八宪章》,由很多右派人士签字。如果中国走

有中国特色的改革

\begin{description}
\item[\emph{绝不挑战一党专政}] 

\item[\emph{随时、随地、随处准备妥协}] 八九学运是一次惨痛的失败,甚至可以说是国家、人民、TG“三输”的结局。

假如提出了一个过高的
\item[不威胁高层的利益]
\end{description}

如果出现了群众运动、学生运动的局面,想要这些人保持克制是很难的。因为在群众运动中,永远是激进者掌握话语权。




\subsection{初级可推动项目}

\begin{description}
\item[财务公开] 有望在习李朝实现。
\item[乡镇直选] 村官直选已经在很多地区实现了,下一步就可以试着去推动乡镇级官员直选。成功之后,再推动县级直选。

如果乡镇级直选没有实现,那么在提案里压根就不提县级以上直选的事,以免遇到太大的阻力。
\item[公务员财产公示] 这一条的阻力会非常大,因为官老爷怕自家的钱多得让屁民吓着。
\end{description}

\section{政治改革}
政治改革牵涉到


传统的产业分类是分为:第一产业农业,第二产业工业,其它为第三产业。一般来说,发达国家的第三产业产值占的比重都在一半以上。

现在,我感觉这种分类法是不太合理的。

一个收入靠旅游业的小国,其第三产业的比值可以是很高的,但是抵御经济危机的能力并不强。因为一旦危机发生,人们最先舍弃的就是旅游。菲律宾这个岛国,农业、工业都不发达,仰仗旅游业和侨汇——也就是本国人到处做“菲佣”往家里寄钱。

虽然把第三产业称为服务业,但怎样去服务确实天差地别,在餐馆端盘子和在高能物理实验室里做研究那是大不一样。故此,我从第三产业中划分出第四产业,把第四产业分出去,才能把真正的强国、富国显出来。

在我的理论中,第四产业中包含的主要是:科研、教育、金融、卫生、管理、法律和文化。第三产业中还剩下的是:餐饮、旅游、仓储物流、交通运输、销售。

第四产业高的国家,可以被称为“智力型发达国家”,英法德美。与之成对比的是“资源型发达国家”,比如沙特、阿联酋。还有些国家兼有二者特征,实在是得天独厚,比如澳大利亚、加拿大、新西兰。

资源型发达国家是不值得羡慕的,因为资源总有用完的一天。中国未来的道路,必须是走向智力型发达国家,因为中国的自然资源极其匮乏。这里,我指的是人均占有的自然资源,比总量实在是太没意思了。中国的教育中,有意无意的在比总量,以此唤起学生们一点点可怜的自尊。我到了国外,越发感觉不能比总量、总值,就是要比人均。

中国的政治体制改革(初稿)

虽然“民主”在中国仍然是个提不得的词,但是中国的民主化进程是不可阻碍的。首先来说,随着教育的普及和网络的发展,青少年对于民主的认同程度大大提高。根据我个人的体会,80前的人对于“泛政治化”的时代还留有一些可怕的记忆,而80后、90后对于民主认同的提升是可以感受到的。其次,网络使得民主的成本大大降低了,独裁的成本大大提升了。

然而,中国的政改究竟走向何方?什么是最好的政治制度?

1. 民主的优势

民主的优势在于群众监督。几千年来,中国把自上而下的、少数监督多数的监督方式发展到极致,可惜这种监督方式对于一个庞大国家的无奈在满清后期甚至现代已经展示得清清楚楚。有效的监督方式,必须是自下而上的、多数监督少数的。不管习近平怎样努力,他都无法打破这个官场圈子。

民主的优势还在于少犯错。历史已经证明,人类的发展中,“少犯错”比“决策快”更重要。

民主还有一个潜在优势,就是更具有合法性,更能赢得国民的效忠。效忠对象越能体现国民利益,这种效忠越真诚、越坚定。中国的历史课本上,总是把满清在19世纪对列强的一败涂地描述为技术原因。说起来令炎黄子孙汗颜的是:中国对列强的战绩甚至不如美洲印第安人、新西兰的毛利人、南非的祖鲁人。究其原因,是满清的统治结构出了大问题。(具体待分析)

2. 民主的劣势

民主最大的劣势,当然就是效率。美国的高铁计划了很久,还是没有修好。(具体描述待补充)

待补充:南非、印度民主化进程中出现的问题

3. 民主的方式

民主的具体方式有很多,比较典型的是精英民主制和普选民主制。

精英民主制的特点是:占人口比例很小的一群人拥有极大的决策权。这种制度在古典时代是很有价值的,因为那时的交通、通讯条件不发达,民主成本很高,精英民主制可以在一定程度上保证决策效率。它的缺陷也很大,精英阶层很容易通过一些有利于本阶级的政策,降低社会的垂直流动性;精英阶层也容易忽视下层人民的需求,造成贫富分化。

古罗马实行的就是古典精英民主制,(描述待补充)其它例子:波兰的贵族统治

普选民主制恰恰相反,常常让所有成年人(或仅成年男子)以公平的每人一票的形式来决策事务。在小国寡民的情况下,这种民主制还是不错的。而对于一个大国来说,不仅仅是民主成本的问题,更由于人民利益和价值观的不同,造成很多“用脚投票”的现象。

社会一般来说呈现金字塔型,也就是富人少、穷人多。当每人一票的选举时,社会下层人民的主张就会占有优势,可这些主张未必是有利于国家长远利益的。比如很容易推出一些高税收、高福利、多假期、过分保护工人的政策,这些措施降低了社会垂直分层,导致人们劳动欲望降低。“欧猪五国”(西班牙、希腊、意大利、葡萄牙、爱尔兰)的经济困境就是最好的例子,北欧国家也有滑入深渊的可能。

政治是复杂的,弄清各政治实体间的关系并做出理性的选择并不容易。有些人的天生智力水平,也许就阻碍了他们对于政治的深层思考;还有很多人,对于政治不敏感或者无兴趣,只是在选举时随随便便投一票而已。(很遗憾这种描述有社会歧视之嫌)

还有一个考量,是尽可能小的改变。我其实是社会改良派。

选举中,不能有直感歧视。直感歧视是一望可知的歧视,比如美国上世纪的黑人投票相等于白人的五分之三,中国的农村居民相等于城镇居民的四分之一。(详细数字待考证)

经过长篇渲染之后,我所设想的中国政治体制是这样的:

1. 实行真正的全国人民代表大会制度。“真正”意味着:既然全国人民代表大会是最高权力机构,那就不容许有任何组织或党派凌驾其上;人大代表必须来自选民,不能由任何组织和个人来指定。这当然也就是议会了。
2. 国家最高领导人由乡镇级和市区级人大代表选择。乡镇级人大是人大的最底层,其代表率(每个代表所代表的有效选民数)大约介于100~1000之间。
举例说明:甲镇有选民15000人,共有人大代表20人。其中代表A的支持率是20%(得票数/所有代表的总得票数。其中总得票数可以大于选民人数,因为选举代表时一般实行一人多票制),代表B的支持率是18%。那么在选举国家最高领导人时,代表A的一票就代表了15000 * 20% = 3000,代表B的一票就代表了15000 * 18% = 2700。可以出现小数,精确到小数点后两位数字。

由于代表率低,这些底层代表生活在“熟人社群”中,一举一动处于被群众监督之下,既没有高高在上的感觉,也不易玩高层黑幕。

投票采取实名制,便于监督。

这种选举方案的优势是:1. 基数大,乡镇级人大代表可达50万以上,不会形成固化的精英阶层。 2. 代表率低,可以让底层人民的主张充分表达。 3. 对选举人进行了有效过滤,剔除了最懒惰、教育程度最低的人群。 4. 没有直感歧视。 5. 全国人大相等于议会,基于权力分离的原则,不能由全国人大来选举国家最高领导人。

\subsection{国际友人}


有没有可能呢?很有可能。CIA在南美的活动臭名昭著

接受帮助后,如何面对


可以有选择的接受独立个人的帮助,但是不能接受国外组织的帮助。



《可知论与不可知论》

“可知论”指的是这样的一种理论,这个世界最终是可以被认知的;“不可知论”当然是相反的,也就是这个世界最终是不可以被认知的。

对此,我的观点可以被浓缩为七个字:求知 不问 或可知。解释如下:

求知:科学和技术的巨大价值已经被人类所了解,所以无论如何人类都必须继续求知。

不问:两种理论的争辩在短期内(也许几百年内)都不会有结果,故此这种争辩价值不大。“不问”指的是不用去急于争辩这个问题。

或可知:我个人倾向于认为这个世界最终是可知的。

这个世界上有很多人,即使从小受到良好的教育,他们的智力仍然不足以承担科研的重任

小国可以这么干,大国不行。比如希腊,它可以完全仰仗旅游业。

第三产业:面向消费的非工农产业。
工作简单重复,不需要太多培训的产业。
甚至有很多工作只需要15分钟的培训。

第四产业:面向生产的非工农产业。


一二三产业创造财富,第四产业分配财富。

一个地区可以这么想,整个中国

一二产业创造财富,三四产业分配财富。

可以看到,第一产业吸纳的劳动力会越来越少。

顺理成章的创造职位,切勿
尤其是切勿增加流通环节的岗位。
高速公路收费员

比如印度,如果不能振兴第二产业,

高速公路怎样管理?

服务业有首先有发达的一二产业,才会产生需求。比如餐饮业,吸收的劳动力极大。但是餐饮业只有在人民忙碌、有闲钱的情况下才会发达。

发达国家的农业,已经开始转变为工业化农业。农民像工人一样,不再拥有土地,而是签订合同、按时上班。

古代对于技术是非常保守的,老师傅都得“留一手”。

成功需要一万小时。换算一下,按每天八小时,一共1250天。每年250天,那就是5年。如果一个人很勤奋,3年就足够了。

在教育方法得当的情况下,一万小时足以把一个普通人培养成某个领域的专家。

包括外语、编程、法官、医生

主要原因是不能冒险。

所以,不存在“培养一个干部不容易”,而是“是个人就会当干部”
对于犯法者狠狠的惩戒吧。


上访是中国诸多制度缺陷形成的怪胎。
假如中央、地方之间权责明确,何须设立信访办?
假如干部任免不看重维稳成果,何须下大力气“截访”?
假如上下欺骗不是形成了潜规则,何须对截访睁一眼闭一眼?
假如贪腐不成风气,何须诸多“驻京办”?
假如财务公开,驻京办的账目又如何掩盖?
假如法院判决具有权威,“刁民”上访有何用?
假如司法独立,屁民即可通过正当程序争取权益,何须哭爹喊娘等青天大老爷?

