\chapter{科学篇}
什么是科学?

\section{科学精神}

\section{科学教}
有人说我是“科学教”的信徒,不管什么事都要扯到科学上。

按照我的\textit{宗教判定标准}\footnote{见《宗教篇》},来判定一下:
\begin{description}
\item[有组织] 科学有相当严格且庞大的组织。
\item[有圣人] 科学没有圣人。伟大的科学家虽然被“称圣”,但他们并非不可置疑的。
\item[有圣经] 科学没有圣经。
\end{description}

所以,科学不是宗教。

\section{学术}
\subsection{引用}

\subsection{抄袭}
中国学术腐败

难道中国学术腐败真的很难对付吗?在我看来,

\textit{不敢放手发动群众}。

为什么不敢放手发动群众?因为官员们的学术论文也经不起检查。

怎样放手发动群众?
规定一个期限,比如说一年,

过了此期限,开始

刚开始,肯定有人抱着侥幸心理。

降低成本

检查抄袭的难度是否很大呢?其实并不大。目前有软件专门检查抄袭,如果有一模一样的

经过这个“立威”的过程,

我大学之前经常到老妈工作的医院里拿废弃的心电图波形纸做演草纸。那是厚厚的一大本,每页都印着淡红色的格子,曲曲弯弯的画满了一个人心脏跳动的图形。上课无聊的时候,我就喜欢观察心电图波形。虽然什么都没看出来(看出来了才奇怪),但我产生了一个疑问:心电图是否可以验孕?

中医有诊脉验孕,看过西游记还知道有“悬丝诊脉”,当然那是神仙的招法。脉象反映的不就是心脏跳动的波形吗?既然可以诊脉验孕,那也就应该可以用心电图验孕。于是我就以此询问做心电图的医生,答案当然是否定的。如果心电图可以验孕的话,早就是盛行的医疗手段了。

从那时起,我对中医诊脉验孕的怀疑就种下了。心电图的灵敏度、精确度,不该是大大高于手指诊脉吗?如果说诊脉可以验孕,那为什么心电图不能?

一种说法是:胎儿在十二周开始有心跳,于是胎儿的脉象就会叠加到孕妇的脉象中。

对诊脉验孕的第二次质疑是在我读到关于胎盘的知识之后:孕妇的血液是不会直接流给胎儿的,而是在胎盘里进行营养物质交换。打个比方:孕妇的血液就像是带着营养物质的运输船队,但这个船队不会直接“开”到胎儿体内,而是在胎盘这个“码头”卸货,然后由胎儿体内的船队运走。胎盘的这种工作机制无疑会大大降低心跳脉冲的传播。

只要在心电图波形上看不出是否怀孕,我就不相信把脉可以验孕。


北京积水潭烧伤科医师阿宝悬赏10万元,挑战

按中医的说法\footnote{《大学副教授迎战把脉验孕:不想让他们黑中医》 \url{http://news.sina.com.cn/s/2014-10-15/024530989470.shtml}},“怀孕的脉象叫滑脉,比如一条绳上串着几个珠子,你摸绳子和珠子的感觉不一样,珠子摸上去有一种滚珠感,而绳子要细一些,怀孕的感觉就像摸珠子。”

这句话如果是真的,那反应在波形上就是

而且,如果是真的,我还可以提出一个解释:孕妇的心跳是大波,胎儿的心跳是小波,大波和小波叠加之后,波形就完全不同了。

\footnote{《“别了,阿宝志安” 北中医教师退出脉诊验孕挑战》 http://www.bjnews.com.cn/news/2014/11/06/340458.html}

如果

只要一天中医不愿意将

什么是科学?

科学是一切可证伪的信息的集合。

CCTV的百家讲坛里有不少中医的内容,我也听过一些。

对于历史记载一定要谨慎,因为

中国古人记录历史的节操实在不怎么样。


在