\chapter{效忠篇}

在西方,英法德美都没有大规模的叛变,

效忠度与什么有关?效忠对象的合法性。

但如果是两个实力相近的近现代国家,一场战争打成了“倾国之战”,国家根本是无法在经济上负担
甚至,到了战争的后期,连士兵的薪水和补给都无法保证。

现在中国人的效忠对象是谁呢?

我们效忠的是中国吗?

理论上应该是?

因何而效忠,也会因何而背叛。

为了钱效忠,就会为了钱背叛;

忠孝篇
有个名士(忘了是谁,以后再找。)见皇帝,皇帝问他:“你也是大儒,你用最简单的话总结一下儒家思想吧。”名士说:“唯忠孝二字。”
不得不说,总结的实在太好了。忠孝思想统治中国近两千年,是至高无上的道德品质。
历朝历代的皇帝都认为孝子也是忠臣。西汉有“举孝廉”的制度,就是选拔各地的孝子当官。如果官员的父母去世,就要“丁忧”三年(后来打个七五折,即二十七个月),也就是三年内不做官,不嫁娶,不赴宴,不应考。这实在是个很奇葩的制度,如果一个官员在年富力强之时父母先后去世,那他就得赋闲四年半。

然而中国几千年来都“有孝无忠”。孝道在民众之中得到了普遍认同,忤逆子会受到民众自发的唾弃。如果汉人拿出来抵制忤逆子的精神来反抗外族侵略的话,断不会有宋明之殇。
有人认为,中国人不到危难关头不会团结。抗日战争是危难关头吧,中国人团结了吗?汉奸和伪军,就像韭菜一样,剪了一茬又一茬,越剪越旺盛。如果日本不自己作死,引得美帝参战,我看不到中国胜利的可能性。要知道,美帝消灭了日本的精锐陆军及主要技术兵器(海空军)。

那么问题来了,为什么中国有孝无忠?怎样培养忠诚的国民?如果中国再次面对抗日战争那样的危机,中国人会团结一致吗?
一、相似度
我先从生物学天性的角度来寻找人忠孝的根源,即,人类有哪些“与生俱来”的爱。
( 1 ) 细菌的故事
有一种细菌在营养液中繁殖。它的密度大于水,但它又需要接触到空气才能存活,所以就需要分泌脂类的菌膜来保证自己浮在水面上。
细菌在分裂的过程中也一直在变异,于是,有些细菌分泌的菌膜多一些,我们且称之为“无私者”;有些分泌的少一些,就叫“自私者”。要知道,能量对于生物体是非常重要的,而分泌菌膜无疑会消耗很多能量。于是,无私者的分裂速度会比较慢,而自私者分裂快,自私者会越来越多。
如果一个菌落的自私者过多,久而久之,菌膜就会太少,无法保持浮力,就会沉入水中,集体死亡。游离的无私者,会在营养液的另一个位置重建菌落。【在果壳看的,找不到了】

对于如此低等的生物,恐怕只能在自私、无私之间循环。而人类的行为要复杂的多。
( 2 ) 黄蜂的故事
黄蜂的蜂后
黄蜂群中的所有工蜂都是“姐妹”,任意两姐妹之间的相似度为75%,
因为相似度较低
黄蜂会把“姐妹们”产的卵杀掉,以此保证孵化的卵都是蜂后的。
( 3 ) 高等生物
对于高等生物来说,“相似度”仍然是形成认同感的重要工具。
我们很难建立一个统一的标准,来计算各种相似度带来的认同感。
比如语言,有共同语言才可以进一步的交流思想。观察欧洲民族国家形成的过程,可以发现语言起到的作用比血缘还大。
宗教
血缘 然而血缘关系常常只能在很小的范围内起作用。

二、忠孝之源
从生物学角度,忠孝都是利他行为,并没有本质区别。我归纳出的忠孝原因如下:
1)	原属感
2)	恩情感
3)	利益链
4)	教育
5)	氛围
( 1 ) 孝道天性
在这五条的基础上,我们来逐一讨论
对父母与生俱来的归属感
父母对子女的恩惠
维护父母利益有利于维护家族利益,维护家族利益有利于维护个人利益
从小到大都在教育要孝顺,而且孝顺的目标、原因和手段都非常清晰。
整个社会已经形成孝顺的氛围

人有从众倾向。

出身于哪里,就爱哪里。这是一种不需要任何理由的爱。
如果有个领养儿童,到了成年才认识自己的亲生父母,他多半会立刻对亲生父母产生不一样的感觉。这就是原属感。
这种感情与生物相似度有关。如果
报恩是人类天性。
通过说教
目标:知道为谁服务
原因:知道为什么服务
手段:知道怎样服务
三、细论投降

投降归投降,投降的方式可大大不同。我简单归纳一下投降的种类 :
将领(指军队最高实际领导人,下同)被收买,叛变后加入敌国
将领被收买,放下武器投降或退出战斗
士兵畏惧战斗,大规模逃散,包括“炸营”
临阵怯懦,开战不久甚至未开战,士气突然崩溃,全体逃散
中下级军官哗变,胁迫将领投降
将领意识到必败的命运,为避免伤亡投降
战至弹尽粮绝,乃至昏迷不醒,被抓俘虏
而明清之际,出现了大量的ABCDEF类投降,可就是罕见G类投降。
这可以说是明朝灭亡的主因。17世纪开始,欧洲民族国家成型,在之后再也没看到过可怕的A类投降了,2类投降也很少
反观日本,明治维新后的历次战争,抛开对正义性的讨论,多为G类F类投降。
投降的原因:
1)	自愿
拿破仑第一次战败后被流放厄尔巴岛。他逃出来后,身边只带了很少的随从。但是在他“向巴黎进军”的过程中,被派来围剿他的军队一批批的叛变,主动加入他的麾下。这就是典型的自愿投降。
2)	收买
金钱、财富、地位,都可以用来收买。例子不胜枚举。
3)	畏惧
在仍然很有战斗力的情况下,仅仅进行小规模战斗甚至未进行任何战斗,即向对方投降。
4)	战败
经过战斗之后, 
投降的主导者
最高指挥
中级军官
下层士兵

投降的方式:
1)	加入
投降之后,成为对方的一员,或者成为臣属、附庸、傀儡。比如在明末,
2)	战俘
这是最常见的情况。
3)	和谈
有条件投降或者停火,然后达成一个协议。

实际情况往往是多种因素的混合。比如在畏惧的基础上加以收买,效果就会好很多。

改朝换代的时候,甚至在异族入侵的时候,


;劳苦大众,不该称之为“奸”,而是一种漠然的态度。


,为何?


有人觉得美帝自由化倾向严重,
遥远的西方,中世纪时各种“奸”的数量也不遑多让。可是在17 世纪以后,民族国家逐渐成形,各国国奸的数量开始直线下滑,战场上的士兵越来越忠诚,民众里的“带路党”越来越少。


为什么中国提倡了几千年都没有成功,但是西方几百年间就形成了非常团结一致的民众基础呢?我认为,这跟一个行为的“正义度”和受益对象有关。
民族认同感。
人是社会性动物。这就决定了在维护自身利益之外,人也天生就有维护本族群利益的倾向。忠——广义的忠——是维护大族群的利益的行为,孝无疑是维护小群体利益的行为。
( 1 ) 概率云

是不是符合原属敢、恩情感和利益链,就一定能获得效忠呢?不一定。
每一个条件、每一个因素,都会改变忠孝的概率,但几乎不可能把概率变为1或者0,也就是说,很难有绝对忠孝或者绝对不忠孝。
所以,我用物理学上的“概率云”来形容这种情况。我们可以大概了解一个人群是否会忠孝,但却无法了解一个人是否忠孝。
四、利他行为

这是一种霉菌
如果
五、对照检查
一般来说,孝是可以自发形成的。西方也有孝道,只是没有提到东方的高度。你看好莱坞大片上,父子情不是一幕幕上演么?
民族国家
英国、法国、德国、美国和日本
原属感
只有美国比较特殊。在美国的教育下,
恩情感
利益链

忠的三个特征却未必一致,也就是说:
• 清末
我们再来看看清末的情况

我们在欧洲,身在异乡为异客,相貌、语言、民族、国籍都是少数派,但我们仍然会对哪怕丝毫的歧视心怀不满。满清时期的汉人,具有压倒性的人数优势,是社会财富和文化的主要创造者,却在自己的土地上沦为三等民族,你还要他们老老实实做顺民,当他们都是傻子吗?
是的,三等民族。满人是最高等的,汉人是最低等的,中间夹着蒙、藏、回等各少数民族,因为满清要靠他们控制人数占绝对优势的汉人。
当我们是傻子吗?看来很多人确实是,因为这种说法已然被社会主流所接受。真是个令人沮丧的发现。
恩情感

利益链

教育
这方面的教育倒是在努力进行。
目标:是为国服务还是为清帝服务?这方面的争论一直在进行
原因:
手段:推翻满清建立共和还是君主立宪?

氛围
这种氛围显然没有真正形成过,任何一个中国封建王朝都没有赢得民众的普遍自发效忠。
宋朝略有例外,由于宋朝的政治氛围比较宽松仁厚,所以士人对于皇帝是比较忠诚的。但这只能算作普遍自发效忠的苗头
清末显然没有

如果美国把Serve your country(为国家服务)改为Serve your president(为总统服务)
1. 人民未必享受皇帝或社会的恩惠
2. 维护皇帝利益未必等于维护国家利益
3. 维护国家利益未必等于维护个人利益

在皇权社会下,忠的正义度是严重不足的。

隋文帝杨坚制定《开皇律》,唐律沿袭隋律,宋律又沿袭唐律,所以说《开皇律》在中国古代法律中的地位大概就跟拿破仑法典在现代法律中的地位一样。《开皇律》有“十恶不赦”,其中恶逆、不孝、不睦是与孝相关的,谋反、谋大逆、谋叛、不义是跟忠有关的[1]。皇家理所当然的把“忠”提到了最高地位

如果

行为本身的正义性

我们必须肯定两个前提,1. 古人不是傻子;2. 小民不是傻子。
六、倾国之战
君主专制国家并非无法培训出一支忠诚的军队。事实上,如果用人得当、不吝赏赐,练出一支战斗力强悍的常备军还是没问题的。

如果两个国家组织水平都很低,把精锐常备军耗光之后,自己的统治都会难以维持,战争也就难以继续。
如果双方组织水平一高一低,差别很大,组织水平低的国家经常会失败,即使双方的表面实力差距很大。明朝就是被一个部落形态的野蛮人政权灭亡的。
如果双方组织水平都很高,而且都具有足够的战略纵深,使得战争不会迅速结束,这场战争就可能打成倾国之战。
生活设施会被摧毁,生活水平大降;国库会迅速耗干,只能用债务支撑,甚至连发债都筹集不到军费;常备军会耗光,必须把平民短暂训练后送上战场。

但是,这样的国家打不了倾国之战。
舍得砸钱


红军时期士兵的忠诚度很高,大约可以这样分析:
1. 农民享受了红军的恩惠——土地
2. 通过宣传,农民们感觉可以建立一个永久维护自己利益的国家

因为宋朝“皇帝与士大夫共治天下”,

七、美国
美国是个自由氛围浓厚的国家。
在《强国篇》里,我论述了强悍国家的必要民族性:勤劳、严谨、勇武和实用主义。
所以,美军的战斗力跟战争的性质有很大的关联。
如果战争性质是“不义”的,美国内部的反战势力是非常强大的,以至于根本无法组织起一场倾国之战。

在红朝,“忠党”高于“忠国”。九长老决策制比皇帝大权独揽好,因为权力分散了很多,受到了更多的限制。忠党也比“忠皇”好,至少这个对象更有价值,更能代表本族群利益——相对于一个至高无上的皇帝而言。但忠党明显还不够好,

[1] 十恶的内容:
	1 谋反,“谓谋危社稷”,即阴谋以各种手段推翻现存的君主制度。
	2 谋大逆,“谓谋毁宗庙、山陵及宫阙”,即企图毁坏皇帝的宗庙、皇陵和皇宫。
	3 谋叛,“谓谋背国从伪”,即企图背国投敌的行为。
	4 恶逆,“谓殴及谋杀祖父母、父母,杀伯叔父母、姑、兄姊、外祖父母、夫、夫之祖父母、父母”
	5 不道,“谓杀一家非死罪三人,支解人,造畜蛊毒、厌魅”。这里造畜蛊毒和厌魅是以巫术害人的行为,和杀一家非死罪三人、肢解人的行为一样恶劣,后果严重。
	6 大不敬,包括盗窃御用物品、因失误而致皇帝的人身安全受到威胁、不尊重皇帝及钦差大臣等三类犯罪行为。
	7 不孝,即控告、咒骂祖父母父母;祖父母父母在,另立门户、分割财产、供养有缺;为父母服丧期间,谈婚论嫁、寻欢作乐、不穿孝服;知祖父母、父母丧,隐瞒不办丧事;以及谎称祖父母父母丧。这些行为在性质上,与恶逆罪一样,都是对尊亲属的侵害,只是侵害的程度更轻。
	8 不睦,“谓谋杀缌麻以上亲,殴告夫及夫大功以上尊长、小功尊属”。缌麻、小功、大功是根据服制确定的亲属范围。缌麻亲是指男性同一高祖父母之下的亲属,小功亲是指同一曾祖父母之下的亲属,大功亲是指同一祖父母之下的亲属。同一亲等的亲属还有尊卑的区别。
	9 不义,“谓杀本属府主、刺史、县令、见受业师。吏、卒杀本部五品以上官长;及闻夫丧,匿不举哀,若作乐,释服从吉及改嫁。”
	10 内乱,“谓奸小功以上亲、父祖妾及与和者”。“和”,指通奸。
效忠篇
当前是个“辫子戏”流行的时代
不要把屁民当傻子
是强烈
“自发效忠”是认同感带来的,认同感来自于利益的一致性。

排队枪毙

16~19 世纪,是军事史上的排队枪毙时代。虽然火枪的射速很低(从三分钟一发逐步进化到一分钟两发),但是已经展现出对弓箭的优势。火枪战术的变化过程将会在《战争篇》讨论,在此我不打算探讨,
士兵们哭泣着向前补位。
“骑士-扈从”的模式只适应于小规模战争。
热兵器在十四世纪
在冷热兵器交替的
无数次,我看着《滑铁卢之战》、《爱国者》《Antietam 之战》《》


论文化
在西方,英法德美都没有大规模的叛变,

效忠度与什么有关?效忠对象的合法性。

但如果是两个实力相近的近现代国家,一场战争打成了“倾国之战”,国家根本是无法在经济上负担
甚至,到了战争的后期,连士兵的薪水和补给都无法保证。

现在中国人的效忠对象是谁呢?

我们效忠的是中国吗?

理论上应该是?

因何而效忠,也会因何而背叛。

为了钱效忠,就会为了钱背叛;

八、异地为官
中国一直实施的是异地为官“流官”制度:县级以上行政区的领导人,不得在原籍为官,也不得在一地久任。这一政策的源头可以上溯到秦始皇的“废分封,行郡县”,经过几千年来的逐渐演化,成为了默认的制度。
本地为官有两个危害:
1)	容易树大根深、盘根错节,形成地方一霸
2)	官员有可能维护地方利益,抗拒中央法令
在古代,交通和通讯是巨大的障碍。中国幅员辽阔,对偏远地区的控制力不足。中央集权最怕的就是地方叛乱,宋太祖赵匡胤说:“一百个文官贪污也不如一个武将造反的危害大”【缺少引用】。所以,皇帝要千方百计阻止官员在地方上“生根”。可以说,异地为官是必然的选择。
现代中国理论上已经没有了地方叛乱的担忧,但中央集权、强干弱枝的习惯保留了下来。

在西方,绝大多数的地方长官都是本地人,这是民主选举制的必然结果。竞选时,外地人很难获得本地民众的认可,很难当选。
西方怎样应对本地为官的两大危害呢?首先,地方行政长官有任期,如果干得太烂就上不了台。其次,西方人觉得地方长官维护地方利益是天经地义,拍上级马屁、不顾地方利益的长官赶紧“滚鸭蛋”。各地长官都要维护本地利益,最后通过一个上层的议会来形成妥协,人家就是这么运转的。另外,从效忠对象的角度说,他们效忠的是国家,
经常被五毛拿来做例子的,是芝加哥市市长被父子俩“霸占”42年。我们来仔细看看这件事:
父亲Richard J. Daley 于1955-1976 担任芝加哥市长,儿子Richard M. Daley 于1989-2011 接老爹的班,颇有世袭的意味。但且不说父子俩中间隔了十几年,仅就政绩来说,父子俩官声还不错,似乎不能反映民主制的伪善。

本地为官只有一个缺点:不利于中央集权。除此之外,全都是优点:更了解当地情况,在熟人社会里更爱惜羽毛,对家乡有感情,更希望把家乡经营好……
或许有人要问:地方上的局长、科长大多数也是本地人,该怎么贪怎么贪,没见他们怀着爱家乡的心为人民服务啊?
我的回答是:
1)	中国的义务教育中,关于爱家乡的内容很少。“爱的教育”的主体是爱党,其次是爱国,捎带着爱人民,只有一点点关于爱家乡。
2)	中国官场已经烂到了骨子里,对家乡的涓滴之爱根本不足以冲淡贪腐的浊流。


效忠
当效忠对象发生变化的时候,
拿破仑可谓战功赫赫,但在他称帝之后,很多人对他极其失望。依我看,如果拿破仑不称帝,而是成为法国总统,利用威逼利诱的方式独掌大权一辈子,那么犯法战争的结果会完全不同。当时,法国的民主、自由思想已经深入人心。在他的26 个元帅中,厌恶他称帝的就有很多 :奥热罗、贝尔纳多特、贝尔蒂埃。
明朝廷,作为唯一有效的效忠核心,一旦这个效忠核心不存在
比如美国,其效忠核心是一套价值观
但是当美国遭受攻击的时候,其自发抵抗的能力是很强的。
( 1 ) 国共争霸

TG如果今天很多创业型公司一样,主要的工作是“画大饼”。在那个时代,能够认清共产主义内在问题的人极少。
应该说,在TG露出獠牙之前,形象还是很正面的。
只有很少的人,像胡适这样的思想家,才能意识到TG的本质:
国民党治下,是“自由有多少”的问题;TG治下,是“自由有没有”的问题。
( 2 ) 现代中国
中国的教育体系试图往学生的脑子里写入一个等式:
党 = 政府 = 国家 = 人民
我们都不止一次听到“党和政府”、“党和国家”、“党和人民”,
何德何能!何德何能?一个党派可以凌驾于政府、国家和人民之上?
根据我的经验,很容易的,就可以在留学生中引发思考,打破原有等式,并建立新的不等式:
党 ≠ 政府 ≠ 国家 ≠ 人民
接受过高等教育之后
红朝的效忠系统是经不起考验的。
在中央控制系统良好运作的情况下,应该说红朝的战斗力还行。但如果碰到了超强的对手——说白了就是美帝——很可能会在开战后不久被切断中央控制系统和
中国人自发抵抗的意志会很薄弱
在《战争篇•战争泥潭》里,我论述了
美帝有战争道德

九、总结
忠孝的本质是认同感。
效忠对象是否代表了效忠者的利益
原属感
恩情感
利益链
教育
氛围
异地为官


中国的朝代

如果仅仅从“皇家维护统治的水平”的角度说,确实是在螺旋上升的。

应该看到,日本天皇能够“万事一系”,跟天皇没有实权有很大关系。天皇很奇妙的在中世纪的社会发展中,沦为了

如果最高统治者权力很大,但是又缺乏“个人崇拜式”的精神统治,而且没有通过某种方式削夺下层的战斗力,会出现什么情况?古罗马的皇帝换的跟走马灯一样,

首先,“罗马皇帝”跟中国皇帝根本不在一个量级上。罗马皇帝的权力要小的多,对国家的控制力也弱的多。而且,很少有罗马皇帝能够成功世袭。

虽然罗马是贵族统治,经常有平民出身的军头成为罗马皇帝?

大众媒体

印刷和造纸 - 印刷和造纸出现后,大规模的教育即成为可能。通过教育,使得民众产生强烈的自发认同。

中国人现在的忠诚度如何?很差

别看网上一片喊打喊杀

要知道,这些都是“廉价忠诚”。正如鸦片战争之时,

难道明末、宋末的汉人就真的甘心做“鞑子”的奴才么?在需要为忠诚付出代价的时候,

进一步的,如果强敌肯于对被征服者分化瓦解,很快就会形成雪崩。

绿教则不同。绿教有一个打不掉的精神核心(这里且不论其正邪),所以不管各国怎样蹂躏其政权,来自民间的小规模反抗是连续且坚决的。

扩张性
排外性
政权核心
精神核心
自组织能力

目前的美国,主动扩张的时候战斗力并不强,而且

绿教打阵地战能力很差,但由于高攻击性

然而,是什么造成了德国的外交败局呢?

是民族自豪感。

所以,在本书后面的章节,你会看到我提出的很多攻击性策略,或许

向不懂围棋的读者阐明

通过巧妙而安全的压迫,迫使对手主动退缩、让度一部分利益,或者使得对手心浮气躁、贸然反击、露出破绽。

打小不打大,打少不打多

吃一口、歇一会 每次成功的扩张之后,都会造成邻近国家的警觉和关系紧张。

我所说的战略听起来当然是好的,而在执行中是极难的。其中,最难保证的就是

战略的连续性 现代政治往往要求有限任期制。故此,怎样让继任者还能保持自己的策略,实在是极大的难题。

国内其它政治势力的影响 比如日本,少壮派军官通过兵变,实际上控制了政府。

一次失败后的自暴自弃,连续成功后的信心膨胀。

有些高层战略是不足与外人道的(所谓“下一盘大棋”)。

我写这段话并不是反对现在的政治开放。

俾斯麦在位期间,可以说成功执行了个人的战略。但生前鼎力支持俾斯麦的威廉二世死后,很快功勋卓著的老臣俾斯麦就下台

说完这些,我们再来看东周魏国。

虽然黄河流过魏国,但黄河自古就不适合航运。

\begin{itemize}
	\item 推行教育,增强单兵战斗力。普鲁士是最早推广男童义务教育的。
	\item 整修道路,增强内线机动性。
	\item 积极宣传,
	\item 运气不错。普鲁士的上升期也是险象环生。拿破仑攻入柏林之时,几有亡国之险。
\end{itemize}

如果我们不给魏国“爆黑科技”的话

不断的挑拨,
马耕与骑射

特种战术

\emph{等}。等周围某个国家手贱,先对其它国家出手。然后抓住良机,同时削弱两者,增强自身。

秦楚赵都很强,哪块骨头都不好肯。韩国不敢随便打,因为太弱了,一不小心打亡国了,就会使得魏国同时面对多方面的责难和威胁。燕国不直接接壤。剩下的,只能打齐国了。齐国本身很富裕,但是在整个战国时期都没强过。

燕国跟齐国仇比较深,而且燕国差点就打下来整个齐国。如果能再次挑拨燕齐之战

夺取战略要冲

魏国周围的战略要冲太多了。

荥阳往西就是不远就是洛阳。洛阳就是周王城的所在地。



比如抗战时期


长期以来,

人才阶层

“人才阶层”是我生造出的一个词。跟一些人的讨论中发现,很多人根本就不管我阐述的“精英统治”到底是什么,看到这个词就开始强烈反对。痛定思痛,我觉得必须换一个词来阐述我的思想。

某些人试图用“精英统治”

比如某些党派声称:本党派就是社会精英的大集合,不为什么,我们就是代表了精英。

21世纪最缺的是什么?人才!

这个国家,要通过某种方式选拔人才,并让人才来统治。

人才是相对的。

《国家是怎么失败的》这本书着重谈到了榨取式和广纳式经济的差异。



谈精英统治

凝聚力

目前看来,最有效的扩张凝聚力是民族主义和宗教。因为要扩张,就免不了要打仗。要打仗就免不了损失财富和生命。怎样让人承受这么大的损失呢?

二战德国和日本采用了民族主义。

