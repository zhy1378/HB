\chapter{制度篇}

郭嵩焘[1]在伦敦做公使的时候,经常去法院旁听。他觉得很奇怪的是:他们审个小案子都似乎要穷追细节。郭嵩焘不明白,这“穷追细节”四字就是法律保护人民的根本。黑格尔在《历史哲学》里提到中国人把井口造得很小,为的是防止他人跳井。大清律例确实规定跳井人的家属可得到井主人家的全部财产,甚至井主会被满门抄斩。
古代太远;对中国人心理影响最大的还是清朝的三百年。解放后人们互相吵架还有人说:“把我逼急了我到你家去上吊!”可见中国人对法律概念何等的模糊,而法律本身又是何等的模糊!台湾的“检举共谍”,大陆的“大字报”随意指控他人,都可能对人造成伤害和不安,无需“穷追细节”。
中国人因此吓破了胆,活的战战兢兢,匍匐在统治者的脚下。西方人民一般不象中国人那般怕事,因为法律会“穷追细节”,会保护公民的合法权益。

2007年南京发生了彭宇案。这是个极简单又极复杂的案子,想必你们有所耳闻。我不再详述,给你们PDF,在附件中。 我没有引用来自国内网站的内容,因为那些内容明显经过了政府部门的审查和篡改。

想写彭宇案,是因为今天看到了一篇报道:
【长沙老人晨练发病倒地 49人经过无人报警】长沙一名61岁的老人出门晨练,突发心脏病仰面倒地,倒地后的33分钟内,先后有49人经过他的身边,但无一人报警,直到第50位路人拨打了报警电话,但此时老人已停止了呼吸……

彭宇案让中国人对“碰瓷儿”产生了深深的畏惧,不得不把自己的善念隐藏起来,使得这个社会更加冷漠。我们的政治书上总说西方人冷漠,可是我在欧洲这么多年,也遇到过大街上突然倒地的人,而他们旁边总是立刻围上一群人来救助,警车和救护车来的也很快,甚至还有一次出动了直升机。我也曾经冬天深夜在路上发现一个躺在草丛里的老头,我毫不犹豫的就问他是否需要帮助,因为我知道这里的暴力犯罪率极低,而且不会遇到讹诈,所以我才可以尽情释放我的善念。那个老头告诉我,他只是走累了,想在这里躺一躺,不需要帮助。

在彭宇案的过程中,最该被指责的就是那个法官了。遮遮掩掩的辩论过程,漏洞百出的证据链,莫名其妙的判决书,给全体国民好好的上了一课:看,这就是祖国的司法。

共党一直在提倡道德,这不是它的专利,因为国民党也提倡道德,之前的两千多年,儒家一直也在提倡道德。可是有用吗?中国是个量产贪官的国度,官场是个大染缸。
能把道德和制度的关系的看清的人不多,在中国要首推胡适,他说:一个肮脏的国家,如果人人讲规则而不是谈道德,最终会变成一个有人味儿的正常国家,道德自然会逐渐回归;一个干净的国家,如果人人都不讲规则却大谈道德,谈高尚,天天没事儿就谈道德规范,人人大公无私,最终这个国家会堕落成为一个伪君子遍布的肮脏国家。

伪君子,多好的词。让我不禁想起来中小学时候,上级来听课的之前,我们要把同一节课排练好几次。到了德国才发现,这边听课的时候,要尽量维持平常的面貌。还有那一次次的学雷锋,我带着同学去审计局打扫卫生就为了收获一封感谢信。上级来检查的时候,学校要动员学生搬家里的花来“装点学校”,我也总当积极分子,至少要搬两盆,最多的时候搬过五盆。到了德国才发觉当年的荒唐可笑,也更加客观的审视中国社会的虚伪和无耻。更可怕的是,这种虚伪和无耻是从小就灌输的,是通过公共教育系统给每个未成年人打上的烙印。

注1:可能是康有为。我以后想要出一套文集,把我的思想全部写出来,目前写给你们就当是练兵。
我读书很快,也很多。我在读书中寻找我觉得有趣的内容,亦或可以证明我观点的东西,无关的内容就略过了。很多细节我是不关注的,比如人名、地名、时间什么的。写文章论证我观点的时候就要做到有理有据,有史可查,必须把这些细节列出来,所以是个很麻烦的事情。为了不太影响写作速度,我把我记不太清的地方标明,留待以后慢慢查证吧。
我初中和高中历史和政治都很差,这你们都知道。有几次还考不及格,不过我偷偷改过成绩。你们也别抓狂,毕竟时间都过了这么久,而且我只改不及格的。再一个,那种政治历史考试我是不可能考好的。我现在是出了名的知识丰富,这种相对的丰富是从小学就开始的。中学时代我的政治历史知识一点也不比别人差,课间经常是前后左右转过头来听我胡喷各种稀奇古怪的历史知识。可惜,一考试我就抓瞎了。过了十来年,我总结三点原因:
1. 历史书上分“大字”部分和“小字”部分。容易考的内容是大字,偏说教;不容易考的内容是小字,讲的是一些历史细节。我就不喜欢看大字,偏喜欢看小字。
2. 我是“读书观其大略”型,不喜欢背时间地点年代什么的,除非我感兴趣。然而考试偏偏喜欢考细节。
3. 我有自己的观点,偏偏和书上的不一样。所以,问答题胡抡的时候抡不到点子上。

还有一点让我高考时候很吃亏:我中国史喜欢的是北宋和春秋战国,最讨厌的就是1644满清入关后的历史;世界史喜欢的是希腊罗马。而高考中国史只考1840后的内容,世界史只考1492后的内容,尤其是党史占了很大比例,这恰恰是我最不喜欢看的。

最后附上一张图片,美国人还是很豁达的。

四川省南部县县委书记何修礼,在县“春晚”上展现了二胡“才艺”,从此震惊网络。这里有一个链接:

http://news.163.com/14/1017/07/A8O9SSS100011229.html

看到了吧,这位县委书记连街边讨钱瞎子的水平都赶不上,却成为乐团里最重要的独奏者,因而被称为“午夜惊魂”。可是,在嘲讽之后,我们又能看出什么呢?

依我看,这充分证明了中国的权力结构出了大问题!

在中国,权力的触角伸展到社会的每一个角落,甚少有体制外的人可以与之抗争。敢于向领导说“不”的勇士,轻则做冷板凳,重则丢官罢职。说到这里,我还是得感谢社会的进步,若是放在几十年前,那就是轻则发配新疆、黑龙江做“两江总督”,重则身死家灭。我小时候议论时政的时候声音稍微大了点,就被爹妈噤若寒蝉的制止。是的,现在我可以充分理解,经历过那个时代的人,怎能不对政治迫害打心眼儿里畏惧?

回到这件事上。这位书记的演奏水平,上台前难道没有被指挥检验过吗?或许有,或许没有。总之,在权力的高压下,指挥连个屁都没放。