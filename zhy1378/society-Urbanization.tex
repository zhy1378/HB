\chapter{城市化}
\begin{quoting}
习上台后,准备推进新一轮城市化。预计投入十万亿,推进城市化进程。
\end{quoting}

\section{最理想的城市化}

\begin{itemize}
\item 交通顺畅
\item 坏境优美
\item 治安良好
\item 市内无大型重工业,工业区与居民区分离。
\item 医院、学校分布较为均匀。
\item 最好将最大的城市控制在两千万人以内。
\end{itemize}

事实上,中国的城市化恰恰走向了反面:
\begin{itemize}
\item 大城大堵,小城小堵,无城不堵
\item 污染严重。我去过的所有城市都笼罩着一股油烟味
\item 
\end{itemize}

\section{畸形大城市}
很久之前在《中国国家地理》上看过一篇文章,说改开之前就有一个北大教授预测中国会大城市化,而不是小城市化。这一番言论很不被党中央喜爱,因为畸形大城市化分明是资本主义帝国主义的表现。所以,他好像去边疆挖鸟粪了。【需考证,好像是在《中国国家地理》上看的】

经济规律终究是规律。中国的大城市不可抑制的发展为畸形,人口达千万的超大城市已经有八个【缺名称】。虽然政府一直想学习欧洲的中小城镇化,但是一直不成功。表面上的原因,是经济发展不均衡。其实,经济发展不均衡是必然,“这么不均衡”是偶然。

虽然(?)教授预测了中国的大城市化,但是我觉得他没有指出最根本的原因:中国的权力太集中了。考虑到他所处的历史环境,他有可能看出来却不敢写出来。否则挖鸟粪的机会都没有,而是直接埋地里“化作春泥更护花”。

\section{社会资源分配}

我到欧洲的感受就是分散和均匀,很多著名的厂商、大学、品牌、美景都在小城,还有很多人觉得大城市不好,拥挤、嘈杂。中国人口和产业的集中是由于中国社会资源的集中。这里的社会资源指的是交通、医疗和教育。那为啥中国社会资源这么集中呢?权力太集中了,地方上自治的权力太小。


\zPar
高等教育资源的影响力是非常大的。
一所大学每年就可以为一个城市带来五千左右受过良好教育的青年劳动力。
而落后地区的受过良好教育的年青人源源不断的被“吸”走,只会更加落后。

要想均衡的城市化,就必须斩断中国的“邪恶正反馈”:大城市要啥有啥,且越来越要啥有啥;乡村要啥没啥,且越来越要啥没啥。
北京

\zPar
以教育资源为例,我查到的资料是:2002年河南重点率3.8\%全国最低,北京28\%全国最高(待考证)。如果没有权力的影响,北京怎么可以有二十多重点大学?正常情况下五六所就可以了,而咱河南连一所好大学都没有。

中国的社会资源由权力分配,向权力倾斜。故此有权的地方就有社会资源,有社会资源就能吸引人口。我要是回国,也得想办法弄个北京户口。要是我的孩子还参加河南高考,我觉得愧对子孙。

\subsection{中国的城市化}

欧洲的城市化比中国温和的多。欧洲各地区发展比中国更平衡,而且欧洲的城市化比中国持续时间长很多。
主要的社会资源包括交通、教育、医疗、零售。其中,适合国家调配的是交通和教育。只要城市发展到一定程度,医疗和零售必然会跟进。

据我对河南一些县城的了解,城内几乎没什么工业。主要分布的是政府机构、中小学校、医院、餐馆、娱乐场所和零售业。其中,政府、学校和医院汇集了全县的财富。

降低行政机关的比重
	清除冗官、冗员

高等教育的

这其中,最容易改变的就是交通条件。

大政府
在《改革篇》里,我也提到了

\subsection{高速路收费}


全面废除收费站

以目前的技术是可以实现的

但是


想要进行

\section{征地}

\subsection{神奇的一平米}

东京成田机场有一平米土地,其产权为数千人共同所有。如果开发商想要获得这一平米,从法律上就必须与数千人同时打官司。

上世纪60年代成田机场的征地过程中,政府确实有错,遭到了居民的抵制,也得到了左翼人士和日本共产党的支持(极左派是社会发展的搅屎棍)。之后,日本政府努力寻求谈判解决,可惜一直无果。黑头发拖成了白头发,拖到了2005年,日本首相都亲自跑去道歉,但这些钉子户就是不肯退让。

成田机场原计划修3条跑道,第1条跑道修了12年,第2条跑道因为避让“钉子户”而变得歪歪扭扭,而第3条跑道至今仍未修建。 甚至在机场里还有民宅,导致机场夜间无法起降飞机,以免影响居民休息。

看这架势,许诺的拆迁补偿费肯定少不了。【没有查到,最好用实际数字】这些钉子户为啥就不搬迁呢?


市场经济的大前提是“理性经济人”假设,它认为所有人都是贪婪且精明的,故此,每个人都会谋求个人利益的最大化。

如果成田机场的钉子户们足够理性的话,早就该拿着大笔补助款,找个风景优美、气候宜人的疗养胜地过自己的小日子了。头发都白了,累不累?还能折腾几年?他们坚守在机场几十年,说好听点是为了信念,说难听点就是“别扭”。

这也是\emph{市场失灵}(Market Failure)的原因之一:\emph{人并不总是理性的}(啊,愚蠢的人类……)。

大型建筑和道路的土地必须连成完整的大块。其中只要有一小块地无法取得,整个工程都无法进行。

在征地时有可能出现由于非理性原因而导致征地失败,进而导致

市场经济讲究公平交换

这已经不能用市场经济的

诸位读者,你认为这些钉子户是正义的吗?
在公平交换的时代,如果有人坚持拒绝交换,似乎与“正义”扯不上联系。

美国宪法第5条修正案规定:“非依正当程序,不得剥夺任何人的生命、自由或财产;非有合理补偿,不得征用私有财产供公共使用。”

这条法律意味着,不能随便征地,但是在提供补偿的前提下,可以征用私人土地供公共使用。


“信念”就是非常不符合市场经济的东西。


故此,国家应该有权强制征地,但不能随意征地。大型公共设施用地和交通用地,
中国缺乏程序正义。建设方案应该早早公布出来,交由社会各界讨论。
补偿方案
天朝的拆迁问题主要是,
有人认为,对本国公民只能用警察,而且应该禁止在任何情况下对本国公民使用军队。但我并没有“政治洁癖”,我的政治观点是“实用至上”。在我看来,出动军队执行国内任务没什么大不了的。

“拆哪儿”(China)国的拆迁大队已经在网上激起太多的民愤。
在此,我

我们公正的说,“钉子户”都是正义一方吗?
中国的人口密度比美国高得多。

由于中国人均土地面积比美国少得多,因此更不能过度保护。否则,大型公共工程将无法修建。

想要认真的统计点东西可不容易。
比如
美剧《罪恶黑名单》(The Blacklist)第一季评分7.2,第二季评分8.0
这部剧拍的越来越好吗?不,是越来越烂了。第二季简直烂的不能看。
但世界上就是有
第二季有1306人评价
第一季有9188人评价。
2015-02-27

\subsubsection{房产税}
连续
我认为最好的征收办法就是:规定一个“免税住房居住面积”。如果某户家庭的人均住房面积小于免税住房面积,则无需缴纳房产税。
假设免税住房面积为80坪。如果一家三口有300坪,则需要为其中的60坪支付房产税。进一步的,可以制定分等递进的房产税。
鉴于各城市情况差别很大,所以各城市可以根据实际情况调整免税住房面积。
降低市民购买多套房产的愿望。

我设想的政策
拥有的住房面积高于这个给每个人规定一个

现代社会中,有些设施总要找地方修建。比如

1. 中央没人。你可以看看中央的大头目,极少河南人。在中国,没有权力就没有利益。所以河南在天下正中,土地平坦,竟然铁路密度不高、高速公路密度也不高,而且竟然将近一亿人共享一所211大学(还不是985)。由此造成的落后,导致河南出身的大头目更少。
如果是民主制,河南的众议员会很多(按人口比例),他们自然会嗷嗷叫的争取利益。But u know, 这种事情不会出现在中国。
1.	河南缺乏资源。因为河南是地理上形成较晚的冲积平原,土壤层很厚,适合农耕,可惜下面埋藏的矿产也少。而山西有矿产,卖矿产获得很多钱,当然也有超级严重的污染。
2.	河南、山西都算是低工业化地区,在这个阶段,农业财富、工业财富都会被人数所稀释。
拜仁已经进入后工业化阶段。这个阶段用智力创造财富,人力就是财富,人就是生产力,如果河南到了这个阶段,人多就意味着更多的财富。
3.	上世纪年代,全国官员都没那么贪,主要是胆量还没那么大。随着经济发展,沿海的官员贪欲和当地经济同步起飞。在河南,领导们的贪欲向沿海看齐了,经济还没看齐,显得河南的官员尤其贪。腐败导致经济进一步落后。

\section{铁路!铁路!}

中国经济最发达的地区全部跟“水”有关——珠三角、长三角、沿海经济带、沿长江经济带。只有一个例外就是北京,它的发达与权力有关,使得一个本应五百万人口规模的城市发展到了两千万。

现代工业除了少数布局在原产地和消费地,绝大多数都布局在交通发达的地区。

河南的省会本来是开封,

水运虽然速度慢,但运输量大、成本低廉。改开初期,中国的基础设施很差,沿海地区相当于拥有“天然铁路”。 \footnote{平均运费每吨•英里,源:美国交通部2009年《国家运输统计》 \url{http://www.c2es.org/technology/factsheet/FreightTransportation}}
方式	空运	卡车	铁路	船运	管道
价格(美元/吨•英里)	0.80	0.27	0.02	0.01	0.01

在柏林坐车到轻轨和地铁的各个终点站,都可以观察到

河南夏日中午的烈日非常可怕,能把人晒脱皮。因此农民都是天蒙蒙亮就下地,干到11点左右找地方睡觉,下午3点以后再上工,一直干到夜里。甚至黑的完全看不见了还能干一些活儿。

德国最富裕的两个州是Bayern和Baden- Wuerttemberg, 它们也恰巧是最南部的两个州。相比之下,这两个州距离大海是最远的。为什么它们可以

玩《文明》时,每当“铁路”科技即将研究完毕,就开始提前几回合将工人布置到
第一条铁路于1825年在英国建成。
我毫不意外又十分遗憾的发现,在中长期铁路网规划图上 ,河南占的比例仍然很低\footnote{中长期铁路网规划图  \url{http://map.hytrip.net/photo/350/6242299E02.jpg}}。

我并不是说西部不需要铁路,而是希望伟大的发改委考虑下效费比。一条横跨新疆的铁路可以绕河南好几圈,而且覆盖的人口数量远不如河南。

这种“被欺负”的感觉

其结果,既然

\subsection{卫星城}
解决大城市拥堵的办法之一是建设卫星城。根据我的观察(直觉),如果没有铁路,大城市周边的城镇不可能成为“卫星城”。

决定人居住位置的不是距离,而是“通勤时间”,也就是人从居所到达工作单位的时间。根据一项研究,半小时以上的通勤时间会对人造成较大压力,很少有人愿意花一小时以上通勤。事实上,绝大多数人的通勤时间在一小时以内。

铁路可以大大缩短通勤时间。

火车的速度在100km/h上下,如果郑州

比较我生活过的两个大城市:柏林和郑州。
我曾经在郑州创新大厦找到一份学生工,但是干了不到三个月就辞了。最大的问题,是周末要坐一个小时的车

柏林市内地铁速度约为26~33km/h \footnote{柏林地铁(德语)\url{http://de.wikipedia.org/wiki/U-Bahn_Berlin}}

都是直达的车,

\begin{description}
\item[速度] 地铁的速度是公交的2~3倍。如果通勤时间从60分钟缩短到20~30分钟,那就可以接受了。
\item[舒适度] 地铁的舒适度是公交车比不了的。
\end{description}


柏林是有340万人口的大城市,波茨坦是。有不少人在柏林工作,但是在波茨坦居住。
柏林到波茨坦大约有30公里。
柏林到波茨坦之间是铁路直达,
一个小时的时间,在铁路线上可以行进60

我亲历的情况,某主干道每三四年大修一次,但很快就坑坑洼洼。
中国有很多公路,或许在通过发达的铁路,交通时间只有21分钟(柏林主火车站~波茨坦主火车站)。

\subsection{春运人口铁路规划学}
我认为,统计一下各地的春运人口,就可以计算各地对铁路的渴求度。

春运人口多,意味着当地人口密集,并且工业化程度低。因此,大批人口只能外出找工作。如果能修通铁路的话,当地的劳动力就可以就地消化。

在西部修路动不动就是“战天斗地”、“克服万难”,开山架桥挖隧道;而在黄河中下游平原上,厚实、平坦、少地震,只需要用最基本的技术、最低廉的费用。

\subsection{私营铁路}
世界上的铁路多数都是国营的,但我觉得私营也没什么不好。

铁路的运价大概是公路的1/2~1/3,利润还是很丰厚的。

铁路的建设费用大概是500万欧/km,这个费用不算太高。

建好铁路之后,这段铁路就属于私人。比起公路,铁路收费更加容易。

从国内高速收费的情况来看,如果是20年收费期的话,差不多10年左右即可收回成本、还清贷款,剩下的都是净赚。

建铁路最大的问题可能就是征地。

\subsection{双层路}
中国虽然有960万平方公里的土地,但其中只有300万左右适合居住和农耕。所以,中国东部的人口密度非常大。伴随着经济发展,农民也学“精明”了,更不愿随意出让土地。因此我预测,征地会越来越难。

所以,在东部地区规划道路的时候就应该直接规划成双层路甚至三层路。尤其是国家投资的高速公路和铁路,最好实现一层铁路、一层高速、一层普通公路。

虽然双层路会多花不少钱,但这笔钱是值得的。如果能保证质量的话(在天朝并不容易),双层路可以持续50~100年。等它们报废的时候,估计就要用3D打印和纳米生长技术来修路了。

\subsection{小型集装箱}
“搞车皮”很难意味着目前铁路的运力不够,矿石、煤炭、粮食和大集装箱等大宗货物都运不完,也就不需要在意小宗货物。但如果铁路运力增长到一定程度,运输大宗货物之外还有富余,要怎么运小宗货物呢?

很小的货物,可以打成包裹,进行邮寄。但还有些不大不小的货物,不值得用集装箱,又难以邮寄。“拼集装箱”是一个办法,但并不方便。

原理很简单,就像两张32开的纸正好拼成一张16开的纸。用同样的思路,假设由铁路运输的最大的封口集装箱(开口集装箱一般运矿石类)为A型,则两个B型集装箱正好可以严丝合缝的填满一个A型集装箱的位置。一个B型的位置可以放两个C型、四个D型、八个E型……

按车厢规格为16.66米长,
以此为标准,还可以规定卡车的大小。比如最大的卡车可以运输A型集装箱,小一点的可以运输B型,以此类推。
如此,就可以比较方便的运输小宗货物了。

\subsection{铁路狂热症}

我刚在网上发布这种观点时,就被称为“铁路狂热症患者”。反对我的观点的人,主要是出于

铁路是夕阳产业,全世界范围内铁路的重要性在下降。在欧美,有不少铁路、车站已经放弃了营运,。

为什么我还在重视铁路呢?

根据调查,德国的火车票价比长途汽车票价贵1/3左右。
在德国,柏林到慕尼黑的票价是,而

我的理由如下:
\begin{description}
\item 运输距离在500km以上时,铁路运费开始明显低于陆运。德国面积比较小,在北部港口画个500km的圈,已经覆盖了整个北德。如果从中国
\item[]
\end{description}
在我看来,铁路的很多费用都是可以省下来的。火车最长可以到
运量的波动

低污染
“著名经济学家”王福重说:“”
我得说,如果资源无限、环境容量无限的话,这句话比较对。

\emph{城市规划千万不要学美国}
美国有很多值得学的地方,但是不包括城市规划。美国可居住面积是中国的3~4倍,人口不到中国的1/4,人均耕地面积是中国的15~20倍。所以,在美国有很多非常低矮的房屋,因为土地够多够便宜。美国有很多摊大饼的城市,六层楼就算一个制高点了。也因为人口稀少,所以很多地方修铁路不如修公路。

同理,也不能学习加拿大、澳大利亚。可以学习xiou西欧,但西欧的人口也远不如中国密集。我想,最有价值的老师是日本、韩国和新加坡。

如果中国人口减少到1/10,那么我提倡的城市规划肯定是另一种模型。但很遗憾,亚洲东部的人口就是这么密集,我们必须寻找密集人口的集中生存策略。

用经济手段打击一家一户的六层以下的自建房。
所有5万人以上的城镇要修通铁路。
回收宅基地,还为林地。

允许自由迁徙,社会资源按人口比例配给,


既然中国已经有这么多人口。


比如
因为异地运输的成本很高。

铁路客运的收费是个不大不小的问题。自动售票机的发展
连郑州的公交车都是司机兼任售票员了,北京的公交车竟然还需要一个额外的售票员,实在太落后了。
Hallo an alle,
Ich komme aus China. Ich habe
%Ein Freund von mir ist ein Geschäftsmann, er möchte ein Container von Hamburg nach Berlin transportieren. Zu dem Fall hat er ein LKW gemietet.
%Bei viele Bahnhöfe habe ich gar keine Umladungsgeräte gefunden. D.h.,
%Deshalb möchte ich wissen, warum ist Bahnticket viel teurer als Busticket?

这就像对付恐怖分子一样,他们根本就不是理性经济人,经济手段对他们的影响很小。
这样的一个车头就相当于76个卡车
我不否认汽车工业很重要。

\section{房事}
\subsection{租房}
在西方,租房住的比例非常高。中国租房率
租房也提高了人口流动性。

\section{城市规划}
“美”是对比出来的。到欧洲之前,我觉得中国的城市都还凑合。在欧洲转了几圈之后,我感觉我去过的所有中国城市,其规划都可以用一塌糊涂、糟糕透顶、惨不忍睹、乱七八糟、丑陋不堪等词汇中的一个或几个来形容。
能创造“美”的只是少数人,绝大多数人只能欣赏美。如果没见过更美的,就会感觉不到
到北京之前,我一直觉得脏兮兮的街道、蛛网般的电线。榨干全国
到德国之前,我既没有看过城市规划方面的书,也没有根据我的二战知识,德国被炸得跟月球似的。战后重建应该是走现代化道路吧。
上大学之前,我一直觉得“俺们那旮旯”还可以。

如果说美西战争 时美帝还有那么一丢丢风险的话,到了20世纪如果美帝还想开疆拓土的话,墨西哥连“弹腾”一下的能力都没有。愿意在

然而其主要原因,是政府太蠢。

西方许多城市成型于汽车时代到来之前,甚至是火车时代之前,街道狭窄情有可原。但是

有些网上五毛那西方一些

的确,欧洲有堵车,也有空气污染。

下午六点的柏林也会堵,柏林人也抱怨,但这里的堵车是经过一小时左右就自动消除了。不像北京,全市堵成一个大型停车场。柏林的交通系统是值得称道的,在市区内任意一点走不太远就可找到地铁站,经零次或一次换乘即可到达市内绝大多数地点。所以,年轻人不买车很正常。

夏天堵在郑州的公交车上,憋闷的我简直想自杀。

这里必须说明的是,

在城市规划中,一定要预见到

\subsection{停车场}
一些十万人规模的县城都已经出现停车难的问题。究其原因,还是愚蠢的政府搞出来的愚蠢规划。
我估计,每四层人住的楼房,就要设计一层停车场。停车方便、快捷,也会减小公路上的车流量。等待停车,

\subsection{绿化}
中国城市的绿化水平要远远低于欧洲城市。
我很喜欢攀援植物,覆盖了爬墙虎的建筑很有感觉。我认为,应该全国推广攀援植物。既然城市里没有留出绿化空间,那攀援植物就是良好的替代品。而且,国内老建筑往往没有隔热层,夏天热的能烤红薯。攀援植物可以吸收很多能量。
另一点,就是多种树。
能种树的地方,统统种树
很多中国城市都
其实那些草根本不适合中西部城市。与欧洲相比,中国的大多数城市不但炎热,而且缺水。华北的全年降水量不算少,但是特别集中,不涝就旱。华北土生土长的树木不但可以
树木还有其它好处,包括降低噪音、粘附尘土

\subsection{预留地铁线}
虽然有些城市暂时没钱修地铁,但只要人口富集到一定程度,修地铁是必然的选择。

\subsection{集中化解决}


\section{环境保护}



\subsection{垃圾}

垃圾问题已经实实在在的困扰了很多城市。

由于中国的权力十分集中,因此大城市的垃圾问题反而得到了解决。不管垃圾是怎样处理的,至少大城市的垃圾会被及时运走。而在很多中小型城市,我观察到了明显的垃圾围城现象。

\subsubsection{垃圾收集}

中国直到现在还没开始垃圾分类,这简直是个不可理喻的国家!

全民普及垃圾分类知识,制定垃圾分类法规,建设垃圾分类设施已经刻不容缓。

在我生长在

我希望我的故乡,也能成为一座花园般的城市。

\subsubsection{垃圾处理}

德国也有不少垃圾山,因为现代工业社会的垃圾量实在是太大了。
垃圾焚烧
有机垃圾占的比例非常大,
垃圾填埋

\subsection{饮水}
中国

\section{农民工}
自由劳动力的产生
自由劳动力的转移
劳动力的异地就业
劳动力的异地生活

\subsection{自由劳动力的产生}
“三农”问题可以这样来形容:\textbf{农民真苦、农村真穷、农业真危险}。但农民是不是特别累呢?与许多人所想的恰恰相反,农民并不太累。

农活有很强的季节性,在某些时间段的确是很累。比如,河南的学校以前在初夏有“麦忙假”,初秋有“秋忙假”,时间各为两周。假期里,半大学生都要回家帮着干活,因为活实在太多了。他们常常是从黎明干到十一点左右,然后找荫凉地吃饭、睡觉,以躲避暴烈的太阳;下午三四点又开始干,一直到天黑的伸手不见五指。

在全年的其它时间段,农活要少得多,可以说是相当清闲,有时候在冬季甚至完全没活干。



先来猜猜看,河南人均有多少土地?

答案是:按2009年的统计是1.23亩\footnote{河南人均耕地 \url{http://www.chinanews.com/gn/2014/06-09/6260538.shtml}}。你或许会说,这个人均耕地需要扣去城镇人口。其实,按照2011年的普查资料,河南农村人口达到五千万,超过一半。即使扣除一半的老弱,每个壮劳力也只有5亩地。而每个壮劳力在不使用机械的情况下,可以耕种多少土地呢?大约20亩。

河南农村的机械化程度当然是很低的,但初步的机械化已经极大的提高了生产效率。华北大平原的南部与北部冬小麦的成熟期有三周左右的时差,河南南部开始割麦子时,河北北部麦子还泛青。于是就有很多人出租大型联合收割机,每年从河南南部出发向北走,沿途边走边干活。原来要耗费全家壮劳力好几天的收割、脱粒、整理麦秸,现在用收割机一两个小时就完成了。


那这些农民在农闲期干什么呢?打牌、下棋、唠嗑、看电视。有没有农民读书、写作、搞研究呢?中国这么大,这种人肯定有,但那是个例。绝大多数农民就是那么浑浑噩噩的生活着。

有那么一个时期,别管电视剧拍的多烂、多漫长都有人看,就是因为农村的精神生活太空虚了。

即使我很喜欢每天下棋、打牌、,但连续玩几天我也会感到腻烦,想要找些书看。

年龄大一点了,我到农村长住的时候,总是要带些书。我起初以为农民是因为没有书所以才不看书,但是当我试着主动借书给他们看的时候,根本没有人看。我后来意识到,读书、读厚书也是难得的技能,是需要从小培养的\footnote{见《读书篇》}。


“苦”是因为他们缺少赚钱的机会,而沉重的社会和家庭负担——诸如老人治病、小孩上学、子女结婚——使得他们格外痛苦。我不止一次听到、看到这样的故事:老人得了重病之后,主动放弃治疗,默默的等待死亡,就是为了避免给家里增加负担。家里人对此也达成了默契,早早买好了棺材(农村土葬依然盛行)和孝布(白布)。

由于土地少、工业少,河南产生大量剩余劳动力也就毫不奇怪。这些人在农村经常是没事干。我小时候很喜欢在农村呆着,跟他们一起打牌、下棋。这几年一琢磨,这不就是“群体游手好闲”的鲜活例子么?
长期赋闲之后,会让人懒惰,懒的不想动,不愿意接受繁重的劳动。这个问题不大不小,还好中国是有勤劳传统的国家,从历史角度说,这种传统起源于河南。勤劳传统会很大程度上抵消了负面影响。
然而,这还只是“剩余劳动力”,而不是“自由劳动力”。很多农民家里一摊子事儿,离开家乡去工作并不容易。在孝道传统下,如果是兄弟几人的话,家里至少要留一两个。
即使出去很多人,地里的活儿仍然是少的可怜。因为自动化程度也在提升,大型播种机、收割机的租赁已经非常流行。

\subsection{劳动力的转移}
有些农民工多年不回家,

我知道一对农民工夫妇,男的58岁,女的和他差不多年龄,还要去新疆打工。

\subsection{劳动力的异地就业}

虽然异地就业已经是常态,但对于很多农民来说,并没有归属感。
几乎所有的农民工,都会提到一些“被欺侮”的故事。

劳动力的
在德国,新到一个城市落户,有“欢迎费”。咱中国人都觉得到了异地肯定被歧视,哪想到人家这么友好。

不但要允许他们户籍变动,甚至要鼓励他们异地落户。
容易回收他在家乡的耕地和宅基地。

为什么中国各地的城市对于外来人口都有不同程度的敌视呢?

中国的户籍制度是“分蛋糕”模式,就是把总量变成固定的,然后由

这就给城市既有住户一种感觉:农民工分享了城市的财富,农民工把城市弄的很脏,农民工让城市拥挤,农民工

\textbf{食品券}是一种专用货币代用券,一般发放给生活贫困户,专用于购买食品。在美帝,这是流行的做法,可叹在中国“不好用”。其原因也毫不意外的具有中国特色:
\begin{itemize}
\item 超市不在乎客户用食品券购买的是什么
\item 超市不在乎食品券的使用者是不是有权享用食品券的穷人
\item 用食品券折价兑换现金很少有人拒绝
\item 官员贪污食品券很难杜绝
\end{itemize}

我得说,食品券的发型目的是好的。很多穷人之所以贫穷,是因为有很多坏习气,比如喝酒、抽大麻(在美帝很常见,在中国穷人中还不盛行)、赌博等。因此,政府希望用食品券限制这些坏习气。可我又要说,这是不符合中国国情的。不管这个政策看起来有多美好,最好还是别在中国推广了。

那如何发放贫困补助才合适呢?



中国粮食政策是“按保护价敞开收购”。这导致农民种了过多的粮食,收购上来的粮食无处存储,只能烂掉\footnote{《王福重经济学五十讲》}。

但是,粮食的产量是很难预测的。虽然现代科技大大增强了农民抵御灾荒的能力,但对于大旱大涝基本上是束手无策。

对于中国这样的人口大国来说,完全依靠国际粮食采购也不太可行。

对于初来乍到的农民工

比如失业保险和

如果这些农民

\emph{改名额制为比例制}
比如教育上有录取名额限制,那么外来人口就会占据本地人的名额。
如果是比例制的话,本地人对外地人的敌视情绪就会小得多。
股份制改革

\emph{改户籍福利为股份红利}

在名额制的
农民进城

农民工在农村
18亿亩红线



\subsection{就地消化}
能否让农民工“就地消化”呢?如果他们在离家乡不太远的地方找到工作,不是更好吗?

\subsection{户籍制度}
中国的城乡二元户籍制度
工业
教育、医疗
在开封的一些县里,当地观察不到任何工业,
郑州本来是一座轻工业城市,
\subsection{耕地}
1. 分地
中国在1990年前后【具体时间待查询】有一次大规模的分地。当时分地主要是按人头划分。很显然的,如果某家人在分地时有老人活着,分地后老人死亡,这家的地就会比较多。如果家里没什么老人,分地后又生了几个娃儿,家里的地就很少。
分地后,将50年不再变更。所以,地少的家庭吃了亏也没办法

为了保证公平,
每家的土地都非常

在此以后,
\subsection{宅基地}
中国的耕地需要维持“18亿亩红线”,这是维持中国粮食自给自足的红线。
这些年来,大片的耕地,尤其是城郊耕地,变成了房屋、道路、工厂、绿化带。毫无疑问,耕地面积在迅速缩小。
把耕地变为宅基地被称作“变更土地用途”。20年前,把耕地转为宅基地是很容易的。后来温家宝上台后,收紧了土地政策,就没那么容易了。
我曾经拜访过一些农村的同学。其中有一家,院子大的可以踢足球!进到院子里,其实居住面积并不大,房屋只占了一点点,甚至还没我家大。那么大的院子做什么用呢?种了大蒜、芫荽和杜仲(一种中药),其实还是耕地。
农村人居住的非常散。按照我自己对开封、郑州市郊一些农村的观察,500~1000人的村庄可以占地1~2平方公里。如果进行城市化,使得农村人住的更集中一些,比如用4~10层楼代替农村的1~2层房屋,用公共绿地代替庞大的场院,估计能节约80%的面积。
宅基地转耕地不太容易,土壤以及严重硬化,甚至是硬邦邦的水泥。最好的做法或许是退房还林,退房还牧。

1. 具体政策
然而,想要农民放弃宅基地相当不易。
近十几年来,大批市郊农民因为卖地、土地开发而一夜暴富,足以让大多数贫穷的农村人眼红。于是,远郊的农民虽然根本没有卖地的机会,但是他们宁可把土地攥到手里,也不愿意出售或者退房还林。

我说过,我讨厌“行政一刀切”的做法。如果可以用经济调节,最好还是采用
如果简单的用经济方式
因为远郊的农民往往比较穷。市郊农民往往可以种植经济作物,或者在市区里找点活干。但是远郊农民收入来源非常少,粮食作物是不值钱的。河南的土地每年可以产一季麦子、一季玉米,即使超级大丰收,每亩毛收入也只有3000元左右【缺少论据】。种子、化肥、农药等占了一半。
这么可怜的人群,怎么忍心对他们采用
比如对四层楼以下的房屋,每坪宅基地每年收10块钱。对于城里人来说,这是个很低的数值。但对于贫穷的农民来说,500坪宅基地就是5000块,相当大的一笔钱。

这里我再阐述一下我的制度设计理念:\textit{实用第一}。比如说,把幼儿园教师的水平提高到

但如果做不到呢?但这种制度很可能会带来一个后果:。所以,这是不具备可执行性的。

把一群私人办幼儿园的妇女、大嫂、大妈集中起来强制培训三个月就行了,总好过没有。幼儿园开始运转后,每周强制来参加90分钟的夜校。

\subsection{技术培训}
德国普通工人的技术水平世界闻名。是的,德国技术工人的培训标准不但严格,而且得到了良好的执行,这是“德国质量”的基础保障。于是中国人便有一个美好的愿景:能不能把中国农民工的技术水平提高到德国工人的水平?

理想是美好的,现实是残酷的。

在三线城市和城镇里,哪有什么培训?想要找工作的农民工往往跟着同乡的包工头干,不需要任何资质和培训,给指派一个师傅,然后戴个安全帽就上工了,所谓的培训,不过是师傅过来指导几句罢了。

就我看到的一些情况,

尤其是在
一、教育
( 1 ) 寄宿制学校
我承认寄宿制学校有很多问题。我在大学之前就生活在近乎于军事化管理的寄宿制学校中,每天都被迫学习。我痛恨那种环境。
但是,有很多
因此,这是个两害相权取其轻的过程。

\subsection{教育平等}

教育是
今天的问题儿童,就是十五年后的潜在罪犯。
二、男女差异


鼓励告密不是问题所在,关键是怎样判定告密的真假。如果是由官吏判定,
如果告密成为官吏的发财手段,那
