\chapter{智力篇}
有个谬误在网上流传很久:大脑使用率不到10\%,如果我们努力开发的话,就会有无限可能。然而,这只是早期对大脑研究不够透彻所造成的错误理解罢了【缺少引用】,现在的研究发现,大脑早已火力全开来应付残酷而复杂的自然界。

目前的智商(即IQ)测试主要测量逻辑推理能力。以前,它的制定标准是:$
IQ = 智龄 / 年龄 x 100 $, 所以平均智商一定是100. 现在,它是按照正态分布来制定 ,平均智商仍然是100。有个笑话是:总统对于一半国民智力低于平均数感到惊讶。

《阿甘正传》里的阿甘智力只有70~75,按照法律只能上特殊学校。靠着亲娘跟老师“娱乐”一晚上,阿甘才获得了上普通学校的机会。

然而,人的智力究竟多有限呢?

这是一个很难回答的问题,正面回答尤其困难,所以我准备“绕着圈子”回答:

\section{围棋}
我学围棋是在2000年暑假。初学围棋的三年,也就是2000~2002年,是中国围棋史上最最惨淡的三年,连一个世界冠军都没拿到。



截止到2002年,中国棋手的成绩只能用“惨淡”来形容,仅仅拿到了59个世界个人赛冠军中的4个。
团体赛最主要的就是三国围棋擂台赛,韩国包揽了1991~2004的冠军,中国的成绩简直不能看。
其中曹李师徒
在现代围棋史上,1988年创办的富士通杯和应氏杯是最早的围棋世界大赛。
第一个冠军在2007年才拿到

当时的娱乐活动少的可怜
如果我以此数据说:韩国人比中国人更聪明,恐怕有读者要激愤的跳起来。

1.  韩国人更勤奋
不管天赋多么卓异,没有足够的努力还是不行。
2.  韩国人更早开始少年集训

人在小时候会产生
3.  韩国人更有钱
看起来很无稽,却是个血淋淋的事实。各种少儿兴趣班都很烧钱。没有钱,也没有学习班。
我小时候想在附近找个兴趣班都找不到,而现在的兴趣班好比雨后狗尿苔。

围棋的复杂度不够高
虽然

\section{多线程能力}
大家有没有玩过需要看小地图的游戏?比如赛车类、DotA、LoL、星际争霸、魔兽争霸等。如果玩过的话,你能一直关注小地图吗?

在游戏刚开始时,一般不太激烈,时不时看一眼小地图是多数人都能做到的。但是在游戏中盘,往往有十几个到几十个目标需要关注,玩家手忙脚乱、顾此失彼是常态。如果这时候你还能做到时不时看一眼小地图的话,恭喜你,你有成为高手的潜质。

人顾及多个目标的能力,我称之为人类的“多线程”能力。“多线程”本来是计算机术语,指的是计算机同时处理多个线程(可以简化理解为任务)。此处借用这个概念。某些游戏高手的多线程能力是非常惊人的,他们可以同时照顾到三条同时开打的战线——你自己玩过就知道有多难了。

比赛只需要人群中最顶尖的一小撮,对于那些需要全民参与的工作,就必须照顾到脑子没那么灵光的人群。

好的制度,是必须客观符合人的基本能力的。我再以军队编制的演化为例:
我测过三次智商,得分分别是146、158、127。这三次的分数差异挺大的,因为考察的侧重点不同,折射出我的各项智力比我的高中成绩还偏科。
大学上了计算机系,号称是“游戏系”,每天玩的不亦乐乎。我的目标是成为绝顶高手,但不管怎么练习,只能达到中上水平。到底是什么阻碍了我的技术提高呢?
我测过手指反应速度,正好是平均数,随便玩玩还可以,当高手是别指望了。还有在一大堆杂乱的图形中寻找目标的能力,也挺一般的。打起团战来光影效果乱飞,我经常找不到自己想找的兵,找到之后发现已经死翘翘了。还有我的多线程能力也挺差的
如果打游戏总赢的话,没准我的瘾头会大得多。老输的灰头土脸,也就慢慢放弃了很多游戏。
我不得不沮丧的承认:我的运动神经反应速度很慢,我对图形的识别速度很低。
按此图【图以后补充】,50\% 的人智商在90~110 之间,这就意味着有25\% 的人IQ 在90 以下。
我的运动神经就不太行,反应速度慢,

\subsection{军事学}
《三国演义》里面,姜维向司马望炫耀阵法:“这是丞相(诸葛亮)传下来的阵法,你懂?”司马望表示:“懂啊,不就是九九八十一种变化。”姜维哈哈大笑:“其实是365种,你学艺不精啊。”听起来很NB对不对?

袁世凯小站(“小站”是地名,在天津)练兵的时候,士兵连左右都分不清。诸葛亮治下的蜀国,三国时期的教育水平能高到哪里去?还365种阵法……将官自己都记不住。

在冷兵器时代,一般用到的振型也就有长途行军时的纵队,防御时的圆阵,攻击时的楔形阵,便于远程武器发射的线形阵。


事实上,如果诸葛亮的阵法真有那么厉害,蜀国早就一统天下了。

在西方,也不过

\section{智力门槛}

四、正视问题
在我的中学时代,几乎所有人都被迫努力学习,学校里把一切不利于学习的活动都取消了。校园小卖部老板本来有几副象棋,大家都喜欢课间去那里杀上一盘,两人对战,一群人观战。像我这种不爱学习、棋瘾极大者自然也是那里的常客。因为我缺乏运动天赋,下象棋是我唯一的展示才艺的方式。可中学校园还是太小了,没多久,校领导就发现并摧毁了这个让学生“分心”的活动平台。
这就是我的中学:封校,每周放假七小时,取消一切娱乐活动,收缴课外书……可以说,让人“无聊到只能学习”。
写这些废话,是想告诉大家:在我的中学时代,学生们的努力程度是比较接近的,多数人都已经全力以赴。
初中班主任提出过一个奇葩思想(他有很多奇葩思想):有些同学成绩差是因为“学习方法不对”。如果找到正确的学习方法,并且足够努力的话,所有人都可以取得优秀的成绩。
很多人都相信了这套“学习方法论”,努力寻找最合适的学习方法。有个很倔强的
这套“学习方法论”最大的作用是麻痹了一群人,让他们不至于为自己的先天智力而自卑,其它的屁用也没有。

我中学时代就经常在班里负责数理化答疑。一方面是我成绩还行;另一方面是我脾气好、会忽悠;还有就是我经常看小说,大家觉得找我帮忙不算“浪费别人时间”。

我中学时代也思考的问题是“怎么让成绩不好的同学也学习好”。

我承认,如果进行智力测验,或者在社会上形成“智力决定论”的风潮,对少年儿童的发展是不利的。孩子之间会互相攀比,甚至家长之间也会攀比,智力较差的孩子很可能会产生严重自卑。
但这不代表我们就可以无视这个问题,假装所有的孩子智力都差不多。

古代军队的编制比较乱,各国差异很大。一般是每个大单位辖5~100 个小单位,各级之间的“进制”还往往不相等。现代军队的进制是比较整齐的,一般都是三三编制,即每个大单位包含三个小单位,一个团有三个营,一个营有三个连,等等。这就意味着,每个军官要应对一个上级、两个平级、三个下级,一共六个目标,不需要太高的IQ。


难道军官的晋升跟智力无关吗?难道排长、连长们不该明显比小兵更聪明一些吗?难道他们的多线程能力不该大大优于平均数吗?
由于一些特殊的文化传统,中国人偏爱“儒将”,我希望大家不要受此偏好影响 。实际上,军官的晋升要看很多特征:强壮、勇气、服从、运动神经
所以,军官即使比士兵聪明一些,也只是有限的聪明。
这些特征,与智力几乎是无关的,用更学术的话讲,叫“与智力的概率相关性极低”。我想大家在学校里都遇到过不少在球场上
军官里,很可能就有不少IQ<90者。因为下层士官的提升,往往要看身体素质、勇敢
脑子笨不代表身体素质差,

\section{信息量}
如果信息量



从小学到高中,几乎所有的班主任都试图向我和同学们灌输一种神奇的理论:学习不好的主要原因是不够努力和学习方法错误。中国是个适合错误理论泛滥的国家。这个似是而非的理论流传甚广,因为所有的老师都没有用科学方法去查证和研究的勇气。


现在基本上可以

学音乐的人经常说:手指是有记忆的。

\section{水平上限}
这时候,便已经遭遇了\textit{水平上限}。


电影《听风者》中,男主角阿兵本来是个瞎子,所以耳朵特别灵敏,可以听到常人听不到的音频和极其微弱的声音,于是被招募为监听电台的反间谍人员。后来阿兵通过手术恢复了视力,但是他的听力随之逐渐下降,渐渐的无法胜任以前的工作了,造成任务失败,梦中情人牺牲。阿兵痛苦的戳瞎眼睛,终于恢复了听力,为梦中情人报了仇。

这其中蕴含的科学原理是:优秀的听觉不仅需要优秀的耳膜和听小骨,还需要优秀的听觉神经。瞎子用不着视觉神经了,所以空闲的视觉神经就可以被转为其它用途,大大增强他们的听力。这在医学上称为“代偿”【缺少引用】。

现代的游戏都很复杂。当我刚开始玩一个游戏的时候,常常会觉得游戏中的元素非常复杂,根本记不住。看到一个图标,不能理解是什么意思。但是玩几小时之后就没有什么问题了,看到一个形象马上就可以识别出来。这说明我的大脑中已经不知不觉中生成了用于形象识别的神经元。

很可惜的是,脑中不但在不断产生神经元,还在不断销毁神经元。所以人会遗忘,而且

人的水平提高到一定程度后,往往就无法继续上升了。这说明此人在练习中

所以,让我来总结下我的理论吧:
1. 知识由神经元存储,神经元总量是有限的,故此每个人的脑部存储的知识都是有限的
2. 嗅觉、视觉、听觉、触觉、味觉、声带、运动都要靠神经元支持。
3.
对多个目标的关注能力并非与智力直接相关。

有一种说法很流行,如果在青少年时期。也许会有不少人现身说法,说自己既可以说好汉语,也可以说好一门外语。这都是扯淡的

我一直都觉得中学数理化太简单,要大大增加难度才行;而且由于随便看看课本就明白了,数理化老师可以取消;总是考课本上的知识多没意思,考题越天马行空越好。

如果说有人能良好

因为这种现身说

事实上,由于中英文的巨大差异,一大半人都无法良好的同时使用两者。

没办法,人的智力就是这样有限。


或者网上有很多需要人一心多用的手机游戏。
人既有暴力倾向,也有共情倾向。
以上各属性在社会中表现为正态分布。
这并不意味着
“人的有限智力”定理将会用于解决:
为什么无神论难以接受?
为什么星座学、算命看相盛行不衰?
为什么我现在反对增加中国古典文学教育?而我十年前是持支持态度的。(《教育篇》)
为什么我坚决要求废除考研中的政治?(《高考篇》)
为什么我痛恨网上“钓鱼”?
为什么我反对一人一票式的民主?而我是坚定支持民主制的。(《改革篇》)
……以及一些我还没总结好的东西
