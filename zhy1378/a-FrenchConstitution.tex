(1958年9月28日通过1958年10月4日公布)
 
                                 序言
 
    法国人民庄严宣告:忠于1789年人权宣言所肯定的、为1946年宪法序言所确认并加以补充的各项人权和关于国家主权的原则。
    根据这些原则和各族人民自决权的原则,共和国为愿意参加的海外各领地提供建立在自由、平等和博爱的共同理想基础上的、并使各领地得到民主发展的新的组织。
    第1条共和国以及海外领地的人民自由决定通过本宪法,并组成共同体。
    共同体建立于组成共同体的各族人民的平等和团结的基础上。
 
                             第一章    主权
 
    第2条法兰西为不可分割、非宗教的、民主的并为社会服务的共和国。全体公民,不论血统、种族和宗教信仰的不同,在法律面前一律平等。法兰西共和国尊重一切信仰。
    国旗为蓝、白、红三色旗。
    国歌为马赛曲。
    共和国的口号是:“自由、平等、博爱”。
    共和国的原则是:民有、民治和民享的政府。
    第3条国家主权属于人民,人民通过自己的代表或者通过公民复决来行使国家主权。
    任何一部分人或任何个人都不得擅自行使国家主权。
    根据宪法所规定的条件,采取直接选举或间接选举。选举一律采取普遍、平等和秘密投票的方式。根据法律所规定的条件,凡享有公民权利和政治权利的法国成年男女国民都有选举权。
    第4条各党派和政治团体协助选举的进行。
    各党派和团体可自由地组织和进行活动,但必须遵守国家主权原则和民主原则。
 
                             第二章    共和国总统
 
    第5条共和国总统监督对宪法的遵守。总统进行仲裁以保证国家权力的正常行使和国家的持续性。
    共和国总统负责保证民族独立、领土完整,以及条约和共同体协定的遵守。
    第6条共和国总统任期7年,由选举团选举产生,选举团包括议会议员、省议会议员、海外领地议会议员,以及由市镇议会所选出的代表。
    市镇议会代表是:
    居民在1000人以下的市镇的市长。
    居民在1000人到2000人的市镇的市长和第一副市长。
    居民在2001人到2500人的市镇的市长、第一个副市长和一名按次序决定的市议会议员。
    居民在2501人到3000人的市镇的市长和前两名副市长。
    居民在3001人到6000人的市镇的市长、前两名副市长和三名按次序决定的市议会议员。
    居民在6001人到9000人的市镇的市长,前两名副市长和六名按次序决定的市议会议员。
    居民在9000人以上的市镇的全体市议会议员。
    此外,居民在30000人以上的市镇,除全体市议会议员外,每多1000人增加代表一名,代表由市议会指定。
    共和国的海外各领地也参加选举团,代表由各行政单位的议会依照组织法规定的条件选举产生。共同体成员国参加共和国总统选举团的事项,由共和国和共同体成员国以协议确定之。
    本条的执行办法由一个组织法予以确定。
    第7条共和国总统的选举,在第一轮投票中采用绝对多数制。如果第一轮投票没有结果,共和国总统在第二轮投票中根据相对多数制选出。
    选举由政府召集举行。
    新总统的选举应在总统任期届满前至少20天、至多50天之内举行。
    共和国总统因故缺位,或由政府提出经宪法委员会以绝对多数票证实总统不能行使职务时,除下列第11条和第12条所规定的各项职权以外,共和国总统的职权,暂时由参议院议长行使。在总统缺位或在宪法委员会明确宣布总统不能行使职务的情况下,除非宪法委员会认定有不可抗拒的情况存在,新总统的选举应在缺位后或明确宣布总统不能行使职务的宣告发表后至少20天、至多50天之内举行。
    第8条共和国总统任命总理。总统于总理提出辞职后解除总理职务。总统根据总理建议任免政府其他成员。
    第9条共和国总统主持内阁会议。
    第10条共和国总统于法律最后通过并送交政府后15天内予以公布。
    在这个期限届满以前,总统得要求议会就该项法律或该法律的某些条文重新进行审议。议会不得拒绝重新审议。
    第11条共和国总统根据政府公报所刊载的、由政府在议会开会期间所提出的或由议会两院联合提出的建议,得将关于授权核准共同体协定或授权批准虽然不违反宪法、但影响政府机构职权行使的条约的一切涉及公共权力组织法律草案,提交公民复决。倘公民复决结果通过了该项法律草案,共和国总统应在第10条第1款规定的期限内予以公布。
    第12条共和国总统在与总理及议会两院议长磋商后,得宣布解散国民议会。国民议会解散后,至早在20天内、至迟在40天以内举行大选。
    国民议会在选出后第二个星期四自行举行会议。如果这次会议是在规定的常会期间以外举行,会期应为15天。
    大选后一年内不得再行解散国民议会。
    第13条共和国总统签署内阁会议所决定的法令和命令。
    总统任命国家行政人员和军事人员。
    行政法院委员、授勋委员会主任、全权大使和公使、审计院的审计官、省长、政府在海外领地的代表、将级军官、大学校长、中央各部司长都应通过内阁会议任命。
    组织法规定通过内阁会议任命其他人员,并规定在何种条件下共和国总统的任命权可以由总统指派他人代行。
    第14条共和国总统委派驻外全权大使及公使并接受外国全权大使及公使。
    第15条共和国总统为军队的最高统帅。总统主持最高国防会议和国防委员会。
    第16条当共和国体制、民族独立、领土完整或国际义务的执行受到严重和直接威胁,并当宪法所规定的国家权力的正常行使受到阻碍时,共和国总统在同总理、议会两院议长和宪法委员会主席正式磋商后,应根据形势采取必要的措施。
    总统以文告把所采取的措施通告全国。
    这些措施的目的应该是为了在最短时间内保证宪法规定的权力机关拥有完成它们任务的手段,在这问题上,应咨询宪法委员会的意见。
    议会自行举行会议。
    国民议会在行使特别权力期间不得被总统解散。
    第17条共和国总统有赦免权。
    第18条共和国总统用咨文同议会两院联系,咨文在两院宣读,但以任何方式进行讨论,皆不得追究。
    如在休会期间,议会可为此举行特别会议。
    第19条共和国总统的命令由总理副署,必要时由有关部长副署,但第8条第1款、第11条、第12条、第16条、第18条、第54条、第56条和第61条所规定的情况,不在此限。
                           
                               第三章    政府
 
    第20条政府决定并执行国家的政策。
    政府掌管行政机构和武装力量。
    在第49条和第50条所规定的情况下,政府依据该两条所规定的程序对议会负责。
    第21条总理领导政府的活动。总理对国防负责。总理确保法律的执行。除第13条的规定外,总理行使制定条例的权力,并任命文武官员。
    总理可以委托部长代行他的某些权力。
    在必要时,总理代替共和国总统主持第15条所规定的最高国防会议和国防委员会的会议。
    在特殊情况下,总理可以根据明示的授权就某一项特定议事日程代替共和国总统任内阁会议主席。
    第22条总理的命令必要时由负责执行该命令的部长副署。
    第23条政府成员不得同时担任任何议会议员职务或保有任何全国性职业代表的职务,亦    不得参加任何职业性活动。
    对担任上项任务、职务或公职的人员,其替换条件,由组织法予以确定。
    议会议员之替换,应按照第25条的规定办理。
 
                               第四章    议会
 
    第24条议会由国民议会和参议院组成。
    国民议会的议员由直接选举产生。
    参议院由间接选举产生。它确保共和国各领地单位有代表权。居住在国外的法国侨民在参议院应有其代表。
    第25条议会两院的任期、议员人数、议员俸给、议员候选资格,以及无资格候选及不得兼任职务的制度,都由组织法予以确定。当议员缺额时,得进行补缺选举,补选的国民议会议员或参议院议员,行使职权至其所属议院全部或部分改选时为止。补选议员的条件,由组织法予以确定。
    第26条不得根据议员在行使职务时所发表的意见或所投的票而对议员起诉、搜查、逮捕、拘禁或审判。议员在会议期间,除现行犯外,非经其所属议院的同意,不得因其犯有刑事罪或轻罪而被起诉或逮捕。
    议员在议会闭会期间,非经其所属议院办公厅的同意,不得逮捕。但现行犯、经核准起诉和经确定判决者除外。如经议员所属议院的请求,对议员的拘禁或起诉,应即停止执行。
    第27条选民对议员的任何强制委托均属无效。
    议员的投票权属于其本人。
    组织法得例外授权委托投票。在这种情况下,任何议员不得接受一个人以上的委托。
    第28条议会每年自行召开两次常会。
    第一次会议从10月第一个星期二开始,至12月第三个星期五结束。
    第二次会议从4月最后一个星期二开始,会期不得超过3个月。
    第29条议会应总理或国民议会大多数议员的请求,得召开特别会议,并按照一个事先决定的议事日程进行讨论。应国民议会议员的请求召开的特别会议,规定的议事日程一经讨论完毕或最迟在会议召开12天后,闭会命令应即生效。在闭会命令发布后1个月内,只有总理才能要求重新召开会议。
    第30条除议会自行召开会议外,特别会议应由共和国总统以命令来宣布召开和闭会。
    第31条政府成员得出席议会会议,当他们要求发言时,议会应倾听他们发表的意见。
    政府成员得指派政府专员随同出席会议。
    第32条国民议会议长的任期与该届议会任期相同。参议院议长在每逢参议员部分重选后选举产生。
    第33条议会两院的会议公开举行。会议记录全文由“政府公报”发表。
    经总理或1/10议员的要求,议会每院得举行秘密会议。
 
                    第五章    议会和政府的关系
 
    第34条一切法律,皆由议会通过。
    第35条议会有批准宣战权。
    第36条内阁会议有权宣布戒严。
    如果戒严延长到12天以上,应经议会批准。
    第37条不属法律范围以内的事项具有条例的性质。
    以立法形式出现的有关这些事项的文件,经征询行政法院的意见后,得以命令予以改变。在本宪法生效后所制定的此类文件,除宪法委员会宣布其依上款规定具有条例性质者外,不得以命令予以修改。
    第38条政府为实施其政纲起见,得要求议会授权它在一定期限内以法令对于通常属于法律范围的事项采取措施。
    上述法令经征询行政法院意见后,由内阁会议予以制定。上述法令自公布之日起发生效力,但如追认该项法令的法律草案未能在授权政府制定该项法令的法律以前在议会提出,则该项法令即行失效。
    在本条第1款所规定的期限届满后,上述法令中有关立法范围的事项仅能由法律加以改变。
    第39条立法的创议权同时属于总理和议会议员。
    在征询行政法院意见后,法律草案在内阁会议上加以审议并提交两院之一的办公厅。财政法律首先提交国民议会。
    第40条议会议员所提出的提案和修正案,如其后果将减少国家收入或将加重国家负担,均不得成立。
    第41条在立法程序中,如某一提案或修正案不属于法律范围或与根据第38条规定的授权相反,政府可不予接受。
    法律规定下列事项:
    有关公民权和对行使公民自由所给予的基本保障;由于国防需要而对公民本身及其财产所作的必要限制。
    有关个人的国籍、身份和法律能力、婚姻制度、继承权及赠予。
    有关确定轻重罪和与之相适应的刑罚、刑事诉讼程序、大赦,以及设立新的司法裁判制度和法官的地位。
    有关征收各种课税的基础、税率和方式;有关货币发行制度。
    法律同时确定下列事项:
    有关议会和地方议会的选举制度;
    有关设立各类公共机关;
    有关给予国家文武官员的基本保障;
    有关企业国有化和企业所有制从公有到私有的转变。
    法律确立以下根本原则;
    有关国防的一般组织原则;
    有关地方单位的自治原则,它们的职权和资源的原则;
    有关教育原则;
    有关所有制的原则,物权与一般的和商业的债务制度的原则;
    有关劳动权、工会和社会保障的原则。
    在组织法规定的条件内,除去保留者外,财政法确定国家的收入与支出。
    有关国家计划的法律确定国家的经济和社会活动的目标。
    本条的各条款得由组织法加以详细说明和补充。
    如果政府和有关议院的议长对于提案或修正案能否成立的问题意见分歧,宪法委员会得应两者之一的要求,在8日内予以裁决。
    第42条首先接到法律草案的议院,根据政府所提出的条文对法律草案进行讨论。
    一个议院接到另一个议院所通过的草案后,应就送来的条文进行审议。
    第43条在政府或受理的议院的要求下,法律草案和提案应提交特设委员会进行研究。
    政府或议院未提出上述要求时,法律草案和提案应送交常设委员会,这样的委员会每一议院不得超过6个。
    第44条议会议员和政府有权提出修正案。
    在辩论开始后,政府可拒绝审议任何事先未经委员会研究的修正案。
    经政府要求,受理的议院应只就政府所提出的或同意的修正案,一次表决审议中的法案全部或一部。
    第45条任何法律草案或法律提案,均在议会两院相继进行审议,以求通过相同文本。
    如果议会两院意见分歧,当某项法律草案或法律提案在每个议院两读后未获通过时,或政府认为有紧急需要时,经两院一读后,总理有权召集一个双方人数相等的混合委员会负责对讨论中的条款提出一个文本。
    混合委员会所拟定的文本得由政府提交两院通过。除政府同意的修正案以外,其他都不予受理。如果混合委员会不能通过一个共同的文本,或者如果该文本未在前款所规定的条件下通过,政府得在国民议会和参议院重读后,要求国民议会最后表决。在这种情况下,国民议会得重新就混合委员会拟定的文本进行表决,或者就国民议会最后通过的文本进行表决,或就经参议院提出某一项或数项修正案加以修改的文本进行表决。
    第46条宪法确认具有组织法性质的法律按照下列程序进行表决和修正。
    草案或提案在提出后15日的期限届满时,始能由首先受理的议院进行审议和表决。
    第45条所规定的程序可以适用。但是,如果议会两院未能取得一致意见时,国民议会对于该项文本必须有议员人数绝对多数的同意,始得在最后一读予以通过。
    有关参议院的组织法,应依据同一方法由议会两院表决通过。
    组织法在宪法委员会宣布其符合宪法后始能颁布。
    第47条议会依照组织法所规定的条件,对财政法案进行表决。
    如果国民议会在草案提出后40天未能一读通过,政府即提请参议院审议,参议院应在15天内作出决定,然后依照第45条所规定的条件进行。
    如果议会未在70天内作出决定,草案的规定得以法令予以实施。如果规定某一会计年度收支的财政法案未在适当时期提出以便在该会计年度开始前公布,政府得向议会提出紧急要求,提请议会授权政府征收租税,并以法令拨出业经表决过的各项经费。
    如议会系在休会期间,本条所规定的期限应予顺延。审计院协助议会和政府监督并执行财政法。
    第48条在议会两院的议程中,应按照政府所规定的次序,优先讨论政府所提出的法律草案和政府所同意的法律提案。
    每周有一次会议专供议会议员提出质询和政府进行答辩。
    第49条总理就内阁会议讨论通过的政府施政纲领或者总政策,得对国民会议提出由政府承担责任的说明。
    国民议会根据所通过的不信任案,得追究政府责任。但该不信任案必须至少有国民议会1/10议员的签名才能提出。不信任案必须在提出48小时后方得进行表决。不信任案的表决只计算赞成票,不信任案在获得国民议会全体议员多数赞成时才能通过。如果不信任案被否决,签名的议员不得在同一次会议上提出新的不信任案。但下款规定的情况不在此限。
    总理就内阁会议通过的议案表决,得对国民议会提出由政府承担责任的说明。如果根据前项条款所规定的条件在24小时内没有通过任何不信任案,该项提案即认为已经通过。总理具有要求参议院同意其总政策说明的职能。
    第50条当国民议会通过不信任案或当它不同意政府的施政纲领或总政策说明时,总理必须向共和国总统提出政府辞职。
    第51条议会常会或特别会议的闭会得自行延长,以便在必要时使第49条条款得到实施。
                    
                       第六章    国际条约和协定
 
    第52条共和国总统缔结并批准条约。
    任何旨在缔结无需批准的国际协定的谈判应通知共和国总统。
    第53条和约、贸易条约、关于国际组织的条约或协定、涉及国家财政的条约或协定、修改属于立法性质的条款的条约或协定、关于个人身份的条约或协定,以及有关领土割让、交换或归并的条约或协定,非根据法律不得批准或通过。
    上述条约或协定在批准或通过后才能生效。
    领土割让、交换或归并,未经有关居民同意均属无效。
    第54条如宪法委员会,经共和国总统、内阁总理或两院中任何一院的议长将一项国际协定提交审议后,宣布该项国际协定含有违反宪法的条款时,必须修改宪法之后,才可以授权批准或通过该项协定。
    第55条依法批准或通过的条约或协定一经公布,具有高于法律的效力,但对于每一条约或协定,均以缔约对方的执行与否作为保留条件。
                      
                          第七章    宪法委员会
 
    第56条宪法委员会的成员为9人,任期9年,不得连任。宪法委员会每三年改选1/3。3人由共和国总统任命,3人由国民议会议长任命,3人由参议院议长任命。
    除上述规定的9个成员外,历届前任共和国总统为宪法委员会终身当然成员。
    宪法委员会主席由共和国总统任命。在裁决时,如双方票数相等,主席有最后决定权。
    第57条凡担任宪法委员会成员职务者,不得兼任部长或议员。不得兼任的其他职务,由组织法规定之。
    第58条宪法委员会监督共和国总统选举的合法性。
    宪法委员会审查争议事项并公布投票结果。
    第59条在发生争议的情况下,宪法委员会就议员和参议员选举的合法性作出裁决。
第60条宪法委员会监督公民投票的合法性,并公布其结果。
    第61条组织法在其颁布以前,议会两院的内部规则在执行以前,均应提交宪法委员会审查,以裁决其是否符合宪法。为了同样目的,各项法律在颁布以前应由共和国总统、内阁总理或两院中任何一院的议长提交宪法委员会审查。
    在前两款所规定的情况下,宪法委员会应在1个月的期限内作出裁决。但在紧急情况下,经政府要求,期限减为8天。
    在此情况下,宪法委员会的受理应中止公布的期限。
    第62条被宣布为违反宪法的条款不得公布,也不得执行。对宪法委员会的裁决不得上告。宪法委员会的裁决对于政府各部、一切行政机关和司法机关具有强制力。
    第63条宪法委员会的组织和行使职权的规则,向该委员会提交审议事项的程序,特别是关于提交争议事项的期限,由组织法予以规定。
                           
                            第八章    司法
 
    第64条共和国总统保证司法独立。
    最高司法会议协助总统。
    法官的地位由组织法决定之。
    法官不受罢免。
    第65条总统任最高司法会议主席,司法部长任当然副主席。在总统不在时,司法部长得代替总统担任主席。最高司法会议还包括依照组织法所规定的由总统任命的9名委员。
    最高司法会议对高等法院法官的任命及上诉法院首席法官的任命提出建议。依照组织法规定,最高司法会议对司法部长有关任命其他法官的建议提出意见。依照组织法规定,最高司法会议对特赦提出意见。
    最高司法会议又是法官纪律委员会,在这种场合下,最高司法会议由高等法院首席法官主持。
    第66条任何人不得被无故拘留。
    作为个人自由保护者的司法机关,依照法律所规定的条件保证尊重这个原则。
 
                          第九章    特别高等法院
 
    第67条设立特别高等法院。
    特别高等法院是由国民议会和参议院在两院每次全部或部分改选后,在各自的议员中选出人数相等的成员组成。特别高等法院从其成员中选出1人担任主席。
    特别高等法院的组成、行使职权的规则,以及向它提交审理的程序,由组织法规定之。
    第68条除叛国罪外,共和国总统对其执行职务的行为不负责任。议会两院只能以公开投票方式并由议会组成人员的绝对多数作出相同的表决时,才能对共和国总统提出控告。共和国总统由特别高等法院审判。
    政府成员在执行职务中犯罪,应按犯罪当时确定的重罪或轻罪负刑事责任。上述程序同样适用于犯有危害国家安全罪的政府成员及其同谋犯。按本款规定,特别高等法院应按照犯罪时所施行的刑法,确定其重罪或轻罪并量刑。
                      
                         第十章    经济与社会委员会
 
    第69条根据政府的请求,经济与社会委员会对由政府提交的法律草案、法令和命令以及法律提案提出意见。
    经济与社会委员会可指派1名委员,向议会申述该委员会对提交给它审议的法律草案或法律提案的意见。
    第70条经济与社会委员会也可以接受政府对于有关共和国及共同体的经济与社会性质的任何问题的咨询。
    任何计划或任何有关经济或社会性质规划的法律草案,均提请经济与社会委员会提出意见。
    第71条经济与社会委员会的组织与行使职权的规则,由组织法规定之。
 
                         第十一章    地方公共团体
 
    第72条共和国的地方为市镇、省和海外领地。其他地方公共团体由法律建立。这些地方公共团体按照法律规定的条件,由选举产生的地方议会自由地进行管理。
    政府在各地方公共团体的代表负责维护国家的利益、监督行政并使法律得到遵守。
    第73条海外各省可以根据它们的特殊情况对立法制度和行政组织采取相应措施。
    第74条共和国海外领地,是在共和国的总利益中,照顾领地固有利益的特殊组织。在征询有关领地议会的意见后,这个组织由法律予以确定或改变。
    第75条凡没有本宪法第34条所规定的普通法律上的公民身份的共和国公民,只要他们本人不放弃,仍旧保留他们的个人身份。
    第76条各海外领地可以在共和国内保持它们的地位。
    如果各海外领地由自己的领地议会在第91条第1款所规定的期限内作出决议,表示这样的愿望,它们可以成为共和国的海外省,或者联合地或单独地成为共同体的成员国。
                        
                           第十二章    共同体
 
    第77条在本宪法规定建立的共同体中,各成员国享有自治权,它们自行治理并民主地自由地管理它们自己的事务。在共同体中,只有一种公民身份。全体公民,不论他们的出身、种族或宗教信仰,在法律上一律平等。全体公民承担同样的义务。
    第78条共同体的管辖范围包括对外政策、国防、货币、共同的财政经济政策,以及有关战略物资的政策。除有特别协定外,共同体的管辖范围还包括对司法的监督、高等教育以及对外运输和公共运输与无线电通讯的一般组织。
    特别协定可以创制其他共同管辖权或规定共同体管辖权向某一成员国的任何转移。
    第79条共同体成员国一经按照第76条的规定作了选择,它们立即可享受第77条的规定。
    在执行本章各条款所必须的措施生效之前,共同管辖权问题由共和国决定。
    第80条共和国总统为共同体的总统并代表共同体。
    共同体的机构为执行委员会、参议院和仲裁法庭。
    第81条共同体成员国按照第6条规定的条件参加总统选举。
    共和国总统以共同体总统的资格在共同体各成员国派驻代表。
    第82条共同体的执行委员会由共同体的总统担任主席。
    执行委员会由共和国总理、共同体各成员国的政府首脑和负责共同体共同事务的各部部长组成。执行委员会组织共同体各成员国之间政府方面和行政管理方面的合作。
    执行委员会的组织与职权由组织法规定之。
    第83条共同体参议院由共和国议会和共同体其他成员国的立法议会从自己的成员中选出的代表组成。每个成员国代表的人数根据其人口及其在共同体内承担的责任多少而定。
    共同体参议院每年举行两次会议,由共同体总统宣布开会和闭会,每次会议不得超过1个月。
    在共和国议会,以及必要时在共同体其他成员国的立法议会,对有关共同体的经济财政政策的法律进行表决以前,经共同体总统请求,共同体参议院对这些政策进行讨论。
    共同体参议院对属于第35条和第53条的范围并使共同体承担义务的国际文件、条约或协定进行审查。
    共同体参议院可在共同体各成员国立法议会委托其代行的职权范围内作出执行性的决定。这些决定以和法律同样的形式,在各有关国家的领土上颁布。
    共同体参议院的组成及行使职权的规则,由组织法规定之。
    第84条共同体仲裁法庭裁决共同体成员国之间出现的争端。
    它的组成和职权由组织法规定之。
    第85条本章有关共同体组织的条款,无须依照第89条所规定的程序,而由共和国议会和共同体参议院所通过的措辞相同的法律予以修改。
    第86条共同体某一成员国的地位,可以根据共和国的请求,或有关成员国立法议会所作出,并由当地公民投票所确认的决议的请求,予以改变。公民投票由共同体有关机关进行组织及监督。这种改变地位的程序应由共和国议会和有关国家的立法议会所通过的协定加以规定。
    在同样条件下,共同体某一成员国可以成为独立国。这样它就不再属于共同体。
    第87条为实施本章而缔结的特别协定由共和国议会和有关立法议会批准。
 
                     第十三章    联合的协定
 
    第88条共和国或共同体可以同为了促进其文明而要求联合进来的国家缔结协定。
                    
                    第十四章    宪法的修改
 
    第89条宪法修改的倡议权属于共和国总统和议会议员,总统依据总理的建议行使倡议权。
    修改宪法的草案或提案应由两院以相同的文本表决通过。经公民投票通过后,宪法修改才最后确定。
    但当共和国总统决定将宪法修正案交付两院联席会议讨论时,该项修正案不交付公民投票,在这种情况下,只有当两院联席会议以有效票的3/5的多数赞成,该项修正案才能通过。国民议会的秘书处就是两院联席会议的秘书处。
    当宪法的修改有损于领土完整时,任何修改程序都不得着手进行或继续进行。
    共和体制不得成为修改的对象。
 
                      第十五章    过渡条款
 
    第90条议会例会中止进行。现任国民议会议员的任期至按照本宪法选出的国民议会举行会议之日起告终。
    在上述会议举行前,只有政府有权召集议会。
    法兰西联邦议会议员的任期和现任国民议会议员的任期同时告终。
    第91条本宪法规定的共和国的机构,应在本宪法颁布之日起4个月以内建立。建立共同体机构的期限可延长至6个月。
    现任共和国总统的职权只有按照本宪法第6条和第7条所规定的选举结果公布时才告终止。
    共同体各成员国将在宪法颁布之日,按照它们的法律规定的条件,参加这首次选举。
    国家现存的政权机构,在宪法生效后,将按照现行法律规定继续行使职权,直至新的政权机构建立为止。
    参议院在它的机构正式建立以前,由现任共和国参议院成员组成。明确规定参议院正式建立的组织法,应于1959年7月31日以前制定。
    宪法第58条和第59条授予宪法委员会的职权,在该委员会建立以前,由行政法院副院长担任主席并由最高法院首席法官和审计院首席审计官所组成的委员会代行。
    在为执行第十二章所必需的措施生效以前,共同体各成员国的人民在议会内继续有其代表。
    第92条为建立机构以及机构建立以前为行使公共权力所必需的立法措施,将由内阁会议在咨询行政法院后,以具有法律效力的法令规定。
    在第91条第1款所规定的期限内,政府有权以上述方式制定具有法律效力的法令确定本宪法所规定的议会选举制度。
    在同一时期内和同样条件下,政府同样可以采取它认为对国家生存、保护公民和保障自由所必需的各种措施。
                    
                    法国1958年宪法修正条款(一)
   (本法经1960年、1962年、1963年、1974年、1998年、1999年(1月25日和7月8日多次修改)
    第6条 共和国总统由普遍的直接的选举产生,任期7年。
    本条款的实施办法由组织法另定之。(本条为1962年10月28日公民投票决定修改,1962年11月6日正式公布)
    第7条 共和国总统以有效票的绝对多数当选。如果第一轮投票无人获得绝对多数,在第一轮投票后的第二个星期日举行第二轮投票。第二轮投票仅在第一轮得票最多的两个候选人中进行;但如第一轮得票多的候选人退出竞选,则以得票次多的两人作为第二轮投票的候选人。
    选举由政府召集举行。
    新总统的选举应在现任总统任期届满前20天到35天内举行。
    共和国总统不论因何原因缺位,或经由政府提出,宪法委员会以绝对多数确认总统因故不能行使职权时,总统职权暂由参议院议长代行,但下列第11条、第12条规定的各项职权除外。而如果参议院议长也因故不能行使职权,则由政府代行之。
    总统缺位或宪法委员会宣布总统确实不能行使职权时,除非宪法委员会确认有不可抗拒的情况存在,新总统的选举应在总统缺位或明确宣布总统确实不能行使职权之日起20天到35天内举行。
    在共和国总统缺位,或在宣布共和国总统确实不能行使职权之日起到选出继任者这一期间,本宪法第49条、第50条和第89条的规定不得实施。(本条为1962年10月28日公民投票决定修改,1962年11月6日正式公布)
    第28条 议会每年依法召开两次例会。
    第一次会议在10月2日开始,持续80天。第二次会议在4月2日开始,时间最长不得超过90天。如10月2日和4月2日恰逢假日,开会日期顺延至下一工作日。(1963年12月30日国民议会和参议院联席会议通过修改)
    第61条 为了同样目的,各项法律在颁布以前,可以由共和国总统、总理、两院中任何一院议长,或由60名国民议会议员或60名参议院议员提交宪法委员会。(1974年10月21日经议会两院通过修改,本条第2款增订,其余未变)
    第85条 本章的各款同样可以由共同体各国家之间缔结的协定予以修改。新条款根据各国宪法所规定的条件生效。(1960年5月11日国民议会通过、5月18日参议院通过、6月3日共同体参议院通过、6月4日颁布的宪法修正案增加第2款)
    第86条 一个共同体成员国也可以通过协议途径,变成独立国家,而并不因此脱离共同体。
    一个非共同体成员国的独立国家可以通过协议途径参加共同体,但仍然为独立国家。
    这些国家在共同体内的地位,由为此目的而签订的协定、特别是上述各款签订的协定,以及必要时由第85条第2款规定的协定所决定。(1960年5月11日国民议会通过、5月18日参议院通过、6月3日共同体参议院通过、6月4日颁布的宪法修正案增加第3款、第4款、第5款)
           
(法兰西共和国1958年宪法即法国现行宪法修正案(一)文稿,选自李龙著《宪法基本理论》一书,武汉大学出版社,1999年版)
 
                      法国1958年宪法修正条款(二)          
              1998年7月20日宪法修正案增加了新的第13章
 
    关于新喀里多尼亚的过渡性规定
    第76条根据法兰西共和国1998年5月27日正式发表的1998年5月5日努美亚协议的规定,新喀里多尼亚人将举行投票。
    凡符合1988年11月9日颁布的第88-1028号法令第2条规定的条件的人可以参加投票。
    组织这次投票的有关办法经与投诉调查部商议并由部长会议讨论后,由法令加以规定。
    第77条当该协议获得根据第76条规定举行的投票通过后,为确保新喀里多尼亚的发展遵循协议所确定的指导路线使得以贯彻实施,经与新喀里多尼亚协商会议讨论将制定组织法规定。
    --最终将移交给新喀里多尼亚的国家权力将于何时,以何种方式移交,由此产生的费用将如何承担;
    --新喀里多尼亚机构的组织和运行规则,尤其是协商会议通过的某种文件发布前提交给制宪会议的审查方面的情况;
    --关于公民权,选举制度,就业和为习惯法所高速的个人身份的规则;
    --新喀里多尼亚人对获得完整主权举行投票的条件和时间限制;
    对第76条规定的有关实施该协议的其他办法都将由法律规定。
 
            1999年1月25日宪法修正案修改了第88-2条和第88-4条规定
 
    第88-2:基于互惠和遵守1992年签署的欧洲联盟条约的条款的遵守,法国同意为建立欧洲经济和货币联盟让渡出必要的权力。
    基于上述保留和对1997年10月2日修改的建立欧洲共同体条约条款的遵守,法国同意将必须的关于人员的流动自由和相关区域的规则制度权让渡出来。
    第88-4:欧洲共同体或欧洲联盟的包括章程规定的文件草案或建议案送到欧洲联盟会议后,政府应将它提交国民议会或参议院。向国民议会或参议院提交的文本,也应包括欧洲联盟某个机构提出的文件草案或建议案。
    根据现行议会议事规则,即使议会没有在开会,所提交的关于上述草案,建议案或文件的议案也可以获得通过。
 
            1999年7月8日宪法修正案修改了第3条、第4条,增加了第53-2条
 
    第3条国家主权属于人民,人民通过其代表和以公民投票的方法行使国家权力。
    任何一部分人民或个人不得擅自行使国家主权。 
    依照宪法规定,选举权可以是直接或间接的行使,但它必须是普遍的、平等的和秘密的。
    凡享有公民权利和政治权利的所有法国成年男女国民依法律规定享有选举权。法律保护女子和男子参与竞选公职和职位的平等权利。
    第4条政党和团体应协助选举的进行,它们可以自由地组织并进行活动,它们必须遵守国家主权原则和民主原则。
    它们应协助上述宪法第3条所确立的原则的实现。
    第53-2根据1998年7月18日签署的条约的规定,共和国承认国际刑事法院的司法裁判权。 
 
(法兰西共和国1958年宪法(即法国现行宪法)宪法修正案(二)的三部分英文版本,由法兰西共和国驻华大使馆提供。由西北政法学院法一系朱继萍翻译、审校,西安翻译培训学院郑娜同学参加了部分内容的翻译。)