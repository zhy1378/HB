%\zBook{书名}{作者}{分类}{时间}{评论}

\newcounter{booknr}
\stepcounter{booknr}
\newcommand{\zBook}[4]{
\noindent[\thebooknr]\hspace{10pt}\emph{#3} \textbf{#1} \textit{#2} {\smaller#4}

\stepcounter{booknr}
}

\chapter{读书录}

我并不认为w

如同我前面提到的,我对爱情不太敏感(活该打光棍!)。所以,我对爱情小说的评分普遍偏低。

评分4~5
我的评分体系:
\begin{description}
\item[5]  强力推荐
\item[4] 好不错,值得看看
\item[3] 不怎么样,不值得看;或者是写的挺好,但是不合我口味
\item[2] 一塌糊涂,最好别看
\item[1] 精神垃圾
\end{description}

\zBook{我的评分}{书名,多本书之间用分号分隔}{作者}{我的评论}

\section{小说}
\zBook{三体}{刘慈欣}{5}
{最好的科幻小说,没有之一。刘慈欣的所有小说都值得一读。}

\zBook{百年孤独}{}{4}
{名著,不解释}

\zBook{哈利•波特}{}{5}
{风靡世界的魔幻小说}

\zBook{冰与火之歌}{[美]George Martin}{5}
{原名<The Song of Ice and Fire>,最爱的西方魔幻小说,第一本用英文读完的书(之前我连英文课本都没看完过)。}

\zBook{陆犯焉识}{严歌苓}{4}
{电影归来的原著}

\zBook{魔戒}{[英]J. R. R. Tolkien}{5}
{电影原著。构思宏大的魔幻小说,但是太面向儿童了,不黄不暴力}

\zBook{生命之歌; 蚁生}{王晋康}{5}
{王晋康的小说良莠不齐,但这两本确属精品。}

\zBook{异时空——间谍}{愚蠢猎人}{5}
{我与作者骂战无数,但不妨碍我推荐他的小说。}

\zBook{窃明}{灰熊猫}{5}
{全新视角看明末历史}

\zBook{科幻世界杂志}{}{5}
{最好的科幻杂志,大多数的期我都看过}

\zBook{1984; 动物庄园}{[英]George Orwell}{5}
{1984写于冷战前,预见了整个冷战时的TG 阵营。没读过George Orwell 的两本书,都不好意思管自己叫自由派}

\zBook{福尔摩斯}{[英]柯南道尔}{5}{推荐语是多余的}

\zBook{十日谈}{[意]薄伽丘}{4}{很有趣的中世纪小说集}

\zBook{金庸武侠全部}{金庸}{4}{}

\zBook{平凡的世界}{路遥}{4}{}

\zBook{三言二拍}{冯梦龙}{4}{明朝的白话文还比较好懂。}

\zBook{白鹿原}{陈忠实}{3}{不太喜欢,觉得里面的人物矫情。}

\zBook{聊斋志异}{蒲松龄}{5}{挺好看的,不难懂。}

\zBook{计算中的上帝}{}{4}{科幻小说}

\zBook{深渊}{}{5}{科幻小说}

\zBook{天火}{}{5}{科幻小说}

\zBook{基地三部曲}{[美]阿西莫夫}{5}{好看,而且引发思考}

\zBook{北欧神话ABC}{}{4}{}

\zBook{俄罗斯童话}{}{3}{}

\zBook{一千零一夜}{}{5}{}

\zBook{希腊神话}{}{5}{}

\section{社会 哲学 人类学}
\zBook{我的西域,你的东土}{王力雄}{5}{系统性研究新疆的民族矛盾}

\zBook{潜规则; 血筹史观}{吴思}{5}{}

\zBook{论美国的民主}{[法]托克维尔}{4}{}

\zBook{乌合之众}{[法]勒庞}{4}{}

\zBook{社会心理学}{}{5}{是美国大学的教材,写的很好}

\zBook{中国哲学史大纲}{胡适}{4}{还好,比较短。太长的哲学书有点看不进去了。}
	
\zBook{正义论}{约翰•罗尔斯}{4}{}

\zBook{大失败:20世纪共产主义的兴亡; 大棋局:美国的首要地位及其地缘战略}{[美]布热津斯基}{4}{}
		社会

\zBook{}{}{1}{}	著名反动分子布热津斯基的两株大毒草,充满了资产阶级和帝国主义对新生的共产主义的刻骨仇恨。

\zBook{通往奴役之路}{[英]F.A.Hayek}{4}{反社会主义的大毒草}

\zBook{枪炮、细菌和钢铁}{}{5}{各民族的社会和技术进步有巨大差别的地理学原因}

\zBook{裸猿}{D. Morris}{5}{人类的演化}

\zBook{异类}{[美]M. Gladwell}{1}{原名<Outliers>,讨论什么样的人有可能成功?}

\section{历史}
\zBook{天朝的崩溃; 近代的尺度}{茅海建}{5}{满清在鸦片战争时期为何不堪一击}

\zBook{叫魂}{}{5}{从一件事观察满清官场}

\zBook{欧洲中世纪史}{Benett/ Hollister}{5}{系统的介绍欧洲中世纪}

\zBook{古拉格群岛}{[俄]索尔仁尼琴}{5}{系统性描述苏联的罪恶}

\zBook{万历十五年; 中国大历史}{黄仁宇}{5}{大历史观}

\zBook{世界通史}{[美]L. Stavrianos}{5}{}

\zBook{旧制度和大革命}{}{4}{}

\zBook{光荣与梦想}{}{5}{美国历史}

\zBook{世界大人物丛书(上)}{}{4}{拿破仑、希特勒、爱因斯坦、牛顿、罗斯福、林肯、华盛顿的传记。我没读过(下)}

\zBook{东周列国志}{冯梦龙}{5}{我觉得比三国演义写得好}

\zBook{明朝那些事儿}{当年明月}{1}{}

\zBook{西方将主宰多久}{伊恩•莫里斯}{5}{原名<Why The West Rules>}	

\zBook{文明的历程}{}{4}{}

\zBook{大国的兴衰}{保罗•肯尼迪}{4}{}

\zBook{墓碑}{杨继绳}{5}{关于TG治下的大饥荒}

\zBook{毛泽东私人医生回忆录}{李志绥}{3}{此回忆录受到诸多质疑,为此我写了一小段,见读书篇}

\zBook{罗马人的故事}{[日]盐野七生}{1}{写的不错}

\zBook{}{}{1}{}
中国新史	[美]费正清	历史	

\zBook{}{}{1}{}
君主论	马基雅维利	权谋	好书,虽然被批判

\zBook{}{}{1}{}
方与圆		权谋	我的人际关系入门书

\zBook{}{}{1}{}
世界是平的
The World Is Flat	Thomas L. Friedman	经济	
大投机家
等,共13本	科斯托拉尼	经济	当时想炒股,就猛看了一阵子。后来觉得自己不适合。
牛奶可乐经济学	Robert Frank	经济	有趣的经济学书籍

\zBook{金融战争}{廖子光}{3}{不是那本垃圾货币战争}


国富论	Adam Smith	经济	经济学开山之作
制海权	马汉	军事	
风帆时代的海上战争		军事	
战争论	克劳塞维茨	军事	现代军事学开山之作
孙子兵法	孙子	军事	不长,两小时就看完了
高卢战记	恺撒	军事	

\zBook{德国人}{}{4}{}
			
\section{科学}
\zBook{从一到无穷大}{伽莫夫}{5}{科普名著}

\zBook{生命的壮阔}{}{5}{进化的极限,进化的一般规律}

\zBook{物种起源}{[英]达尔文}{5}{这本书意外的很有趣,讲述了很多即使到今天也很有用的知识}

\zBook{西方伪科学种种}{马丁•加德纳}{5}{西方科学界的“跳大神”时代,照样是群魔乱舞。}

\zBook{环球科学}{}{5}{科学美国人中文版,相比之下,中国的科普杂志都是小儿科。2006 开始在中国发行,我每期必读。}

\zBook{奇妙的人体之谜}{}{4}{适合儿童阅读,对于成年人有点简单}

\zBook{《十万个为什么》文革前版; 文革版; 文革后版}{}{4}{文革版内容不错,可充满了意识形态说教。文革后版内容太简单。}	
	
\zBook{趣味心理学系列}{}{4}{适合儿童阅读}

\zBook{阿西莫夫百科全书}{阿西莫夫}{5}{一个人写出来的百科全书,比成群作者写出来的还要好看}

\zBook{粘住}{}{5}{怎样吸引人的注意力}

\zBook{失控}{}{5}{关于混沌系统}

\section{经济}
\zBook{集装箱改变世界}{ [美]马克 莱文森}{}{4}

\section{军事}
\zBook{黑鹰坠落}{[美]Mark Bowden}{3}{原名<Black Hawk Down>电影《黑鹰坠落》的原著,讲述1993 年美军在索马里与军阀的冲突。}

\zBook{我的诺曼底}{唐师曾}{3}{写的不怎么样}

\zBook{战争就是这么回事儿}{袁腾飞}{4}{不够严谨}


中国哲学大纲	张岱年	哲学	不是写的不好,而是太难读了。只要你不是特别喜欢哲学,还是别看了。
中国哲学史	劳思光	哲学	写的还行。只适合想要深入学习者

\zBook{}{arg2}{arg3}{arg4}
龙枪编年史		小说	太长了
卫斯理全集	倪匡	小说	我花了一个月,看完了卫斯理全集160 余本。仅推荐三本:《眼睛》《妖火》《蓝血人》,其余都是垃圾。
童年、在人间、我的大学	高尔基	小说	虽然有名,我不喜欢
故事的事故	张远山	小说	个人文集
杞县故事卷		神话	杞县民间故事收集
开封故事卷			开封民间故事,TG在开封的故事
制空权	杜黑	军事	1921 年飞机还不够先进时就已经发表本书,是注重制空权的开山鼻祖。但,本书中讨论的技术已经明显过时。由于技术的进步,空战思想也有明显变化。
			
历史	希罗多德	历史	古希腊的史书。写的很好,但不适合今人阅读
台江内海研究	吴建昇	历史	对郑成功感兴趣的时候读过
史记	司马迁	历史	二十四史读起来很费劲。
史记看了三分之一,其它看了一小部分。
三国志	陈寿	历史	
宋史		历史	
资治通鉴	司马光	历史	看了1/20,费劲。
我和斯大林•斯大林情妇回忆录	Vera Davydova	历史	太八卦
世界历史知识	编者:孟涵	历史	不敢管自己叫“作者”的典型代表
中国人口史	葛剑雄	历史	不想研究历史就别看了,较无聊
西西里人	马里奥•普佐	历史	描述西西里的黑帮
奥德赛	荷马	历史	小说+神话+历史,不好看
罗马帝国衰亡史	E. Gibbon	历史	写的很好,但是有点深

郎咸平的书一大堆	郎咸平	经济	深入深出,难懂
领导经典		权谋	包装精美,价格昂贵,骗了老爹

\zBook{科学的历程}{}{3}{主要讲科学史}

中国大百科全书	四本巨厚	科学	我竟然看完了……
好像什么都没记住
生活中的化学		科学	写的很好的书,就是太老了
自然哲学的数学原理	牛顿	科学	开创了科学上的牛顿时代。太老
几何原本	欧几里德	科学	几何学开山之作。太老
中国地理常识		科学	中德对照,学德语用
中国西藏•事实与数字2007		社会	国家出的宣传品
共产党宣言	马克思	社会	
资本论	马克思	社会	没有自虐倾向别看
 
七、读过,但批判的书
书名	作者	题材	简述
\zBook{货币战争}{宋鸿兵}{1}{阴谋论的代表,从论据到结论都一塌糊涂}

\zBook{狼图腾}{}{1}{故事还行、结论呆蠢}

\zBook{}{}{1}{}
很多网络小说			太多,无法列举

\zBook{网上开店创业指南}{}{1}{}
编者:王剑		垃圾

\zBook{偏要和陌生人说话}{编者:黄志坚}{1}{段子集锦}

\zBook{圣经; 古兰经; 金刚经}{}{2}{不了解怎么批判?}

\zBook{本草纲目; 黄帝内经}{}{2}{首先,太难懂。其次,太多臆测不实之处需要后人去伪存真。}
