\chapter{共产篇}
“我们是共产主义接班人,沿着革命先辈的光荣路程……”

对我们来说,加入少先队、做共产主义接班人是没有选择余地的“理想”。然而大家在唱队歌之时,有没有考虑过在何时、在何地、在何等条件下才能实现共产主义呢?

本章就来讨论这个问题。

\zPar

 “实践是检验真理的唯一标准。”
 
共产主义,或者说共产主义的前身社会主义,已经在地球上诸多国家进行了实践。实践结果如何呢?

二战后在苏联淫威下,建立起庞大的社会主义阵营。然而运转40年后,社会矛盾激化,人民生活比起“邪恶堕落的资本主义”有了很大差距。于是在1989-1991,短短两三年间,红色帝国轰然倒塌。

现在的社会主义阵营只剩下“中朝越古老”,也就是中国、朝鲜、越南、古巴和老挝,普遍贫穷、愚昧、落后。中国稍微好点,在TG领(宣)导(传)下是一片繁荣富强稳定,除了官员贪污腐败、环境污染严重、物价飞速上涨、社会贫富悬殊……之外就没什么问题了。

简单来说,共产主义和社会主义用哪哪不灵,放哪哪出事。这么邪门的表现就说明它的理论基础一定是有点问题的。

\zPar

写到这里,我应该终结本章,并宣布共产主义是错误的。但按照我的“复杂历史观”,不能轻谈“历史必然性”。所以,我将从理论的角度加以分析。

\section{共产主义乌托邦}

我们先来假设一个美好的共产主义社会存在于“乌托邦”。这个乌托邦要有哪些特征呢?

按照老马的原初设想,共产主义是完美的人类社会。所以,一切有关“社会”的褒义词都应该可以用来形容乌托邦。但这样的词太多了,我只好择其要点:

\begin{description}
\item[富裕] “贫穷不是共产主义”。共产主义是要实现“按需分配”的,它建立在“社会物资极大丰富”的基础上。
\item[平等和公平] 老马和其他一群人,因为看不惯旧制度下的贫富分化才提出了共产主义。如果共产主义的贫富分化更加严重,那还不如资本主义呢。
\item[高素质人民] 如果人民的素质不高,这样一个美好的乌托邦会在转瞬间被一群恶棍毁掉。所以,乌托邦必须由“素质极大提高”的人民组成。然而,提高到何等程度才算是“素质极大提高”呢?这是一个值得思考的问题。
\item[民主和自由] 老马认为资本主义是假民主、假自由,所以要在共产主义社会实现真民主和真自由【缺少引用】。
\end{description}

接着,我们来探讨一下这个乌托邦会怎样运转。

\subsection{按需分配}

按照当下中国的政治理论,共产主义要实现“按需分配”。理想很丰满,现实很骨感,这个目标很可能永远都无法实现。

\subsubsection{欲望无限}

“需”体现的是人类的欲望,而人的欲望是无止境的。

每人一辆自行车很容易,每人一辆汽车也不难,那每人一架直升机呢?每人一艘游艇呢?每人一架喷气机呢?地球上的有限资源不可能满足人类的无限需求。

如果想要的不是物资,而是“空间”,则是另一个难题。地球已经很拥挤了,每个人的空间都很有限。共产主义要怎样分配有限的生活空间呢?肯定是无法满足每个人的“需”。

以上说的只是自然资源,再来说说“人的资源”。比如有人想要左拥右抱、妻妾成群,或者被一群人服侍,这就彻底无法满足了。

对此,恐怕只能寄希望于素质极高的人民了,他们会自发的不索取那么多。

\subsubsection{需求和供给}

共产主义的实现,是在“物质极大丰富”的前提下,实现“按需分配”。要知道,需求和供给有关。绝大多数人都想长生不老,所以长生不老药的需求量极大,可惜供给为零,故而此类需求是没有意义的。

即使是完全相同的供给,获得时间的不同,也会造成不公。苹果手机发布时,有人在商店门口彻夜排队,只为提前几天拿到手里。其实他们只要耐心的等候几个月,在店里不用排队就可以买到。

所以,按需分配的时候还要考虑各人获得的顺序,难度更大了。如果非要保证所有人同时获得,那就先得囤积巨量的供给,然后全乌托邦统一在某个时间点开始发放,搞的像高考一样。

如果技术停滞不前、一直没有新的供给的话,确实有可能实现“按需分配”。假如出现了一种新的供给,比如治疗癌症的特效药,可惜这种药物的供给量极小,如何去分配呢?社会的平等不是又要被打破了吗?

技术发展对于人类的重大意义已无需敷言。目前看来,技术的发展是无止境的。难道我们要为了人人平等而放弃技术进步吗?

\subsubsection{创造需求}

我们逛商店、逛网店时,会突然看到一个有趣的产品,然后才觉得自己需要这个产品。在发现供给之前,我们或许根本不知道自己有此需要。个人的需求是复杂而难以确定的,甚至自己都不清楚。这就是Steve Jobs所说的:“用户不知道他们想要的是什么。”【缺少引用】

这个世界上,只有少数人才知道众人的潜在需求,并发明、创造出供给。没有人能完全知道自己需要什么产品,最伟大的发明家也会购买其他人的产品。

如果没有一种机制,促使有才能的人去发明创造,那这个社会就会逐渐落后。但是在人类的诸多发明创造中,只有很小一部分是有实际价值的。在资本主义社会,通过市场竞争来筛选出符合人类需求的产品。乌托邦要怎么做这个筛选工作呢?

\subsubsection{广告效应}

合理的广告,会强烈的调动人的需求欲望。广告产业已经发展为价值数百亿的巨型产业,可见广告的巨大效果。

是否允许在乌托邦里做广告呢?

如果可以随意做广告,那就会调动起人的欲望。有了欲望的乌托邦人民就会过度索取,超出乌托邦的生产力。

看来广告真是个麻烦事,要不干脆把广告禁了?

\zPar

但如果禁用广告的话,恐怕也会有许多难题。

乌托邦也要和外国有所交流,如果外国的媒体在乌托邦里做广告,允许吗?

即使禁用外国媒体的广告,乌托邦作为一个“真自由”的国度,它的人民自然是可以访问外国人的网站。如果在外国网站上看到了一些广告,刺激了欲望呢?

假设有外国在乌托邦拥有某些地产,那么外国人在自己的地产上树立一个巨大的广告牌,是否合法呢?

再比如,现在有很多“软广告”、“植入广告”,乌托邦的法律把这些广告也要禁止吗?

\subsubsection{性选择}

雄孔雀的尾巴曾经让达尔文费尽心思,这么美丽的尾羽用处不大,却是身体的巨大负担,为什么没有进化掉呢?

最终他想明白了:这是“性选择”的结果。那些拥有绚丽尾羽的雄孔雀才可以获得雌孔雀的青睐,并留下后代,丑陋的雄孔雀都被淘汰了。

\zPar

不管物资的供应怎样变化,美女帅哥的供应总是有限的,而发情期的男女智商和理性总是要差一些。在美好的乌托邦社会里,略有一点私心就可以很容易的获得巨大的优势,于是这些人更有机会把基因传播下去。最终,整个乌托邦都会变成自私者的后裔——如果乌托邦没有在此之前崩溃的话。

\subsubsection{外观与品质}

各位读者(尤其是女读者)可以想一下,自己的衣服有几件是必须的。

从最小需求的角度说,现代工业社会解决所有人的穿衣问题是很轻松的。但即使像我这样对穿戴很不讲究的人,也拥有远超需求的衣服储备。

衣服除了遮蔽、保暖的效果外,还是表达个性、展示品位的工具,并在性选择的过程中发挥巨大作用。

\zPar

女人经常会担忧“撞衫”(有些男人也会,但很多男人都无所谓,比如我),也就是说,她们追求对某种款式的独占性。

那么问题来了,如果某个款式非常漂亮,所有女人都喜欢,那要分配给哪些人呢?如果穿的人太多,就没有了独占性,其他人就不愿意要了,所以只能分配给很少的一部分人。

啊,女人,就这样敲响了共产主义的丧钟。

\zPar

物资还有品质的差异。最优质的物资数量往往非常稀少,但需求量又很大,比如足球看台上最好的几十个座位。资本主义的处理方式是将其价格炒高,牟取暴利。

或许有人会提出这样的方案:足球看台上一个最好的座位等于三个较差的座位。如果这样做的话,那不就成了“货币式”解决方案了吗?又退回到了资本主义的方式。

\subsection{最大的欺骗性}
如上节所言,按需分配或许永远也无法实现。

现有的共产主义理论最大的欺骗性就在于:它假设人民素质可以无限提高,最终达到“至公无私”的地步。

乌托邦的人民素质必须达到这样的程度:

\begin{enumerate}
\item 在需要工作的时候,没有人偷懒,也没有人懈怠
\item 在从事科研和发明创造工作的时候,所有人都竭尽所能、释放才华
\item 在分配物资的时候,每个人都满足于自己应得的一份,从不奢求更多
\item 如果分给了自己过多的物资,要主动将其退回
\item 在从事高危行业和参加战争的时候,每个人都奋勇向前、无所畏惧
\item ……
\end{enumerate}

我还可以举出更多的例子,但已经没有必要了。你或许会说,只有人像蜜蜂或者蚂蚁那些“真社会性”昆虫一样,才能达成共产主义社会。但人类是多么复杂的生物,岂是蜜蜂蚂蚁比得了的。更何况,蜜蜂和蚂蚁也会偷懒【缺少引用】。

\subsection{微自私假设}

我们这样假设,有一小撮乌托邦人民有那么一丝贪欲。我将贪欲进一步区分为“违反法律和道德的贪欲”与“不违反法律和道德的贪欲”。乌托邦的人民当然是相当高尚的,他们即使贪,也不会违反法律和道德。

如果乌托邦的人民只有一点点贪欲,又会产生怎样的后果呢?

\subsubsection{加班和自由交换}

乌托邦里,每个人的工作时间应该是一样长的,至少是差不多长的。每个人的财富也是差不多的。

但如果有人自愿加班,并且想要用加班创造的额外财富与其他人进行自由交换呢?

如果禁止加班,禁止自由交换,那乌托邦就不是一个自由的国度。如果全都允许,那就会有一些人的财富超过其他人。

这还只是小问题。

\zPar

大问题是:按照老马的设想,货币在共产主义阶段会消亡。

假如共产主义社会的初次分配不够理想,或者是有些旧物品不想要了,或者交易加班创造的额外财富,都需要自由交换。

如果有自由交换的话,还是保留货币才会比较方便,否则就退回到了以物易物的原始社会。

只要有自由交换的存在,就会出现一批“可恶的”掮客,他们很容易积聚起远远超过平均水平的财富。可以说,自由交换必然会产生贫富差距。贫富差距产生之后怎么办?乌托邦要怎么对待这批“钻共产主义空子”的人呢?

历史上出现的办法是“割资本主义尾巴”,结果是造成了全民贫穷。不仅如此,“割尾巴”的做法是严重损害共产主义的初衷的,因为这个乌托邦理应是更自由、更民主的国度,与资本主义“虚伪的自由、伪善的民主”有本质区别。

\zPar

如果全面禁止自由交换,还会有更大的问题。因为\textbf{交换可以创造财富,即使没有任何交换是绝对公平的。}

比如:甲有100个香蕉,乙有100个苹果。甲愿意用5个香蕉换1个苹果,乙愿意用5个苹果换1个香蕉。如果两人交换1个香蕉和1个苹果,那两人的财富都得到了提升。这是因为\emph{交换使得财富得到了更优化的配置}。

\textbf{禁止交换不单单破坏了乌托邦的自由,而且阻碍了财富的优化配置}。

\subsubsection{社会分工和人才外流}

老马设想的“乌托邦”,社会分工程度是要降低的,每个人都有闲暇做自己的事情【缺少引用】。可是早在三百多年前的《国富论》就已经指出,社会效率来自于社会分工,社会分工的细化程度甚至直接反映人均产值。这样一个降低了社会分工的乌托邦会在国际竞争中取胜吗?

如果乌托邦也实行高度细化的社会分工,是否可以解决这个问题呢?

很难。

\zPar

有一些创造性的劳动是“不可数值堆叠”的,比如一些数学难题,不可能用“全民大炼钢铁”的方式解决。有能力从事此类劳动的人,是国家的宝贵财富。

我在《智力篇》提到,社会上有2.2\%的人智力超过130。智力不足者,不管怎么努力,都无法处理一些艰深的难题。一个高素质人才所能创造的财富,要数倍甚至数千倍于平均值。故而,这些受过良好教育的聪明人就可以待价而沽。

虽然乌托邦是全世界平均最富裕的国度,但资本主义国家贫富分化严重,其中的富人要远比乌托邦的公民有钱。如果这些邪恶资本主义国家前来挖墙脚,用金钱、美女什么的诱惑乌托邦里的科学家和工程师,乌托邦要怎么留住他们呢?

如果用行政手段限制出境,那就违反了人身自由;如果用更多的金钱、更美的美女(美男)就破坏了平等和公平;如果任由他们离开,就会导致人才外流,很多行业都会被卡在一些门槛上无法进步。

\subsubsection{机器人时代}

马克思很有远见的预见到了机器的功效。他认为,未来工作可能全部都由机器完成,人只需要轻松的发明机器、管理机器即可,其它时间都可以用来休闲,做自己想做的事 [1]。

我得说这是个很棒的想法,很适合写成科幻小说。

以当今科学的发展速度而论,50年内推出足以完成大多数工作的人形机器人绝不是开玩笑。这本书的绝大多数内容是一个程序员的“跨界”闲扯,但这句话是罕有的专业预测。

机器人发展的下一步就是“像人类一样思考”,能够完成人类的工作并不代表能像人类一样思考。突破这一步相当不易,需要一些创造性的进展。

突破之后又如何呢?我们又要面对机器人的社会伦理问题。假如机器人比人更加智慧,难道要把人类像宠物般养起来吗?

这些令人头疼的问题很容易就可以提一大堆。机器人时代究竟会如何,我们现在是不可能想明白的。

所以,想不明白的事情干脆就别想。

\zPar

想明白又如何呢?即使机器人时代到来,也不能解决人的贪欲问题。人还是会想要更大的房子、更先进的机器人。

而且,在机器人时代到来之前,技术型工作的岗位一直在增加,技术型工作的难度也在增加,以至于要接受20年以上的教育才能从事尖端研究。

对于社会上的多数人来说,思考并不是一件轻松的事,脑力劳动比体力劳动还要艰苦许多。很多人宁可从事体力活也不愿学习。有时候,我也有类似的感觉。大学之前,连下地干农活都比学习有趣许多。

如果从事科研的人不能获得数倍于社会平均的收入,他们有什么理由克服学习过程中的无聊呢?

\section{社会主义实践}
在上一节中,我用“反证法”讨论了共产主义乌托邦运转中存在的问题。在这一节,我将向大家展示共产主义/社会主义实际存在的问题。

即使人口素质极高,只是“微自私”,都会导致乌托邦运转失灵。那如果人口素质不高,却强行要执行共产主义政策,会出现什么情况呢?

\subsection{按劳分配}
社会主义的分配方式是“按劳分配”,听起来是不错的。然而,劳动的形式多种多样,怎样来评估劳动贡献呢?

我分别从劳动量和劳动价值的角度来讨论。

\subsubsection{劳动量}
如果是挖沟,那判别每个人的劳动量是很容易的,只要测量一下土石方数即可。

但如果是锄地呢?表面上看起来跟挖沟差不多,只要看锄的面积即可。然而,不同的庄稼要求锄的深度不同,如果有人锄的面积看似很大,但其实不够深呢?锄地的时候还要小心的避开苗根,如果乱锄一通,轻则减产,重则死苗。

判别每个人的劳动量绝不是个简单的问题。中国曾经实行工分制,上工就可以得到工分,最后按照工分来进行分配。但工分制忽略了实际劳动态度和工作质量,多劳不能多得,偷懒也不会受到惩罚。最终还是损害了劳动积极性。

资本主义的工人由资本家监督,如果某个资本家监督不当,那他的工人劳动生产率就会比较低。他赚取的金钱就会减少,甚至有可能导致破产。最终,只留下监督效果好的企业。

为什么流水线能够大大提高劳动效率?除了反复从事同一个生产环节,提高了熟练度之外,还因为在流水线上,偷懒的人会一目了然。流水线的速度经过调整,所有工人都必须全力以赴才能保证流水线正常运转。大家可以到流水线上体验一下生活。第一天下来,绝对是腰酸背痛。

\subsubsection{劳动价值}
目前看来,技术含量低的体力劳动的价值是接近的,比较符合马克思主义经济学里的“价值是凝结在商品中无差别的人类劳动”。对于科技和工艺有很高要求的劳动就不是这么简单了。

我家的瓷砖有一部分大概是个新手贴的,砖缝粗细不均,差大约0.5厘米。这本来是个小事,但随着我完美主义倾向的滋长,每次看到那一部分瓷砖都让我很不舒服,甚至有将其砸了重贴的想法。如果忽视劳动价值的话,劳动者就缺乏提升技艺水平的愿望。

贴瓷砖的水平还比较容易评判。最难评判的,是讲究独创性的艺术和研究了不知道有什么用的科学。在《艺术篇》和《科学篇》里,我会进一步加以讨论。

\subsubsection{私有制}
如果一个地区车祸频发,那这个地区的汽车往往得不到良好的保养。因为保养的再好,也不知什么时候被撞烂。

中国的房子是70年产权,现在还不知道产权到期后到底会发生什么。但如果你知道你的房子五年后就要被政府收缴,好吧我们换个说法,五年后就要“交还”给政府,你还会再维修、养护它吗?你心里大概会默默祈祷:别坏、别坏,撑到交房那天就行。

\textbf{公有制最大的误区在于,把财富当作固定值}。如果财富的总值有限,当某些人占有的财富太多,其他人的财富就会减少,甚至陷入贫穷。但财富总量是可变的。随着科技的进步,财富与资源的关联度越来越低,与勤劳和智慧的关联度越来越高。私有制保证了劳动的回报,才能激发出人的能力。

\zPar

中国的例子我们已经看腻了,看看越南的。

越南有约80\%的人口生活在农村。上世纪七八十年代,越南每年都要进口一百多万吨粮食来解决百姓的吃饭问题。1987年越南学习中国,实施承包责任制。到了2005年,越南大米出口达520万吨,成为世界第二大大米出口国。

\zPar

中东有石油富豪,山西有煤老板,都让人羡慕嫉妒恨。对于矿产资源,是否应该实行公有制呢?

2015年初有一条新闻,一个新疆牧民捡到一块7.85kg的“狗头金”(天然金块),估价300万元以上,引起热议。有些人说,这块金子要归国家,捡到的也要交公,因为在法律上,无主土地属于国家。

截止本文,这块金子的归属尚未确定。但在我看来,“矿产”如果不被开发,那就只是“矿”;开发之后,才算做“产”。与其让矿产躺在地下睡觉,不如让私人将其开发出来。

在大漠上找寻矿产不是一件轻松愉快的差事,如果有人愿意去做,就说明这件事值得做。他们把矿产带入社会流通环节,才实现了矿产的价值。

如果不被人发现,狗头金只是戈壁上一块普通的石头,跟其它亿万块石头没有本质区别。

开矿往往是暴利行业,而且会造成严重污染。目前最好的做法是,用矿产税来保证其他人也可以分享矿产收益,用环保税来抑制污染,用劳动法防止工人的劳动条件恶化。

\subsection{决策机制}

资本主义的决策成本是很高的。生产资料的分配、人力的分配、产品的分配都需要大量的人工。每个企业的人力资源部、会计、推销员,还有庞大的广告产业,都可以算作决策成本的一部分。

如果能够大大缩减这部分决策成本,似乎可以让社会美好许多。

于是,在社会主义国家里,统一调度代替了市场竞争,工作分配代替了人力市场。

在市场经济下,决策者要为自己的抉择负责。如果他做出了错误的选择,就会为此付出代价。所以,决策者要下很大的功夫分析、比较各选项,才能做出正确的抉择。

在社会主义下,决策者负的责任要少得多,而且,往往决策者的选项也少得多。历史表明,他们的决策水平,实在不怎么样。
这群社会主义的决策者缺乏监督,实际上形成了社会主义的特权阶层。

\subsection{阶级论}

“全世界的无产者,联合起来!”(德文:Proletarier aller Laender, vereinigt euch!)

马克思和恩格斯在《共产党宣言》里写下这句话,认为全世界的无(穷)产(光)者(蛋)会是一条心。

马克思主义的阶级论有四个要点:
\begin{description}
\item[阶级差异] 不同的阶级之间有很大的差别,主要体现在财富和社会地位上。
\item[阶级固化] 不同的阶级之间难以互相转换
\item[阶级认同] 同一个阶级内的人群会产生强烈的认同感。
\item[阶级对立] 不同阶级之间有很强的对立情绪,最终导致阶级矛盾、阶级仇恨和阶级冲突。
\end{description}

可以说,阶级差异和阶级固化是“因”,阶级认同和阶级对立是“果”。

一个良好运转的社会,必须把阶级对立控制在一定范围内,防止其激烈到阶级冲突的地步。

\subsection{暴力革命}

“\textit{每个人的自由发展是一切人的自由发展的条件}”。《共产党宣言》里的这句话说明马克思也是追求自由的。

对于阶级对立和阶级冲突,马克思提出的解决方案是无产阶级革命,创造一个公平的社会,彻底消除阶级差异。

但是旧有的资产阶级肯定不乐意把财富乖乖上缴,无产阶级只能通过暴力革命才能实现目标。

暴力革命的过程中,必须建立独裁统治,否则就无法形成战斗力。任何一个国家的军队内部都是独裁的。

但是独裁统治建立之后,如果权柄落在大野心家的手里,会出现什么情况?

请君只看前苏联。

\subsection{共产不腐}

很多人怀念毛时代。\\
对此的解释之一是“不患寡而患不均,不患贫而患不安”。很多人觉得,毛时代的中国贫富差距小,那时的官员还比较老实。

现在的中国贫富差距大,这没有异议,然而毛时代的贫富差距小吗?\\
现在的官员贪的丧心病狂,这没有异议。然而毛时代的官员真就不贪么?

\zPar

有个游戏叫《文明》,模拟了人类社会的发展和世界各主要文明的特征。在游戏中可以选择国家制度,其中共产主义的优势是:它是贪腐率最低的制度。

提到这一点有些搞笑的意味。但我在网上辩论时,有论敌将其作为共产主义贪腐率低的证据。

\zPar

跨越时间和空间来比较贪腐确实不易,我想从三个方面着手:

\begin{description}
\item[基本生存权] 有一座城,每个人拥有的物资财富都一样多,人人都能吃饱穿暖。有一只恶龙,它每天都来吃一个人,居民拿它没有办法。但恶龙有个怪癖:从来不吃当官的。请问:这座城里的官民差距有多大?这个官位,等价于多少财富呢?

如果连基本的生存权都无法保障,就无法进一步比较物资财富的差距。

在饥荒时期,如果家里有个干部,就能多获得一些宝贵的粮食保住性命。这要怎么评估?

\item[贪腐/平均收入系数]
从“贪腐/平均收入系数”的角度,毛时代的贪官不比现在的贪官差。比如1951年的刘青山、张子善案件,两人共贪、骗171.6亿元,折合新币 171.6万元。按照当年的工人平均收入\footnote{},够3万个月的。

【由于我是理工科,对这类考据实在乏力。以后补充数据吧。】

\item[特权式贪腐]

俗话说,皇帝有“三宫六院七十二妃”(实际还不止这个数),这就是制度性腐败:从制度上就享有特权。

在知青返乡时期,很多女知青为了获得一两个公章,不惜和生产队干部睡觉。

【这一段有待补充】
\end{description}

\subsection{历史事实}

我们来看看历史上实际发生的事:

\begin{itemize}
\item 由于实行平均主义,导致人民生产积极性下降,出现了广泛的饥荒,数千万人饿死(苏联、中国)
\item 不管哪个地区,只要管的稍微松一点,黑市立刻就繁荣起来【秦晖】
\item 分配工作时,往往要贿赂官员,才能得到较好的职位
\item 	为了缓解内部矛盾,社会主义阵营强调与资本主义阵营的对抗,工业生产普遍向军工倾斜,轻工业产品匮乏。【苏联的实际军费甚至达到GDP的1/3,需查证】
\item 	工业产品重视产量和某些片面的指标,很多产品虽然数量充足但实际不符合人民的需求。【苏联某冰箱以噪音巨大著称,需查证】
\item 艺术设计和工业设计是在管理中难以量化的,因此不被重视,发展严重滞后。产品丑陋,服饰制服化,异常单调
\item 同样的原因,武器设计重视火力、产量等片面指标。比如飞机不考虑逃生效果;坦克不考虑乘员舒适性,像蒸笼
\item 为了维持汇率稳定,实施普遍的外汇管制。但实际上有价无市,汇率黑市盛行。在放开外汇管制的时候,汇率一落千丈(苏联卢布)
\item 宣称TG是永远伟大光荣正确的党派,拒绝社会监督,形成了特权阶层
\item 	野心家导演了全国性的个人崇拜,即使是错误的决策也会被坚决执行。错误的决策对国家造成了巨大的危害(斯大林、毛、铁托、金氏家族、齐奥塞斯库)
\item 	由于限制人身自由和言论自由,高层次人才大量流失海外。东德是社会主义阵营中经济最发达的国家【待查证】,仍不能阻止人民逃亡
\item 由于掌握了社会的调度权力,贿赂盛行,形成了特权阶层
\item ……
\end{itemize}

\section{理论根源}

有一种说法是,马克思主义本身是好的,但是被苏俄这个“歪嘴和尚”把“经”念坏了。

我得说,苏俄在其中确实起到了极坏极坏的作用,它把共产主义变成了残暴的统治工具,并且“以点带面”,带坏了整个社会主义阵营。如果不是苏俄,共产主义至少会温和很多。

如果马克思生活在1980年之后,我想他很难得出“共产主义社会是人类社会发展的终极形态”这一结论。

这让我忍不住去探寻马克思的理论根源:是什么样的社会环境导致马克思得出了这个结论呢?

\subsection{城市化进程}

1830年欧洲的城市化率是12.6\%,1850年是16.4\%,1880年是23.5\% [2]。相比之下,2000年中国大陆的城市化率为36.2%。
这说明当时的欧洲城乡分野非常明显,呈现出“城乡二元经济”的特点。

农产品的生产效率很难大大提高,从没有化肥、农药、良种的时代到袁隆平的“超级杂交水稻”,亩产也不过提高十倍而已。但在这个时间段里,工业产量提升了何止千倍。
在城市化的进程中,“新来乍到”的失地农民往往只能从事较为辛苦且低下的工作。
如果马克思在北京接触过蜗居的“北漂”,他大概会说:“全中国的蚁族,联合起来!你们失去的只有锁链,得到的是整个中国。”

\subsubsection{漫长的第一次工业革命}

膛线的最早应用是在15世纪末,而广泛应用于火枪是在19世纪中期。膛线对于火器的射程、精度和威力都有很大提升,但在又细又薄的枪管内镗出均匀且精密的膛线实在是太难了,还要保证不炸膛。火炮的膛线很早就开始应用了,因为炮管很粗,而且膛壁很厚。

膛线广泛应用于火枪,可以说是冶金学和机械加工学全面发展的结果。

虽然在第一次工业革命中出现了很多发明,但比起第二次工业革命及此后的飞速进展实在算不了什么。因为世界各国逐渐意识到科技很重要、教育很重要,并且人类搞懂了怎样去研究。

\subsubsection{老欧洲的贵族气}
资本主义兴起的时候,欧洲还有大量的贵族。贵族天生就有特权,在社会中高高在上。他们是相当固化的阶级。

\subsubsection{结论}
当一个行业充分发展后,就会被占满。先发优势是很大的,后来者很难跟跟先行者竞争。

第一次工业革命相对漫长,有助于阶级固化的发生。

如果技术飞速发展,就会不断产生新的行业、新的机会。无产阶级可以通过勤奋和智慧上升为资产阶级,资产阶级也会因为投资失利而沦为无产阶级。
也就是说,不再满足阶级论里的“阶级固化”条件。

\zPar

在“城乡二元经济”走向“一元经济”的过程中,最贫穷的一群人会减少,也就是说最穷的无产阶级越来越少了,他们的革命意愿也就不再坚定。
也就是说,阶级差异和阶级对立减轻了。

更进一步的,发达国家推行社会福利和劳动法,进一步减轻了阶级差异,缓解了阶级矛盾。

所谓的“共产主义第一伟人”马克思,其实也只不过根据他那个的时代所发生的事情做出的判断。他既没有看清共产主义的问题,也没有看到资本主义的解决方案。

\section{倒立的不倒翁}
有没有可能把一个不倒翁倒立在桌子上?

很多人会说不可能。其实从物理学原理的角度说,是可能的。任何物体都存在一个重心,只要让重心垂直于支撑点,它就会稳定。

然而从生活实际的角度说,是不可能的。因为倒立的不倒翁的支撑点很小,很难摆对角度。即使摆对了,细微的晃动甚至空气的扰动、走路的震动都有可能使其重心偏离支撑点,最终回到站立状态。

共产主义就是这样一个倒立的不倒翁,它必须建立在所有人拥有完美人格、至公无私的基础上。更何况,也许人类永远都无法使其倒立起来。

\zPar

人类是非常贪婪的。\textbf{资本主义市场经济假设所有人都是“理性经济人”——既贪婪、又精明,并由此演化出一套经济学理论}。

这套理论的效果如何呢?

虽然资本主义国家也有穷国,但所有的“智力型发达国家”都实行资本主义。从现实世界可以得出结论:实行资本主义是成为智力型发达国家的必要非充分条件。

资本主义,就是一个站立的不倒翁,它把人类的欲望转化为动力,把民主和自由作为巨大的弧形支撑面。它摇摇晃晃,却从未真正的倒下。

\section{共产主义的死刑}
写了这么多,你是否觉得我已经给共产主义判了死刑?

不,我还不打算给它判死刑。因为技术的发展有可能带来嬗变。

\zPar

如前文所述,\textbf{共产主义之难局,在于人类欲望和分配方式}。人类欲望是可以控制的。人的欲望、人的意志、人的思维,仅仅是由不到两公斤的脂肪团里的电脉冲和化学信号产生的。因此,至少在理论上存在可能性,用药物或者基因的方式降低人的欲望,使得人人“从骨子里”、“打娘胎里”就道德高尚。

人类社会之所以充满了各种黑暗与丑恶,一个重要原因是我们无法探知其他人的真实想法。假如某一天,技术发展使得人类大脑直接相连,带来通讯速度的迅猛提升的同时,也使得每个人的思维暴露在众人的面前。从此,阴谋诡计再无隐藏之地。到了那一天,共产主义会降临世间吗?

我们必须看到,这种技术的背后暗藏着巨大的风险。假如有阴谋家通过修改大脑通讯程序,使得自己可以隐藏真实想法,那他无疑会获得巨大的优势。更进一步的,如果他能隐藏自己的真实想法,那他离控制别人的思维也不远了。他甚至可以控制全世界!想想就令人不寒而栗。

我们愿意将自己的想法暴露给他人吗?人类会对这类技术手段充满疑虑吗?我们愿意将自己处于此等风险之下吗?

\zPar

那为什么TG还要把实现共产主义当作自己的口号?中国不是早就资本主义化了吗?

TG的上台,是因为把共产主义作为大目标,当作整个党派的努力方向。即使它的高层已经意识到自己的错误,它也无法改口,因为会损害它自己的执政合法性。换句话说,TG被自己的理论“绑架”了。

可是,把这个虚无缥缈、约等于零的目标作为十几亿人的奋斗方向,合适吗?

\section{总结}

\begin{itemize}
	\item 按需分配不可能实现
	\item 现有的共产主义理论最大的欺骗性就在于:它假设人民素质可以无限提高,最终达到至公无私的地步。
	\item 自由交换必然会产生贫富差距
	\item 没有任何交换是绝对公平的
	\item 社会主义阵营的国家普遍落后
	\item 毛时代照样腐败
\end{itemize}

结论:\emph{共产主义难以实现且极不稳定,不宜作为国家的奋斗目标。}


%
%“纳粹”的意思是“民族社会主义”,他们信奉的可也是一种“社会主义”。希特勒是这么幻想未来的:“第三帝国到处都是仓库”。
%
%\footnote{德语中Nationalsozialismus 由两个词组成,National 意为“国家”、“民族”,}
%
%
%即使这个社会的分配已经保证了人人富足,
%
%
%\underline{平均主义}无疑是一种错误的分配方式,因为每个人的需求是不同的,同样的物质给不同的人会有不同的效果。比如有一盒围棋,360个棋子倒是很容易分配的很均匀,可是每人拿几个棋子有什么意思?不如集中起来给我这个围棋爱好者使用。
%
%
%不同的物品分配给不同的人会有
%
%共产主义的决策机制,也是个令人头疼的问题。
%
%之前讨论的都是共产主义国家内的情况,如果探究乌托邦和其它“落后的”资本主义国家的贸易,那更是一个令人头疼的问题:
%
%假如外国要输入一些会导致乌托邦内部不平等的商品,
%
%假如有才能的本国居民不满足于平等的生活,去外国发挥自己的才能,浸泡在资本主义的灯红酒绿,怎么办?
%
%如果现在要向
%
%\section{倒立的不倒翁}
%
%有没有可能把一个不倒翁倒立在桌子上?很多人会说不可能。其实从物理学的角度说,是可能的。任何物体都存在一个重心,只要让重心垂直于支撑点,它就会稳定。然而从生活的角度说,是不可能的。因为倒立的不倒翁的支撑点很小,很难摆对角度。即使摆对了,细微的晃动甚至空气的扰动、走路的震动都有可能使其重心偏离支撑点,最终回到站立状态。
%
%共产主义就是这样一个倒立的不倒翁,也许人类永远都无法使其倒立起来。而资本主义,是一个站立的不倒翁,它把人类的欲望转化为动力,把民主和自由作为巨大的弧形支撑面。它摇摇晃晃,却从未真正的倒下。
