\chapter*{自序}

2015-01-07 巴黎 《查理周刊》事件

这是欧洲的911 事件,却促使我完成计划已久《宗教篇》。

然而,我发现我又给自己挖了一个大坑,不,是巨坑、是天坑啊。

一如既往的,我想把事儿整的特别完备。宗教总是和历史、政治、文化、科学、战争发生着剪不断、理还乱的关系。描述宗教的时候,顺手写了写《战争篇》,然后《文化篇》、《科学篇》……每篇都写了一点,每篇都没写尽兴。
	哎呀呀,这就收不住了。十几年来枕上、厕上、车上的思索连绵涌来,促使我写个不停。原本只想写个《宗教篇》,但这一篇不好写,所以先完成了些好写的。
	如果我要把所有的文章全部完成再发布的话,可能拖到90 岁也完成不了。按编程习惯,阶段性发布吧。

你现在读到的这本书,是一个叫张航宇的小青年写的。

也许有一天,我会躺在床上对子孙说:“我一生的荣耀,不在于成功架构了52个大项目中的40个,上次的系统崩溃已经让这一切化为泡影”

这不是我从哪里看的,而是我原创的思想。如果你在任何地方看过相同乃至相似的说法,请联系我。我会考虑修改我的文章,至少会将前人的思想加入引用。

圣岛上,垂死的拿破仑说:“打赢了”

这本书,有什么特色呢?

我尽可能把原因拆解为极简单的道理,如果你觉得

“这么简单,谁不知道。”

扯远点,大学时候学了“布尔代数”。

这种数学的原理简单极了,可能的数值一共就两种:0或1.

有年冬天我和老妈都咳嗽了,于是就一起去喷喉。喷喉机的作用是把液体药物雾化,然后通过一根软管喷到喉咙里。我很想知道喷喉机的原理,于是当医生不在的时候,

美剧《英雄》(Heros)里的大反派Sylar有一项超能力,可以打开别人的脑壳,观察别人脑子的运行来

同样的,我对万物运行的原理都有强烈的兴趣。

题记
2015-01-07 巴黎 《查理周刊》事件
	这是欧洲的911 事件,却促使我完成计划已久《宗教篇》。
	然而,我发现我又给自己挖了一个大坑,不,是巨坑、是天坑啊。
	一如既往的,我想把事儿整的特别完备。宗教总是和历史、政治、文化、科学、战争发生着剪不断、理还乱的关系。描述宗教的时候,顺手写了写《战争篇》,然后《文化篇》、《科学篇》……每篇都写了一点,每篇都没写尽兴。
	哎呀呀,这就收不住了。十几年来枕上、厕上、车上的思索连绵涌来,促使我写个不停。原本只想写个《宗教篇》,但这一篇不好写,所以先完成了些好写的。
	如果我要把所有的文章全部完成再发布的话,可能拖到90 岁也完成不了。按编程习惯,阶段性发布吧。

自序
若干年前,看过一个某“写书最快的人”的采访。此人三十不到的年龄(被采访时),已经出了二十多本书。这个成绩确实令人惊讶。据他说,写一本书只需要几十天时间,最快的时候两天就能写完 。
当时我刚上大学,独立的“三观”还没有成型,思想上匍匐在地,仰望全世界的牛人。这位“写作大师”出身名校,让我不禁对所处的大学自惭形秽,对高考的失败痛彻心肺,恍然间有回去复读一年的冲动。可是对噩梦般的高中生活的回忆,总是将这一丝冲动吹散。

大学时适逢起点中文网兴起,立即召唤了一批怀揣文学梦、武侠梦、历史梦的小青年。据说有些顶级写手年入十万(现在还不止这个数),引得我垂涎欲滴,哈喇子把键盘都短路了。想当顶级写手也甚是不易,首先你得保证有人看,其次你得保证更新速度,保底是五千字/天,顶级的起码上万。
我一直很臭美的觉得自己才思敏捷,搁古代是个“上马击狂胡,下马草军书”的人物。再加上我从小就在各类小说(除言情)上浪费了大把大把的时间,肚子里积攒的素材也是满满的。咱不说七步成诗、倚马千言,达到个中上等水平总是没问题吧?然而我很快发现这对我来说是个不可逾越的障碍。
我构思一个完整的小情节就需要两三天。设计完人物和场景,斟酌每个人的对话又得花好几个小时。最头疼的是查证资料,动辄费去一两周时间,——我怕写错啊。即便什么都想好了,写出来的时候还是不免要修改好几次。不管怎么努力,我一小时也就是写400~600 字。高考作文要求600 字以上,费时1.5 小时,我还是停留在语文高考水平上。
我一算,不行啊,这每天写八小时也不到五千字,再加上有时候我没灵感,什么都写不出来,照这个速度我是要饿死街头啊。得了,网络文学这条路不适合我。该干嘛干嘛,老老实实写代码去吧。
一、寻求答案
我想寻找很多问题的答案。
为什么罗马可以从一个小城邦扩张成庞大的罗马帝国?为什么最文明的希腊没能建成希腊帝国?无敌舰队为什么会被击败?英国为什么可以成为日不落帝国?德国能否打赢两战?东方为什么只有日本崛起?韩国围棋为什么可以压制中日?春秋魏国能否一统天下?项羽为什么迅速被刘邦击败?汉武帝为什么越打越穷?司马懿怎样权倾朝野?蜀国为何北伐无功?唐朝为何衰落?宋朝为何没走上大航海之路?开封为何衰落?TG为何用三年就能击败国民党?社会发展的极限是什么?正义为什么一定会战胜邪恶?什么是正义的?……
大多数问题的答案在中学时代已经被灌输过了,但那些简单的解释远不能让我满意。所以,我开始独立寻求解答。

如果我发现博客、文章或者小说写得很有见地,我就会试图与作者联系,向他们提问。
然而,

除了读书和胡思乱想之外,

我们被灌输了很多既成答案,但是
二、建立自信
中学时代,我估计自己会成为数学家、物理学家或者工程师。如果有人说,你以后会写一本关于政治和历史的书,我会嗤之以鼻以为是嘲笑我。原因很简单,我的文科成绩太惨了。
我从小就喜欢看各种课外书,其中也包括政治类的。但我的大脑跟教育部不太合拍,总是
	
高一高二没把政史当回事,
高三知道要考大综合,我
我总觉得自己没想对。

随着“中子豆式三观”的逐渐成型,以前心中的那些牛人不再高高在上,我开始敢于批评所谓的“作家”。起点的网文,写的都是些什么玩意儿?故事架构抄来抄去,情节错漏百出、毫无新意,语言粗俗不堪,连标点符号都用不对。这也能叫小说?
在欧洲混了几年之后,越来越感受到西方出版业的不同。比如,写历史书不是简单的资料收集。由于历史资料庞杂繁多,作者要大量阅读之后,有选择的放入书中,而且要籍由自己选择的资料,表达个人的某些观点。创作了一些东西,这才叫“作者”。
说到资料的查证,尤能显示中国出版业的堕落。随便抓来一本西方的书,那里面的注解和引用都是规规矩矩。中国的书什么幺蛾子都可以有,宋鸿兵的《货币战争》充满了揣测和妄想,竟然能成为畅销书,充分反映了中国人的平均见识。
日前,高晓松做的谈话节目很火爆,然而他开头就说:“我这是跟观众随便聊聊,都是凭我的记忆,也就不去查证了。”我嘞个去这是什么态度?你能想象BBC 的《探索》频道说:“我们就是拍着玩玩,也就不管历史上实际发生什么了。”
这种差异还体现在网络上的免费资料中,维基百科的严肃认真不知超越百度百科多少倍。所以现在都说“维基百科可供参考,百度百科可供娱乐”。
中国那些个几天写一本书的,都不敢管自己叫“作者”,只敢叫“编者”。因为他们只是把各种扯得上一点关系的网络段子编到一块,别说抒发观点了,连去伪存真都做不到。要是我愿意,我也可以一天“编”一本人物传记。
是的,我永远也达不到他们的速度,因为我还不能把自己的格调降低到那个程度。
三、发表观点
写作本文,是为了发表我自己的观点。

许多事是需要信心作为支撑的。我的作文得分也不算高,
我擅长批评、讽刺、搞笑,不擅长赞美和抒情。政治和历史答题的时候,也经常“踩不到点子上”。
刚开始是看小说。小说中会描述很多既有问题,但小说中不会提供解决方案。
但是我喜欢寻根究底。在我
我思考很多问题,是因为我好奇。
由于长久以来,随意


隐藏在细节中的证据,如果没有正确的解读,只是一堆无价值的信息,而经过福尔摩斯的推理,就再现出了整个犯罪场景。对于一个历史学家来说,重要的不是“看到”了什么,而是“看出”了什么。历史记载就摆在那里,怎么去分析才是关键。
可惜,由于一些众所周知的原因,中国人研究历史是不敢乱说话的。这些年中国人写的历史书,越来越空洞无物。悲哀呀,悲哀。
我25岁之前,在人文学科方面一直不够自信。一方面因为我是学理科的,另一方面我中学政治和历史都一塌糊涂。大约25岁的时候,我开始相信我多年以来的许多思考一直是正确的。
我写这本书,是想发表一些观点。
本文中的所有观点,如无特别声明,均为本人独立思考得出。
注意,我是说这些观点是我独立思考出来的,并不是说,这是我首先创造的。世上能人有的是,很有可能某个人已经在我之前发表了相同或类似的观点,只不过我没看到罢了。
用词严谨
每一篇文章我都会反复审阅,确保没有歧义。如果你发现了错别字或者可能导致歧义、误解的文字,请来信告诉我。
我希望我的文字达到细节上完美无缺。举例,《宗教篇》初稿里有一句:“邪教会用各种方式伪装”。
阅读时一般会断句为:邪教 会用 各种 方式 伪装。
但是不排除有人看错成:邪 教会 用 各种 方式 伪装。
即使影响不大,仍然背离了我的本意,故此将其改为:邪教懂得用各种方式伪装。

即便是朱熹这样的大家,我们仍得从那令人头昏的各种各样记诸文字的言语、古籍的评注、写给友人的书信与其它零散文件中归纳出他的系统来。大师本人并没有一部总结性的论述。
本书中,你会发现大量的逐条归纳。甚至,我会追求用等长的汉字来进行表述。
至少在中学语文中,我归纳中心思想的水平差极了。
这不是因为我试图成为条分缕析的人,似乎我天生就是这样的人。
善恶篇
( 1 ) 道可道非常道
《道德经》开头一般是这样断句的:“道,可道,非常道;名,可名,非常名”。
但又有另一种断句:“道可道,非常道;名可名,非常名”。
接着我又看到了一个“半仙”的断句方式:“道可,道非,常道;名可,名非,常名”。
后来又看到了汉墓帛书版本:“道,可道也,非恒道也。名,可名也,非恒名也。”连文字都不一样了,但也证明了“道可,道非,常道;名可,名非,常名”的断句是错误的。

且不说本身含义如何,仅仅是断句的不同,语义就发生了巨大的变化。
所以,批判中国古籍是一件非常费劲的事。当我试图从一个角度驳斥其观点,就有人跳出来说“其实你理解错了”,当我换个角度驳斥,又有人跳出来说“你要考虑引申含义”。
与宗教界人士探讨宗教典籍也会遇到类似的问题。如果一个论述明显不符合现代科学,比如圣经里的“六天创造宇宙”,就被说成是比喻。
其实,模糊不清就是最大的问题。古籍的模糊不清很可能是书写者刻意营造的。古人并不傻,他们也懂得“如果不能把读者整明白,就彻底整糊涂”。

现代学术的发展,是在走向精确化,用无歧义的语言来构建理论体系。

\section{目标决定手段}


如果与中国古典派或者宗教学者有过辩论经历,就知道跟他们辩论是多么费劲的一件事。
我希望建立一个系统、精确、自洽的逻辑体系。

中国古人和宗教圣经的作者,
不能把你整明白,就把你整糊涂
我追求用词准确。
半真半假的论据 -> 歪曲的解释 -> 扯淡的结论
1. 有迹可寻
本文中涉及的所有资料,我都力求真实客观;所有的论据,我都会尽力找到引用;鉴于中文资料鱼龙混杂、以讹传讹,如果我感觉到中文资料不太准的话,我就会查阅英文或德文资料。
我本来想一步步找齐所有的证据,然后再把一本完善的文集奉献给大家。但那样的话写作进度实在是太慢了。所以,我采取一个折中的方案:我会首先对证据快速检索,如果看起来问题不大,就写上去,具体的查证留待以后——也许是我退休后。
即使是英文的维基百科都不能算作严格的数据源,百度百科就更不能算了。“维基百科可供参考,百度百科可供娱乐”。
本文引用的所有书目,以后都会在附录中列出。
“引用”的重要性毋庸敷言,这已经是现代学术的基本原则。中国人不重视引用,是因为引用会打击盗版、增加工作量,也会使得自己的学说很容易被驳倒。


本文涉及的书目,脚注中只有书名,附录中有关于此书的详细信息。
我的阅读量还不错,每周都会找本新的书看,这还不算平时看的大量网页。
比如杜黑的《制空权》,我一个小时就读完了。
我可能不太擅长
比如两本烂书,《狼图腾》和《货币战争》,连一点引用都没有。

有很多引用,因我记不清具体在哪里看过
我看过很多电影,但几乎不认识什么导演和演员。这是我的风格:关注主要内容,忽视其它信息。就如同我去购物时直奔目标,连
关于排版
中国的出版物,普遍来说排版是很烂的。本文的排版是我自己做的。我虽然没什么艺术细胞,但看了些关于排版的文章,算是有点心得,希望这个排版能赢得大家的喜爱。
我用PDF 发布此文,是因为PDF 可以做出比较漂亮的格式,而且我不喜欢别人修改我的思想。
目前,手机阅读大行其道。经我自己测试,目前的排版还是比较适合手机阅读的。但是,这个版式的页边距非常小,不适合打印。如果有人想要打印的话,请给我发邮件,我会考虑再出一个适合打印的版本。
页面布局15 x 24cm,页边距0.3cm
基础字号15,每行27个字

比如太平天国、克里米亚战争、美国南北战争都发生在19世纪中期,他们的发生顺序我经常搞迷糊。
如果觉得好,请传播

\chapter{缩写表}

\begin{description}
\item[TG] “土共”的首字母,中国共产党的简称,图形上T 像锤子,G 像镰刀,所以也有另一个俗称:锤镰帮。
\item[天朝] 中国的别称。
\item[红朝 后清] 特指中华人民共和国。
\item[改开] 改革开放。
\item[真人] 赵紫阳,因“紫阳真人”听起来很霸气而得名。
\item[毛] 特指毛泽东,算是中性称呼。常见的褒义称呼为“太祖”。常见的贬义称呼为“腊肉”、“尸王”、“僵尸”、“粽子”,皆因其尸体保存方式而得名。
\item[江、胡、习] 分别指连续三任中国国家主席。
\item[蘑菇] 指核弹,因爆炸产生的蘑菇云而得名。“大蘑菇”又特指氢弹。“种蘑菇”指释放核弹。
\item[灯塔国] 美国,又称美帝。
\item[油霸] 中东石油富国。
\item[真理部] 因《1984》而著名,泛指一切压制言论和新闻出版自由的政府部门。
\item[洗地] 为某种坏事、错事进行遮掩。原见于电影《功夫》,斧头帮头目杀人后,血溅四地,头目高喊了一声“警察,快来洗地了”后,方才施施然离去。
\item[五毛] 泛指在网上支持TG 的网民。2006年,曝出一份文件说:招募“网络评论员”,引导网上言论,每个贴子五毛。故而得名。
\item[smaug] “四毛狗”的谐音。形容此人为TG 洗地能力太差,达不到五毛的基准线,每贴只能拿四毛。
\item[简网 繁网] 汉语简化字网络,汉语繁体字网络。
\end{description}

